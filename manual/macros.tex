\label{macros}
\subsection*{Introduction}

Macros are simple executable files containing standard UNIX
commands.  A number of the are supplied with \D and are found in
the {\em execute} sub-directory\index{sub-directory}.  These are
not guaranteed to be immaculate but with little adaptation they
can become a useful tool to a researcher.  The available macros
are as follows:

{\sl
\begin{itemize}
\item cleanup
\item copy
\item gopoly
\item gui
\item select
\item store
\end{itemize}
}

/noindent The function of each of these is described below. It is
worth noting that most of these functions can be performed by the
DL\_POLY Java GUI \cite{smith-gui}.

\subsubsection*{{\sl cleanup}}

{\sl cleanup} removes several standard data files from the {\em
execute} sub-directory\index{sub-directory}.  It contains the UNIX
commands:
\begin{verbatim}
rm OUTPUT STATIS REVCON REVOLD REVIVE RDFDAT ZDNDAT DEFECTS gopoly.*
\end{verbatim}
\noindent and removes the files OUTPUT, REVCON, REVOLD, STATIS,
REVIVE, DEFECTS and gopoly.* (all variants).  It is useful for
cleaning the sub-directory\index{sub-directory} up after a run.
(Useful data should be stored elsewhere however!)

\subsubsection*{{\sl copy}}

{\sl copy} invokes the UNIX commands:

\begin{verbatim}
mv CONFIG CONFIG.OLD
mv REVCON CONFIG
mv REVIVE REVOLD
\end{verbatim}

\noindent which collectively prepare the \D files in the {\em
execute} sub-directory\index{sub-directory} for the continuation of
a simulation.  It is always a good idea to store these files
elsewhere in addition to using this macro.

\subsubsection*{{\sl gopoly}}

{\sl gopoly} is used to submit a \D job to the HPC$x$, which
operates a LOAD-LEVELER job queuing system.  It invokes the
following script:

\begin{verbatim}
#@ shell = /usr/bin/tcsh
#
#@ job_type = parallel
#@ job_name = gopoly
#
#@ cpus = 32
#
#@ node_usage = not_shared
#@ network.MPI = csss,shared,US
#
#@ wall_clock_limit = 00:30:00
#@ account_no = my_account
#
#@ output = $(job_name).$(schedd_host).$(jobid).out
#@ error  = $(job_name).$(schedd_host).$(jobid).err
#@ notification = never
#
#@ bulkxfer = yes
#@ data_limit = 850000000
#@ stack_limit = 10000000
#
#@ queue
#
# ENVIRONMENT SETTINGS
#
setenv MP_EAGER_LIMIT 65536
setenv MP_SHARED_MEMORY yes
setenv MEMORY_AFFINITY MCM
setenv MP_TASK_AFFINITY MCM
setenv MP_SINGLE_THREAD yes
#
poe  ./DLPOLY.Z
\end{verbatim}

\noindent Using LOADLEVELLER, the job is submitted by the UNIX command: \\
\\
{\sl llsubmit} {\sl gopoly} \\
\\
\noindent where {\sl llsubmit} is a local command for submission
to the IBM SP4 cluster.  The number of required nodes and the job
time are indicated in the above script.

\subsubsection*{\sl gui}

{\sl gui} is a macro that starts up the \D Java GUI.  It invokes
the following UNIX commands:

\begin{verbatim}
java -jar ../java/GUI.jar $1 &
\end{verbatim}

\noindent In other words the macro invokes the Java Virtual Machine
which executes the instructions in the Java archive file GUI.jar,
which is stored in the {\em java} subdirectory of \D. (Note: Java
1.3.0 or a higher version is required to run the GUI.)

\subsubsection*{{\sl select}}

{\sl select} is a macro enabling easy selection of one of the test
cases.  It invokes the UNIX commands:

\begin{verbatim}
cp ../data/TEST$1/CONTROL   CONTROL
cp ../data/TEST$1/FIELD     FIELD
cp ../data/TEST$1/CONFIG    CONFIG
cp ../data/TEST$1/TABLE     TABLE
cp ../data/TEST$1/TABEAM    TABEAM
cp ../data/TEST$1/REFERENCE REFERENCE
\end{verbatim}

\noindent {\sl select} requires one argument (an integer) to be specified: \\
\\
{\sl select n} \\
\\
\noindent where {\sl n} is test case number, which ranges from 1 to
18.

This macro sets up the required input files in the {\em execute}
sub-directory\index{sub-directory} to run the {\sl n}-th test
case.  The last three copy commands may not be necessary in
most cases.

\subsubsection*{{\sl store}}

The {\sl store} macro provides a convenient way of moving data
back from the {\em execute} sub-directory\index{sub-directory} to
the {\em data} sub-directory\index{sub-directory}.  It invokes the
UNIX commands:

\begin{verbatim}
mkdir        ../data/TEST$1
cp CONTROL   ../data/TEST$1/CONTROL
cp FIELD     ../data/TEST$1/FIELD
cp CONFIG    ../data/TEST$1/CONFIG
cp TABLE     ../data/TEST$1/TABLE
cp TABEAM    ../data/TEST$1/TABEAM
cp REFERENCE ../data/TEST$1/REFERENCE
mv OUTPUT    ../data/TEST$1/OUTPUT
mv STATIS    ../data/TEST$1/STATIS
mv REVCON    ../data/TEST$1/REVCON
mv REVIVE    ../data/TEST$1/REVIVE
mv HISTORY   ../data/TEST$1/HISTORY
mv DEFECTS   ../data/TEST$1/DEFECTS
mv RDFDAT    ../data/TEST$1/RDFDAT
mv ZDNDAT    ../data/TEST$1/ZDNDAT
chmod -R a-w ../data/TEST$1
\end{verbatim}

\noindent which first creates a new DL\_POLY {\em data/TEST..}
sub-directory\index{sub-directory} and then moves the standard \D
output data files into it.

{\sl store} requires one argument:\\
\\
{\sl store n}\\
\\
\noindent where {\sl n} is a unique string or number to label the
output data in the {\em data/TESTn} sub-directory.

Note that {\sl store} sets the file access to read-only.  This is
to prevent the {\sl store} macro overwriting existing data without
your knowledge.
