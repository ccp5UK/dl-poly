\subsection{Choosing Ewald Sum Variables}
\label{ewald-precision}

\subsubsection{Ewald sum and SPME}

This section outlines how to optimise the accuracy of the Smoothed
Particle Mesh Ewald\index{Ewald!SPME} sum parameters for a given
simulation..

As a guide\index{Ewald!optimisation} to beginners \D will calculate
reasonable parameters if the {\bf ewald~precision} directive is used
in the CONTROL file (see Section~\ref{control-file}).  A relative
error (see below) of 10$^{-6}$ is normally sufficient so the
directive

\vskip 1ex
\noindent {\bf ewald precision 1d-6}
\vskip 1ex

\noindent will make \D evaluate its best guess at the Ewald
parameters $\alpha$, {\tt kmaxa}, {\tt kmaxb} and {\tt kmaxc},
or their doubles if {\bf ewald} rather than {\bf spme} is specified.
(The user should note that this represents an {\em estimate}, and
there are sometimes circumstances where the estimate can be
improved upon.  This is especially the case when the system
contains a strong directional anisotropy, such as a surface.)
These four parameters may also be set explicitly by the
{\bf ewald~sum} directive in the CONTROL file.  For example the
directive

\vskip 1ex
\noindent {\bf ewald sum 0.35 6 6 8}
\vskip 1ex

\noindent which is equvalent to

\vskip 1ex
\noindent {\bf spme sum 0.35 12 12 16}
\vskip 1ex

\noindent would set $\alpha=0.35$~\AA$^{-1}$, ${\tt kmaxa}=12$, ${\tt
kmaxb}=12$ and ${\tt kmaxc}=16$\footnote{{\bf Important note}:
As the SPME method substitues the standard Ewald the values of
{\tt kmaxa}, {\tt kmaxb} and {\tt kmaxc} are the double of those in
the prescription of the standard Ewald since they specify the sides
of a cube, not a radius of convergence.}. The quickest check on the
accuracy of the Ewald sum\index{Ewald!summation} is to compare the
coulombic energy ($U$) and virial ($\cal W$) in a short simulation.
Adherence to the relationship $U=-{\cal W}$, shows the extent to
which the Ewald sum\index{Ewald!summation} is correctly converged.
These variables can be found under the columns headed {\tt eng\_cou}
and {\tt vir\_cou} in the OUTPUT file (see Section~\ref{output-file}).

The remainder of this section explains the meanings of these
parameters and how they can be chosen.  The Ewald
sum\index{Ewald!summation} can only be used in a three dimensional
periodic system.  There are five variables that control the
accuracy: $\alpha$, the Ewald convergence parameter; $r_{\rm cut}$
the real space force cutoff; and the {\tt kmaxa}, {\tt kmaxb} and
{\tt kmaxc} integers that specify the dimensions of the SPME
charge array (as well as FFT arrays).  The three integers
effectively define the range of the reciprocal space sum
(one integer for each of the three axis directions).  These
variables are not independent, and it is usual to regard one of
them as pre-determined and adjust the others accordingly.  In this
treatment we assume that $r_{\rm cut}$ (defined by the {\bf cutoff}
directive in the CONTROL file) is fixed for the given system.

The Ewald sum splits the (electrostatic) sum for the infinite,
periodic, system into a damped real space sum and a reciprocal
space sum.  The rate of convergence of both sums is governed by
$\alpha$.  Evaluation of the real space sum is truncated at
$r=r_{\rm cut}$ so it is important that $\alpha$ be chosen so that
contributions to the real space sum are negligible for terms with
$r>r_{\rm cut}$.  The relative error ($\epsilon$) in the real
space sum truncated at $r_{\rm cut}$ is given approximately
by\index{Ewald!optimisation}
\begin{equation}
\epsilon \approx {\rm erfc}(\alpha~r_{\rm cut})/r_{\rm cut}
\approx \exp[-(\alpha~r_{\rm cut})^{2}]/r_{\rm cut}~~, \label{relerr}
\end{equation}
which reciprocally gives an estimate for $\alpha$ for a given $\epsilon$:
\begin{equation}
\alpha \approx \frac{\sqrt{|{\rm ln}(\epsilon~r_{\rm cut})|}}{r_{\rm cut}}~~.
\end{equation}

The recommended value for $\alpha$ is $3.2/r_{\rm cut}$ or greater
(too large a value will make the reciprocal space sum very slowly
convergent).  This gives a relative error in the energy of no
greater than $\epsilon = 4 \times 10^{-5}$ in the real space sum.
When using the directive {\bf ewald~precision} \D makes use of a
more sophisticated approximation:
\begin{equation}
{\rm erfc}(x) \approx 0.56 \; \exp(-x^{2})/x
\end{equation}
to solve recursively for $\alpha$, using equation~\ref{relerr} to give
the first guess.

The relative error in the reciprocal space term is approximately
\begin{equation}
\epsilon \approx \exp(- k_{max}^{2}/4\alpha^{2})/k_{max}^{2}
\end{equation}
where
\begin{equation}
k_{max} = \frac{2\pi}{L}~\frac{\tt kmax}{2}
\end{equation}
is largest $k$-vector considered in reciprocal space, $L$ is the
width of the cell in the specified direction and {\tt kmax} is an integer.

For a relative error of $4 \times 10^{-5}$ this means using
$k_{max} \approx 6.2~\alpha$.  {\tt kmax} is then
\begin{equation}
{\tt kmax} > 6.4~L/r_{\rm cut}.
\end {equation}

In a cubic system, $r_{\rm cut}=L/2$ implies ${\tt kmax}=14$.
In practice the above equation slightly over estimates the value
of {\tt kmax} required, so optimal values need to be found
experimentally.  In the above example ${\tt kmax}=10$ or $12$
would be adequate.

If you wish to set the Ewald parameters manually (via the
{\bf ewald sum} or {\bf spme sum} directives) the recommended
approach is as follows.  Preselect the value of $r_{\rm cut}$,
choose a working a value of $\alpha$ of about 3.2/$r_{\rm cut}$
and a large value for the {\tt kmax} (say 20 20 20 or more).
Then do a series of ten or so {\em single} step simulations with
your initial configuration and with $\alpha$ ranging over the
value you have chosen plus and minus 20\%.  Plot the Coulombic
energy (-$\cal W$) versus $\alpha$.  If the Ewald sum is correctly
converged you will see a plateau in the plot.  Divergence from the
plateau at small $\alpha$ is due to non-convergence in the real
space sum.  Divergence from the plateau at large $\alpha$ is due to
non-convergence of the reciprocal space sum.  Redo the series of
calculations using smaller {\tt kmax} values.  The optimum values
for {\tt kmax} are the smallest values that reproduce the correct
Coulombic energy (the plateau value) and virial at the value of
$\alpha$ to be used in the simulation.  Note that one needs to
specify the three integers ({\tt kmaxa}, {\tt kmaxb}, {\tt kmaxc})
referring to the three spatial directions, to ensure the reciprocal
space sum is equally accurate in all directions.  The values of
{\tt kmaxa}, {\tt kmaxb} and {\tt kmaxc} must be commensurate with
the cell geometry to ensure the same minimum wavelength is used in
all directions.  For a cubic cell set {\tt kmaxa} = {\tt kmaxb} = {\tt kmaxc}.
However, for example, in a cell with dimensions 2A = 2B = C,
(ie. a tetragonal cell, longer in the c direction than the a and b directions)
use 2{\tt kmaxa} = 2{\tt kmaxb} = {\tt kmaxc}.

If the values for the kmax used are too small, the Ewald sum will
produce spurious results.  If values that are too large are used,
the results will be correct but the calculation will consume unnecessary
amounts of cpu time. The amount of cpu time increases proportionally
to ${\tt kmaxa}  \times {\tt kmaxb} \times {\tt kmaxc}$.

It is worth noting that the working values of the k-vectors may be larger
than their original values depending on the actual processor decomposition.
This is to satisfy the requirement that the k-vector/FFT transform down
each direction per domain is a multiple of 2, 3 and 5 only, which is due
to the GPFA code (single 1D FFT) which the DaFT implementation relies on.
This allowes for greater flexiblity than the power of 2 multiple
restriction in \D predicessor, DL\_POLY\_3.  As a consequence, however,
execution on different processor decompositions may lead to different
working lengths of the k-vectors/FFT transforms and therefore slightly
different SPME forces/energies whithin the same level of SPME/Ewald
precision/accuracy specified.  {\bf Note} that although the number of
processors along a dimension of the DD grid may be any number, numbers
that have a large prime as a factor will lead to inefficient performance!
