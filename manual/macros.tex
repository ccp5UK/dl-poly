\label{macros}
\subsection*{Introduction}

Macros are simple executable files containing standard UNIX
commands.  A number of the are supplied with \D and are found in
the {\em execute} sub-directory\index{sub-directory}.  These are
not guaranteed to be immaculate but with little adaptation they
can become a useful tool to a researcher.  The available macros
are as follows:

{\sl
\begin{itemize}
\item cleanup
\item copy
\item gopoly
\item gui
\item select
\item store
\end{itemize}
}

\noindent The function of each of these is described below.  It is
worth noting that most of these functions could be performed by the
DL\_POLY Java GUI \cite{smith-gui}.

\subsubsection*{{\sl cleanup}}

{\sl cleanup} removes several standard data files from the {\em
execute} sub-directory\index{sub-directory}.  It contains the UNIX
commands:

\lstinputlisting{../execute/cleanup}

\noindent  It is useful for cleaning the sub-directory\index{sub-directory}
up after a run. (Useful data should be stored elsewhere however!)

\subsubsection*{{\sl copy}}

{\sl copy} invokes the UNIX commands:

\lstinputlisting{../execute/copy}

\noindent which collectively prepare the \D files in the {\em execute}
sub-directory\index{sub-directory} for the continuation of a simulation.
It is always a good idea to store these files elsewhere in addition to using this macro.

\subsubsection*{{\sl gopoly}}

{\sl gopoly} is used to submit a \D job to the HPC$x$, which operates
a LOAD-LEVELER job queuing system.  It invokes the following script:

\lstinputlisting{../execute/gopoly}

\noindent Using LOADLEVELLER, the job is submitted by the UNIX command: \\
\\
{\sl llsubmit gopoly} \\
\\
\noindent where {\sl llsubmit} is a local command for submission
to the IBM SP4 cluster.  The number of required nodes and the job
time are indicated in the above script.

\subsubsection*{\sl gui}

{\sl gui} is a macro that starts up the \D Java GUI.  It invokes
the following UNIX commands:

\lstinputlisting{../execute/gui}

\noindent In other words the macro invokes the Java Virtual Machine
which executes the instructions in the Java archive file GUI.jar,
which is stored in the {\em java} subdirectory of \D. (Note: Java
1.3.0 or a higher version is required to run the GUI.)

\subsubsection*{{\sl select}}

{\sl select} is a macro enabling easy selection of one of the test
cases.  It invokes the UNIX commands:

\lstinputlisting{../execute/select}

\noindent {\sl select} requires one argument (an integer) to be specified: \\
\\
{\sl select n} \\
\\
\noindent where {\sl n} is test case number, which ranges from 1 to
18.

This macro sets up the required input files in the {\em execute}
sub-directory\index{sub-directory} to run the {\sl n}-th test
case.  The last three copy commands may not be necessary in most cases.

\subsubsection*{{\sl store}}

The {\sl store} macro provides a convenient way of moving data back
from the {\em execute} sub-directory\index{sub-directory} to the
{\em data} sub-directory\index{sub-directory}.  It invokes the UNIX commands:

\lstinputlisting{../execute/store}

\noindent which first creates a new DL\_POLY {\em data/TEST..}
sub-directory\index{sub-directory} and then moves the standard \D
output data files into it.

{\sl store} requires one argument:\\
\\
{\sl store n}\\
\\
\noindent where {\sl n} is a unique string or number to label the
output data in the {\em data/TESTn} sub-directory.

Note that {\sl store} sets the file access to read-only.  This is
to prevent the {\sl store} macro overwriting existing data without
your knowledge.
