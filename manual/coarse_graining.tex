\section{User-Defined Coarse-Grain Models with Tabulated Force-Fields}
\label{cg-intro}

One can use \D for preparing and running simulations of (numerically)
coarse-grained (CG) models by using tabulated effective force-fields
(FF) derived from either {\bf (i)} potentials of mean force (PMF);
or {\bf (ii)} iteratively optimised CG models.

In outline, systematic coarse-graining (SCG) of an atomistic system
implies the application of a geometrical projection, or ``mapping'',
of the original system onto a considerably reduced set of degrees of freedom (DoF),
whence referred to as a ``coarse-grained'' model.  The procedure is to recover
the average configurational (often termed ``physical'') and topological
(often termed ``chemical'') force-field related properties of the original
full-atom (FA) model.  Integrating out degrees of freedom ultimately
leads to loss of information.  However, if this is done selectively and
consistently, the intrinsic information pertaining to the dominating
interactions within the FA system (that govern its behaviour towards
phenomena of interest) and the most important thermodynamic properties
are retained.  Then the greatly reduced phase-space of the CG model system
allows for efficient access to much longer length and time scales for
investigating microscopic phenomena using of the well-established
classical MD machinery.  The reduced DoF mapping leads to the generation
of an effective CG FF in terms of the effective interaction potentials
and forces between the CG particles, which are often tabulated numerically.

The initial coarse-grain mapping of the original FA trajectory can be done
with the aid of DL\_CGMAP tool -- \href{http://www.ccp5.ac.uk/software/}
{http://www.ccp5.ac.uk/software/}\index{WWW}. After the CG mapped trajectory
has been obtained, the relevant distribution analysis and the Boltzmann
Inversion procedure (producing tabulated PMF:s) can be performed by using \D
in ``replay history'' mode, with the corresponding directives in CONTROL file.
For further iterative optimisation of the CG model the user is advised
to use \D as simulation engine within the framework of VOTCA package --
\href{http://www.votca.org/}{http://www.votca.org/}\index{WWW}, which
provides a handful of various systematic coarse-graining methodologies.
For more details the user should refer to the manuals of these tools.

This section describes how to use \D for two SCG tasks:

\begin{itemize}
\item Post-simulation analysis of the intramolecular (bonded and angular)
mean-force interactions, based on the calculation of the corresponding
intramolecular distributions.
\item Preparation and use of the tabulated intramolecular potentials.
\end{itemize}

\noindent {\bf Note:} Although these steps are also applicable to
atomistic systems, which can be useful for benchmarking purposes,
below we shall assume that the CG mapping has been done and the following
data files have been generated for the CG mapped model system:
CONTROL, FIELD, CONFIG, and, alternatively, HISTORY.

\section{Intramolecular Probability Distribution Function (PDF) Analysis}
\label{IPDF-analysis}

Albeit the distribution analysis can be performed on the fly,
in the course of a simulation run, for CG purposes it has
to be done on an existing CG mapped trajectory, by invoking
the {\bf replay history} and {\bf analysis} directives,
see Section~\ref{control-file}.
To trigger the PDF collection and subsequent output of PMF:s
for any of the following intramolecular interactions: bonds,
angles, dihedrals and/or inversions, the CONTROL file must
contain any combination of the options demonstrated
in the example below.  {\bf Note} that such analyses will
only be carried out if the desired intramolecular types of
interactions are defined within FIELD!

\begin{lstlisting}
TITLE: EXAMPLE OF DL_POLY_4 PDF ANALYSIS DIRECTIVES SNIPPET

# DIRECTIVES TO INVOKE INTRAMOLECULAR PDF ANALYSIS BY TYPE
analyse  bonds       sample every 100  nbins 250  rmax 5.0
analyse  angles      sample every 100  nbins 360  # [ 0 : pi]
analyse  dihedrals   sample every 100  nbins 720  # [-pi: pi]
analyse  inversions  sample every 100  nbins 360  # [ 0 : pi]

# DIRECTIVES TO INVOKE INTRAMOLECULAR PDF ANALYSIS FOR ALL TYPES
analyse  all         sample every 100  nbins 1000  rmax 5.0

# DIRECTIVES TO INVOKE PRINTING FOR ANY DEFINED
# INTER-(RDF->VDW) & INTRA-(bonded) MOLECULAR PDF ANALYSIS
print analysis
\end{lstlisting}

The {\bf analyse} directive acts in a similar manner to that
of the {\bf rdf} directive outlining {\bf (i)} the frequency
of sampling (in steps/frames) and {\bf (ii)} the number of
bins over the cutoff interval of the specified interaction.
It is only for bonded pairs that this cutoff needs specifying,
whereas for angles the possible ranges are known {\em a priory}
by definition. If no cutoff is supplied for bonds, then it
defaults to 2~\AA.  It is worth noting that, for the sake of
accuracy, the number of bins per PDF must be larger than or equal
to $N_{\rm min} = {\tt Nint} \left( \frac{\rm PDF~cutoff}{\Delta_{\rm max}} \right)$,
where the max bin size, $\Delta_{\rm max}$, is defined internally
for each type of PDF (see {\tt setup\_module.f90}).  Otherwise,
it will default to ${\tt max}(N_{\rm min},N_{\rm tab})$,
where $N_{\rm tab}$ is the number of bins on the grid defined
in the corresponding tabulated force-field data file (TAB*),
if it is provided, otherwise $N_{\rm tab}=0$.
In particular, for bonds $\Delta_{\rm max}$ is 0.01~\AA~and
for angles it is $0.2 \pi/180$, i.e. 0.2~degree.

If {\bf analyse all} option is used in conjunction with any specific
directive, then it triggers the analysis on all PDFs while enforcing
on the individually targeted PDF(s) the following parameters: {\bf (i)}
its sampling interval (frames) -- only if it is smaller than,
and {\bf (ii)} its grid number -- only if it is larger than
those specified for the targeted PDFs.
In the case of bonds it will also enforce the grid range {\bf rmax} --
only if it is larger than that specified in the {\bf analyse bonds}
directive. Hence, the {\bf analyse all} directive allows to
quickly override and/or unify the sampling frequencies and
max grid numbers for all PDFs, provided its parameters improve
on the accuracy of the collected data (compared to the specifications
for the individually targeted PDF:s).

While the statistics are always collected and stored for the
future use in the (binary) REVIVE file for all targeted PDF:s
(see Section~\ref{revive-file}), using the {\bf print analysis}
directive will instruct \D to additionally print the data in
the OUTPUT and *DAT files, the latter containing each {\it type}
of PDF:s separately (the asterisk stands for one of the following: BND,
ANG, DIH, or INV).  As a result, apart from OUTPUT, three data files
will be created for each of the targeted distribution types:
BNDDAT, BNDPMF~\&~BNDTAB -- for bonds, ANGDAT, ANGPMF~\&~ANGTAB
-- for angles, DIHDAT, DIHPMF~\&~DIHTAB -- for dihedrals, and INVDAT,
INVPMF~\&~INVTAB -- for inversions; the *PMF and *TAB files containing
tabulated data for the respective potential of mean force and
force/virial (*TAB files bearing the data from *PMF files but
{\em resampled onto a finer grid}; see the last paragraph
in this section).

Partial examples of the *DAT files for bonds and angles are given below.

\begin{lstlisting}
[user@host]$ more BNDDAT
# TITLE: Hexane FA OPLSAA -> CG mapped with 3 beads (A-B-A)
# BONDS: Probability Density Functions (PDF) := histogram(bin)/hist_sum(bins)/dr_bin
# bins, cutoff, frames, types:        250    5.000       2285          1
#
# r(Angstroms)  PDF_norm(r)  PDF_norm(r)/dVol(r)   @   dr_bin =  0.02000
#

# type, index, instances: A        B                 1       2000
    0.01000  0.000000E+00  0.000000E+00
    0.03000  0.000000E+00  0.000000E+00
    0.05000  0.000000E+00  0.000000E+00
...
    4.95000  0.000000E+00  0.000000E+00
    4.97000  0.000000E+00  0.000000E+00
    4.99000  0.000000E+00  0.000000E+00

[user@host]$ more ANGDAT
# TITLE: Hexane FA OPLSAA -> CG mapped with 3 beads (A-B-A)
# ANGLES: Probability Density Functions (PDF) := histogram(bin)/hist_sum(bins)/dTheta_bin
# bins, cutoff, frames, types:        360        180       2285          1
#
# Theta(degrees)  PDF_norm(Theta)  PDF_norm(Theta)/sin(Theta)   @   dTheta_bin =  0.50000
#

# type, index, instances: A        B        A                 1       1000
    0.25000  0.000000E+00  0.000000E+00
    0.75000  0.000000E+00  0.000000E+00
    1.25000  0.000000E+00  0.000000E+00
...
  178.75000  1.380569E-02  6.328564E-01
  179.25000  8.368490E-03  6.393238E-01
  179.75000  2.901532E-03  6.649842E-01
\end{lstlisting}

One can see that all the header lines are commented out due
to starting with the hash symbol, {\bf \#}.  Nonetheless, the
header contains some useful information.  The title is, as usual,
placed in the first line, which is followed by an explanatory
line with the definition of a normalised PDF.
The third line provides the four most important descriptors:
the number of {\em bins} on the histogram grid, the {\em cutoff}
interval (absolute value of the span) over which the distributions
are sampled, the number of {\em frames} (samples) used, and
the number of unique unit {\em types} analysed (where ``unit''
is one of the following: bonds, angles, dihedrals or inversions).
The last explanatory line in the header, found in between
two empty commented-out lines, defines the meaning of
the columns and, at the end, the grid bin size.

The PDF histograms within a *DAT file are separated by uncommented
empty lines, which makes it possible to directly import and plot
all the data as separate lines in [Xm]Grace 2D plotter.  For clarity,
the data of each histogram are preceded by a commented-out line
specifying the {\em type}, {\em index} and number of {\em instances}
of the analysed interaction unit (the latter being counted over
the entire system).

In all *DAT files the first column bears the {\em bin-centered}
abscissa values (distance or angle), the second column is the distribution
histogram normalized to unity, i.e. its {\em integral} equals 1, whereas
the third column, if any, contains the PDF data corrected for the volumetric
(entropic) degeneracy of the grid points.  Thus, it is the data of the last
column found that are used for calculating PMF $\sim -\ln$(PDF).

The OUTPUT file will contain copies of the first and second
columns of all collected PDF:s, but normalised so that the figures
in the second column {\em sum up} to 1, which can be checked
by examining the third column as it bears the running sum of
a PDF histogram.

In addition to the PDF:s, \D also calculates the respective PMF:s and
pairwise force dependencies (virial for bonds), which are stored in
the *PMF and, upon resampling onto a {\em bin-edge} grid, *TAB files.
{\bf It is important to note} that, unlike the *PMF files containing
the bare $-\ln$(PDF) data (converted to the requested energy units)
{\em on the same grid as the PDF histograms}, the force-field data in *TAB
files are {\em resampled} onto $\max(N_{\rm min},N_{\rm tab})$ grid points
located at the bin edges (as expected by \D when reading the potential
and force tables), where $N_{\rm tab}$ is the grid number for
the respective intramolecular unit type read-in from its TAB* file,
if provided (otherwise $N_{\rm tab}=0$). Thus, the *TAB files obey
the \D format for numerically defined intramolecular force-field tables
(TAB*, see below) and, hence, can be directly used as such upon renaming:
BNDTAB~$\to$~TABBND, ANGTAB~$\to$~TABANG, DIHTAB~$\to$~TABDIH and INVTAB~$\to$~TABINV.
{\bf The user is, however, strongly advised to check the quality
of the obtained tabulated force-fields before using those as input
for a CG simulation.}  Albeit \D uses a simple smoothing algorithm
for noise-reduction in PMF:s and implements capping of the forces
in the regions of zero-valued PDF:s, in undersampled regions the PDF
and PMF data are likely to suffer from inaccuracy and increased noise
which, most often, require extra attention and re-fitting manually.

The general format of the above discussed files is shown in Section~\ref{bonded-files}

\section{Setting up Tabulated Intramolecular Force-Field Files}
\label{bonded-tables}

For a user-defined, e.g. coarse-grained, model system
the effective potentials must be provided in a tabulated form.
For non-bonded short-range (VdW) interactions the TABLE file
must be prepared as described in Section~\ref{table-file}.
However, the tabulated data format for intramolecular
interactions (bonds, bending angles, dihedral and inversion
angles in a polymer) differs from that of the TABLE file and
assumes three columns: abscissa (distance in \AA~or angle
in degrees), and two ordinates: potential and force data
(virial for distance dependent interactions -- e.g. bonds,
and force for angle dependent interactions -- e.g. angles).
Shown below are examples of TABBND and TABANG files,
corresponding to the above PDF examples. {\bf Note} that
the PMF and force data have been resampled onto a finer grid
with points located {\em at bin edges}.

\begin{lstlisting}
[user@host]$ more TABBND
# TITLE: Hexane FA OPLSAA -> CG mapped with 3 beads (A-B-A)
# 5.0 500

# A B
 1.00000e-02  -9.0906600e+02   1.3954000e+00
 2.00000e-02  -9.1046200e+02   2.7908000e+00
 3.00000e-02  -9.1185700e+02   4.1862000e+00
...

[user@host]$ more TABANG
# TITLE: Hexane FA OPLSAA -> CG mapped with 3 beads (A-B-A)
# 1000

# A B A
 1.80000e-01   8.8720627e+01   6.9119576e-01
 3.60000e-01   8.8596227e+01   6.9119227e-01
 5.40000e-01   8.8471827e+01   6.9118704e-01
...
\end{lstlisting}

The input tables for bonds, angles, dihedrals and inversions are
named TABBND, TABANG, TABDIH and TABINV, correspondingly.
The format of these files is fixed in terms of the line,
or {\em record}, order.  In particular, the initial two
header lines must contain a title and a record with the grid
specification, and each of the following blocks of tabulated
data must be preceded by an empty line and a one-line
descriptor record containing the (white-space delimited) names
of the atoms making up the given intermolecular interaction
unit (same as a {\em unit type} in *DAT files).  These descriptor
lines can be commented out or not, i.e. having {\bf \#} as the
first symbol would not affect the reading operation, but it would
ease importing of the data for plotting and manipulating
in [Xm]Grace software.

In all TAB* files the number of grid points (bins) must be
specified on the second line (commented-out or not).  For
angles (TABANG, TABDIH, TABINV) no other information needs
to be provided as their ranges are pre-determined:
$0 < {\Theta\ \rm in\ TABANG\ \&\ TABINV} \le 180$
and  $-180 < {\Theta\ \rm in\ TABDIH} \le 180$. In the TABBND
file, however, the ``bond cutoff'', $r_{\rm max}$ (\AA),
must be precede the grid number.

{\bf Note} that all potential and force data are to be provided
in the same energy units as specified by the user in the FIELD
file (see Section~\ref{field-file}) with distances in \AA~and
angles in degrees.  All the data related to angles are internally
transformed and handled by \D with angles measured in {\em radians}.

Finally, in order to instruct \D to use tabulated intramolecular
force-fields read from the TAB* files the user has to specify
in the FIELD file the keyword {\bf tab} or {\bf -tab} for each
intramolecular interaction thereby chosen for tabulation (similarly
to how it is described in Section~\ref{field-file}).
The dash symbol (-) in front of the keyword {\bf tab} is only valid
for bonds and angles, and is interpreted in the same manner as in
Table~\ref{bond-table} and Table~\ref{angle-table}.

{\bf Note} that VOTCA package is also capable of collecting both
intra- and inter-molecular stats and producing correct TAB* files,
provided the FIELD and HISTORY files exist (albeit VOTCA saves
the distributions in a format different from *DAT files).

Below we summarise the sequence of operations the user has to
follow in order to perform the CG distribution analysis and
prepare the TAB* files for a newly coarse-grained system.

\begin{itemize}
\item Perform CG mapping of the original FA system with the aid
of DL\_CGMAP or VOTCA (in the case of using VOTCA follow its manual;
the remainder of the list describes using \D only);
\item Move the data files for the newly CG-mapped system
(FIELD\_CG, CONFIG\_CG, HISTORY\_CG) into a separate directory
under the standard names (FIELD, CONFIG, HISTORY) and create
the corresponding CONTROL file containing the {\bf analysis}
and {\bf replay history} directives.
\item As no TAB* files are yet available for the CG system
at this stage, one can either {\bf (i)} create the initial TAB* files
padded with zeros, or {\bf (ii)} use a FIELD file with fictitious records
for all the interactions to be tabulated, with the interaction
keywords and parameters chosen arbitrarily in accord with
Table~\ref{bond-table}, Table~\ref{angle-table},
Table~\ref{dihedral-table} and Table~\ref{inversion-table}.
{\bf Note} that DL\_CGMAP (as well as VOTCA) creates FIELD\_CG files
that already contain interaction descriptors with the {\bf tab}
keyword(s) in place, so if the route {\em (ii)} is chosen,
the user needs to replace those records with the fictitious ones
(it is advisory to store the initial FIELD\_CG for the future
use, when the actual TAB* files are ready).
\item Run \D with the {\bf replay history}, {\bf rdf} and/or
{\bf analysis} options invoked in the CONTROL file, which
will result in creation of the targeted inter- and intra-molecular
PDF data files (RDFDAT, BNDDAT, ANGDAT, DIHDAT, INVDAT) and the
respective PMF files as described above.  {\bf Note} that only
and only when the {\bf rdf} and {\bf analysis} options are both
active then the VDWPMF and VDWTAB files (derived from RDF:s) will
be produced, along with RDFDAT.  They are structured in the same
manner and format as their intramolecular counterparts.  The user
can then convert the VDWTAB file into a correctly formatted TABLE
file by using the utility called {\tt pmf2tab.f} (subject to
compilation; found in \D directory {\tt utility}) as follows.

%{\tt [user@host]\$ pmf2tab.exe < VDWTAB}

\begin{lstlisting}
[user@host]$ pmf2tab.exe < VDWTAB
\end{lstlisting}

\item Check the data for accuracy and amend the tabulated
force-fields. Redo the analysis on coarser/finer grid(s),
if necessary.

\item When satisfied with the created TAB* files, run
a \D simulation for the prepared CG (or simply user-defined)
model system.
\end{itemize}
