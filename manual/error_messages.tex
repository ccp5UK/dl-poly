\label{error-messages}
\index{error messages}
\subsection*{Introduction}

In this appendix we document the error messages\index{error
messages} encoded in \D and the recommended user action.  The
correct response is described as the {\bf standard user response}
in the appropriate sections below, to which the user should refer
before acting on the error encountered.

The reader should also be aware that some of the error messages
listed below may be either disabled in, or absent from, the public
version of \D.  Note that the wording of some of the messages may
have changed over time, usually to provide more specific
information.  The most recent wording appears below.

\subsubsection*{The Standard User Response}

\D uses FORTRAN90\index{FORTRAN90} dynamic array allocation to set the array sizes
at run time.  This means that a single executable may be compiled
to over all the likely uses of the code.  It is not foolproof
however.  Sometimes an estimate of the required array sizes is
difficult to obtain and the calculated value may be too small.
For this reason \D retains array dimension checks and will
terminate when an array bound error occurs.

When a dimension error occurs, the {\bf standard user response} is
to edit the \D \\ subroutine {\sc set\_bounds}.  Locate where the
variable defining the array dimension is fixed and increase
accordingly.  To do this you should make use of the dimension
information that \D prints in the OUTPUT file prior to
termination.  If no information is supplied, simply doubling the
size of the variable will usually do the trick.  If the variable
concerned is defined in one of the support subroutines {\sc
scan\_config, scan\_field, scan\_control} you will need to insert
a new line in {\sc set\_bounds} to redefine it - after the
relevant subroutine has been called!  Finally the code must be
recompiled, as in this case it will only be necessary to recompile
{\sc set\_bounds} and not the whole code.

\subsection*{The \D Error Messages}

\subsubsection*{\underline{Message 1}: error - word\_2\_real failure}

The semantics in some of the INPUT files is wrong.  \D has tried
to read a number but the has found a word in non-number format.

\noindent \underline{\em Action}:

Look into your INPUT files and correct the semantics where
appropriate and resubmit.  \D will have printed out in the OUTPUT
file what the found non-uniform word is.

\subsubsection*{\underline{Message 2}: error - too many atom types in FIELD (scan\_field)}

This error arises when \D scans the FIELD file and discovers that
there are too many different types of atoms in the system (i.e.
the number of unique atom types exceeds the 1000).

\noindent \underline{\em Action}:

Increase the number of allowed atom types (mmk) in {\sc
scan\_field}, recompile and resubmit.

\subsubsection*{\underline{Message 3}: error - unknown directive found in CONTROL file}

This error most likely arises when a directive is misspelt in the
CONTROL file.

\noindent \underline{\em Action}:

Locate the erroneous directive in the CONTROL file and correct
error and resubmit.

\subsubsection*{\underline{Message 4}: error - unknown directive found in FIELD file}

This error most likely arises when a directive is misspelt or is
encountered in an incorrect location in the FIELD file, which can
happen if too few or too many data records are included.

\noindent \underline{\em Action}:

Locate the erroneous directive in the FIELD file and correct error
and resubmit.

\subsubsection*{\underline{Message 5}: error - unknown energy unit requested}

The \D FIELD file permits a choice of units for input of energy
parameters.  These may be: electron-Volts ({\bf eV}); k-calories per mol
({\bf kcal}/mol); k-Joules per mol ({\bf kJ}/mol); Kelvin per Boltzmann
({\bf K}elvin/Boltzmann); or the \D internal units, 10 Joules per mol ({\bf internal}).
There is no default value.  Failure to specify any of these correctly, or reference
to other energy units, will result in this error message.  See
documentation of the FIELD file.

\noindent \underline{\em Action}:

Correct energy keyword on {\bf units} directive in FIELD file and
resubmit.

\subsubsection*{\underline{Message 6}: error - energy unit not specified}

A {\bf units} directive is mandatory in the FIELD file. This error
indicates that \D has failed to find the required record.

\noindent \underline{\em Action}:

Add {\bf units} directive to FIELD file and resubmit.

\subsubsection*{\underline{Message 7}: error - selected external field incompatible with selected ensemble (NVE only!!!)}

\noindent \underline{\em Action}:

Change the external field directive in FIELD file and or the type of ensemble in CONTROL and resubmit.

\subsubsection*{\underline{Message 8}: error - ewald precision must be a POSITIVE real number}

Ewald precision must be a positive non-zero real number.
For example 10e-5 is accepted as a standard.

\noindent \underline{\em Action}:

Put a correct number at the "ewald precision" directive in the CONTROL file and resubmit.

\subsubsection*{\underline{Message 9}: error - ewald sum parameters must be well defined}

Ewald sum parameters must be well defined.

\noindent \underline{\em Action}:

Referer to the manual and references within for understanding the meaning of the
parameters and how to chose them.  Alternatively, try using the ``ewald precision''
CONTROL directive with a sensible precision value, of say 10$_{-5}$.

\subsubsection*{\underline{Message 10}: error - too many molecular types specified}

This should never happen!  This indicates an erroneous FIELD file
or corrupted \D executable.  Unlike \C, \D does not have a
set limit on the number of kinds of molecules it can handle in any
simulation (this is not the same as the number of molecules).

\noindent \underline{\em Action}:

Examine FIELD for erroneous directives, correct and resubmit.

\subsubsection*{\underline{Message 11}: error - duplicate molecule directive in FIELD file}

The number of different types of molecules in a simulation should
only be specified once.  If \D encounters more than one {\bf
molecules} directive, it will terminate execution.

\noindent \underline{\em Action}:

Locate the extra {\bf molecule} directive in the FIELD file and
remove and resubmit.

\subsubsection*{\underline{Message 12}: error - unknown molecule directive in FIELD file}

Once \D encounters the {\bf molecules} directive in the FIELD
file, it assumes the following records will supply data describing
the intra-molecular force field\index{force field}.  It does not
then expect to encounter directives not related to these data.
This error message results if it encounters a unrelated directive.
The most probable cause is incomplete specification of the data
(e.g.  when the {\bf finish} directive has been omitted.)

\noindent \underline{\em Action}:

Check the molecular data entries in the FIELD file, correct and
resubmit.

\subsubsection*{\underline{Message 13}: error - molecule species not specified}

This error arises when \D encounters
non-bonded\index{potential!non-bonded} force data in the FIELD file,
{\em before} the molecular species have been specified.
Under these circumstances it cannot assign the data
correctly, and therefore terminates.

\noindent \underline{\em Action}:

Make sure the molecular data appears before the
non-bonded\index{potential!non-bonded} forces data in the FIELD
file and resubmit.

\subsubsection*{\underline{Message 14}: error - too many unique atom types specified}

This should never happen!  This error most likely arises when the
FIELD file or/and \D executable are corrupted.

\noindent \underline{\em Action}:

Recompile the program and/or recreate the FIELD file afresh.
If no combination of these works, send the problem to us.

\subsubsection*{\underline{Message 15}: error - duplicate vdw potential specified}

In processing the FIELD file,  \D keeps a record of the specified
short range pair potentials as they are read in.  If it detects
that a given pair potential has been specified before, no attempt
at a resolution of the ambiguity is made and this error message
results.  See specification of FIELD file.

\noindent \underline{\em Action}:

Locate the duplication in the FIELD file, rectify and resubmit.

\subsubsection*{\underline{Message 16}: error - strange exit from FIELD file processing}

This should never happen!  It simply means that \D has ceased
processing the FIELD data, but has not reached the end of the file
or encountered a {\bf close} directive. Probable cause: corruption
of the \D executable or of the FIELD file.  We would be interested
to hear of other reasons!

\noindent \underline{\em Action}:

See action notes on message 14 above.

\subsubsection*{\underline{Message 17}: error - strange exit from CONTROL file processing}

This should never happen!  It simply means that \D has ceased
processing the CONTROL data, but has not reached the end of the file
or encountered a {\bf close} directive. Probable cause: corruption
of the \D executable or of the FIELD file.  We would be interested
to hear of other reasons!

\noindent \underline{\em Action}:

Recompile the program and/or recreate the CONTROL file afresh.
If no combination of these works, send the problem to us.

\subsubsection*{\underline{Message 18}: error - duplicate three-body potential specified}

\D has encountered a repeat specification of a three-body
potential\index{potential!three-body} in the FIELD file.

\noindent \underline{\em Action}:

Locate the duplicate entry, remove and resubmit job.

\subsubsection*{\underline{Message 19}: error - duplicate four-body potential specified}

A 4-body potential has been duplicated in the FIELD file.

\noindent \underline{\em Action}:

Locate the duplicated four-body
potential\index{potential!four-body}, remove and resubmit job.

\subsubsection*{\underline{Message 20}: error - too many molecule sites specified}

This should never happen!  This error most likely arises when the
FIELD file or/and \D executable are corrupted.

\noindent \underline{\em Action}:

See action notes on message 14 above.

\subsubsection*{\underline{Message 21}: error - molecule contains more atoms/sites than declared}

The molecule contains more atom/site entries that it declares in the beginning.

\noindent \underline{\em Action}:

Recreate or correct the erroneous entries in the FIELD file and try again.

\subsubsection*{\underline{Message 22}: error - unsuitable radial increment in TABLE\textbar\textbar TABBND\textbar\textbar TABANG\textbar\textbar TABDIH\textbar\textbar TABINV file}

This arises when the tabulated van der Waals potentials\index{potential!tabulated}
presented in the TABLE file have an increment that is greater
than that used to define the other potentials in the simulation.
Ideally, the increment should be $r_{\rm cut}/({\tt mxgrid}-4)$,
where $r_{\rm cut}$ is the largest potential cutoff of all supplied
,for the short range potentials and the domain decomposition link
cell size, and {\tt mxgrid} is the parameter defining the length
of the interpolation arrays.  An increment less than this is
permissible however.  The same argument holds for the tabulated
intra-molecular interactions that are possibly supplied via the
TABBND, TABANG, TABDIH and TABINV files.  All should have grids
sized less than the generic {\tt mxgrid$-4$}.

\noindent \underline{\em Action}:

The tables must be recalculated with an appropriate increment.

\subsubsection*{\underline{Message 23}: error - incompatible FIELD and TABLE file potentials}

This error arises when the specification of the short range
potentials is different in the FIELD and TABLE files.  This
usually means that the order of specification of the potentials is
different.  When \D finds a change in the order of specification,
it assumes that the user has forgotten to enter one.

\noindent \underline{\em Action}:

Check the FIELD and TABLE files.  Make sure that you correctly
specify the pair potentials in the FIELD file, indicating which
ones are to be presented in the TABLE file.  Then check the TABLE
file to make sure all the tabulated
potentials\index{potential!tabulated} are present in the order the
FIELD file indicates.

\subsubsection*{\underline{Message 24}: error - end of file encountered in TABLE\textbar\textbar TABBND\textbar\textbar TABANG\textbar\textbar TABDIH\textbar\textbar TABINV file}

This means the TABLE\textbar\textbar TABBND\textbar\textbar TABANG\textbar\textbar TABDIH\textbar\textbar TABINV file
is incomplete in some way: either by having too few potentials
included, or the number of data points is incorrect.

\noindent \underline{\em Action}:

Examine the TABLE file contents and regenerate it if it appears to
be incomplete.  If it look intact, check that the number of data
points specified is what \D is expecting.

\subsubsection*{\underline{Message 25}: error - wrong atom type found in CONFIG file}

On reading the input file CONFIG, \D performs a check to ensure
that the atoms specified in the configuration provided are
compatible with the corresponding FIELD file.  This message
results if they are not {\em or the parallel reading wrongly
assumed that CONFIG complies with the DL\_POLY\_3/4 style}.

\noindent \underline{\em Action}:

The possibility exists that one or both of the CONFIG or FIELD
files has incorrectly specified the atoms in the system.  The user
must locate the ambiguity, using the data printed in the OUTPUT
file as a guide, and make the appropriate alteration.  If the
reason is in the parallel reading then produce a new CONFIG
using a serial reading and continue working with it.

\subsubsection*{\underline{Message 26}: error - neutral group option now redundant}

\D does not have the neutral group option.

\noindent \underline{\em Action}:

Use the Ewald sum option.  (It's better anyway.)

\subsubsection*{\underline{Message 27}: error - unit's member indexed outside molecule's site range}

An intra-molecular or intra-molecular alike interaction (topological)
unit has member/site which is given a number outside the scope of
the molecule it is part of.

\noindent \underline{\em Action}:

Find the erroneous entry in FIELD, correct it and try running \D again.

\subsubsection*{\underline{Message 28}: error - wrongly indexed atom entries found in CONFIG file}

\D has detected that the atom indices in the CONFIG file do not
form a contnual and/or non-repeating group of indices.

\noindent \underline{\em Action}:

Make sure the CONFIG file is complies with the \D standards.
You may use the {\bf no index} option in the CONTROL file to
override the crystalographic sites' reading from the CONFIG file
from reading by index to reading by order of the atom entries
with consecutive incremental indexing.  Using this option assumes
that the FIELD topology description matches the crystalographic
sites (atoms entries) in the CONFIG file by order (consecutively).

\subsubsection*{\underline{Message 30}: error - too many chemical bonds specified}

This should never happen!  This error most likely arises when the
FIELD file or/and \D executable are corrupted.

{\em Action}:

See action notes on message 14 above.

\subsubsection*{\underline{Message 31}: error - too many chemical bonds per domain}

\D limits the number of chemical bond\index{potential!bond} units
in the system to be simulated (actually, the number to be processed
by each node) and checks for the violation of this.  Termination will
result if the condition is violated.

\noindent \underline{\em Action}:

Use {\bf densvar} option in CONTROL to increase {\tt mxbond}
(alternatively, increase it by hand in {\sc set\_bounds} and
recompile) and resubmit.

\subsubsection*{\underline{Message 32}: error - coincidence of particles in core-shell unit}

\D has found a fault in the definition of a core-shell unit in the
FIELD file.  The same particle has been assigned to the core and
shell sites.

\noindent \underline{\em Action}:

Correct the erroneous entry in FIELD and resubmit.

\subsubsection*{\underline{Message 33}: error - coincidence of particles in constraint bond unit}

\D has found a fault in the definition of a constraint bond unit
in the FIELD file.  The same particle has been assigned to the
both sites.

\noindent \underline{\em Action}:

Correct the erroneous entry in FIELD and resubmit.

\subsubsection*{\underline{Message 34}: error - length of constraint bond unit~$>=$~real space cutoff (rcut)}

\D has found a constraint bond unit length (FIELD) larger than the
real space cutoff ({\tt rcut}) (CONTROL).

\noindent \underline{\em Action}:

Increase cutoff in CONTROL or decrease the constraint bondlength
in FIELD and resubmit.  For small system consider using \C.

\subsubsection*{\underline{Message 35}: error - coincidence of particles in chemical bond unit}

\D has found a faulty chemical bond in FIELD (defined between the
same particle).

\noindent \underline{\em Action}:

Correct the erroneous entry in FIELD and resubmit.

\subsubsection*{\underline{Message 36}: error - only one *bonds* directive per molecule is allowed}

\D has found more than one bonds entry per molecule in FIELD.

\noindent \underline{\em Action}:

Correct the erroneous part in FIELD and resubmit.

\subsubsection*{\underline{Message 38}: error - outgoing transfer buffer size exceeded in metal\_ld\_export}

This should not usually happen!

\noindent \underline{\em Action}:

Consider using {\tt densvar} option in CONTROL for extremely
non-equilibrium simulations.  Alternatively, increase {\tt mxbfxp}
parameter in {\sc set\_bounds} recompile and resubmit.  Send the
problem to us if this is persistent.

\subsubsection*{\underline{Message 39}: error - incoming data transfer size exceeds limit in metal\_ld\_export}

See notes on message 38 above.

\noindent \underline{\em Action}:

See action notes on message 38 above.

\subsubsection*{\underline{Message 40}: error - too many bond constraints specified}

This should never happen!

\noindent \underline{\em Action}:

See action notes on message 14 above.

\subsubsection*{\underline{Message 41}: error - too many bond constraints per domain}

\D limits the number of bond constraint\index{constraints!bond} units
in the system to be simulated (actually, the number to be processed
by each node) and checks for the violation of this.  Termination will
result if the condition is violated.

\noindent \underline{\em Action}:

Use {\bf densvar} option in CONTROL to increase {\tt mxcons}
(alternatively, increase it by hand in {\sc set\_bounds} and
recompile) and resubmit.

\subsubsection*{\underline{Message 42}: error - undefined direction passed to deport\_atomic\_data}

This should never happen!

\noindent \underline{\em Action}:

Send the problem to us.

\subsubsection*{\underline{Message 43}: error - outgoing transfer buffer size exceeded in deport\_atomic\_data}

This may happen in extremely non-equilibrium simulations or
usually when the potentials in use do not hold the system stable.

\noindent \underline{\em Action}:

Consider using {\tt densvar} option in CONTROL for extremely
non-equilibrium simulations.  Alternatively, increase {\tt mxbfdp}
parameter in {\sc set\_bounds} recompile and resubmit.

\subsubsection*{\underline{Message 44}: error - incoming data transfer size exceeds limit in deport\_atomic\_data}

\noindent \underline{\em Action}:

See action notes on message 43 above.

\subsubsection*{\underline{Message 45}: error - too many atoms in CONFIG file or per domain}

This can happen in circumstances when indeed the CONFIG file has more
atoms listed than defined in FIELD, or when one of the domains (managed
by an MPI process) has higher particle density than the system average
and contains more particles than allowed by the default based on the
system.

\noindent \underline{\em Action}:

Check if CONFIG and FIELD numbers of particles match.  Try executing on
various number of processors.  Try using the {\bf densvar} option in
CONTROL to increase {\tt mxatms} (alternatively, increase it by hand in
{\sc set\_bounds} and recompile) and resubmit.  Send the problem to us
if this is persistent.

\subsubsection*{\underline{Message 46}: error - undefined direction passed to export\_atomic\_data}

This should never happen!

\noindent \underline{\em Action}:

Send the problem to us.

\subsubsection*{\underline{Message 47}: error - undefined direction passed to metal\_ld\_export}

This should never happen!

\noindent \underline{\em Action}:

Send the problem to us.

\subsubsection*{\underline{Message 48}: error - transfer buffer too small in *\_table\_read}

\noindent \underline{\em Action}:

Standard user response.  Increase {\tt mxgrid} parameter
in {\sc set\_bounds} recompile and resubmit.

\subsubsection*{\underline{Message 49}: error - frozen shell (core-shell) unit specified}

The \D option to freeze the location of an atom (i.e. hold it
permanently in one position) is not permitted for the shells in
core-shell units.

\noindent \underline{\em Action}:

Remove the frozen atom option from the FIELD file. Consider using
a non-polarisable atom instead.

\subsubsection*{\underline{Message 50}: error - too many bond angles specified}

This should never happen!  This error most likely arises when the
FIELD file or/and \D executable are corrupted.

\noindent \underline{\em Action}:

See action notes on message 14 above.

\subsubsection*{\underline{Message 51}: error - too many bond angles per domain}

\D limits the number of valence angle\index{potential!valence angle} units
in the system to be simulated (actually, the number to be processed
by each node) and checks for the violation of this.  Termination will
result if the condition is violated.

\noindent \underline{\em Action}:

Use {\bf densvar} option in CONTROL to increase {\tt mxangl}
(alternatively, increase it by hand in {\sc set\_bounds} and
recompile) and resubmit.

\subsubsection*{\underline{Message 52}: error - end of FIELD file encountered}

This message results when \D reaches the end of the FIELD file,
without having read all the data it expects.  Probable causes:
missing data or incorrect specification of integers on the various
directives.

\noindent \underline{\em Action}:

Check FIELD file for missing or incorrect data, correct and
resubmit.

\subsubsection*{\underline{Message 53}: error - end of CONTROL file encountered}

This message results when \D reaches the end of the CONTROL file,
without having read all the data it expects.  Probable cause:
missing {\bf finish} directive.

\noindent \underline{\em Action}:

Check CONTROL file, correct and resubmit.

\subsubsection*{\underline{Message 54}: error - outgoing transfer buffer size exceeded in export\_atomic\_data}

See notes on message 38 above.

\noindent \underline{\em Action}:

See naction otes on message 38 above.

\subsubsection*{\underline{Message 55}: error - end of CONFIG file encountered}

This error arises when \D attempts to read more data from the
CONFIG file than is actually present.  The probable cause is an
incorrect or absent CONFIG file, but it may be due to the FIELD
file being incompatible in some way with the CONFIG file.

\noindent \underline{\em Action}:

Check contents of CONFIG file.  If you are convinced it is
correct, check the FIELD file for inconsistencies.

\subsubsection*{\underline{Message 56}: error - incoming data transfer size exceeds limit in export\_atomic\_data}

See notes on message 38 above.

\noindent \underline{\em Action}:

See action notes on message 38 above.

\subsubsection*{\underline{Message 57}: error - too many core-shell units specified}

This should never happen!

\noindent \underline{\em Action}:

See action notes on message 14 above.

\subsubsection*{\underline{Message 58}: error - number of atoms in system not conserved}

Either and an atom has been lost in transfer between nodes/domains or your FIELD is ill defined with respect to what is supplied in CONFIG/HISTORY.

\noindent \underline{\em Action}:

If this error is issued at start before timestep zero in a simulation
then it is either your FIELD file is ill defined or that your CONFIG file
(or the first frame of your HISTRORY being replayed).  Check out for
mistyped number or identities of molecules, atoms, etc. in FIELD and for
mangled/blank lines in CONFIG/HISTORY, or a blank line(s) at the end of
CONFIG or missing FOF (End Of File) character in CONFIG.  If this error
is issued after timestep zero in a simulation that is not replaying
HISTORY then it is big trouble and you should report that to the authors.
If it is during replaying HISTORY then your HISTORY file has corrupted
frames and you must correct it before trying again.

\subsubsection*{\underline{Message 59}: error - too many core-shell units per domain}

\D limits the number of core-shell\index{polarisation!shell model} units
in the system to be simulated (actually, the number to be processed
by each node) and checks for the violation of this.  Termination will
result if the condition is violated.

\noindent \underline{\em Action}:

Use {\bf densvar} option in CONTROL to increase {\tt mxshl}
(alternatively, increase it by hand in {\sc set\_bounds} and
recompile) and resubmit.

\subsubsection*{\underline{Message 60}: error - too many dihedral angles specified}

This should never happen!

\noindent \underline{\em Action}:

See action notes on message 14 above.

\subsubsection*{\underline{Message 61}: error - too many dihedral angles per domain}

\D limits the number of dihedral angle\index{potential!dihedral} units
in the system to be simulated (actually, the number to be processed
by each node) and checks for the violation of this.  Termination will
result if the condition is violated.

\noindent \underline{\em Action}:

Use {\bf densvar} option in CONTROL to increase {\tt mxdihd}
(alternatively, increase it by hand in {\sc set\_bounds} and
recompile) and resubmit.

\subsubsection*{\underline{Message 62}: error - too many tethered atoms specified}

This should never happen!

\noindent \underline{\em Action}:

See action notes on message 14 above.

\subsubsection*{\underline{Message 63}: error - too many tethered atoms per domain}

\D limits the number of tethered atoms\index{potential!tether}
in the system to be simulated (actually, the number to be processed
by each node) and checks for the violation of this.  Termination will
result if the condition is violated.

\noindent \underline{\em Action}:

Use {\bf densvar} option in CONTROL to increase {\tt mxteth}
(alternatively, increase it by hand in {\sc set\_bounds} and
recompile) and resubmit.

\subsubsection*{\underline{Message 64}: error - incomplete core-shell unit found in build\_book\_intra}

This should never happen!

\noindent \underline{\em Action}:

Report problem to authors.

\subsubsection*{\underline{Message 65}: error - too many excluded pairs specified}

This should never happen!  This error arises when \D is
identifying the atom pairs that cannot have a pair potential
between them,  by virtue of being chemically bonded for example
(see subroutine {\sc build\_excl\_intra}).  Some of the working
arrays used in this operation may be exceeded, resulting in
termination of the program.

\noindent \underline{\em Action}:

Contact authors.

\subsubsection*{\underline{Message 66}: error - coincidence of particles in bond angle unit}

\D has found a fault in the definition of a bond angle in the
FIELD file.

\noindent \underline{\em Action}:

Correct the erroneous entry in FIELD and resubmit.

\subsubsection*{\underline{Message 67}: error - coincidence of particles in dihedral unit}

\D has found a fault in the definition of a dihedral unit in the
FIELD file.

\noindent \underline{\em Action}:

Correct the erroneous entry in FIELD and resubmit.

\subsubsection*{\underline{Message 68}: error - coincidence of particles in inversion unit}

\D has found a fault in the definition of a inversion unit in the
FIELD file.

\noindent \underline{\em Action}:

Correct the erroneous entry in FIELD and resubmit.

\subsubsection*{\underline{Message 69}: error - too many link cells required in three\_body\_forces}

The number of link cells required for the build up of the Verlet neighbour list
(as in link\_cell\_pairs) or the calculation of three- \& four-body as well tersoff
forces (as in three\_body\_forces, four\_body\_forces, tersoff\_body\_forces)
in the given model exceeds the number allowed for by the \D arrays.  Probable cause:
your system has expanded unacceptably much to \D.  This may not be
physically sensible!

\noindent \underline{\em Action}:

Consider using {\tt densvar} option in CONTROL for extremely non-equilibrium simulations.

\subsubsection*{\underline{Message 70}: error - constraint\_quench failure}

When a simulation with bond constraints\index{constraints!bond} is
started, \D attempts to extract the kinetic energy of the
constrained atom-atom bonds arising from the assignment of initial
random velocities.  If this procedure fails, the program will
terminate.  The likely cause is a badly generated initial
configuration.

\noindent \underline{\em Action}:

Some help may be gained from increasing the cycle limit, by using
the directive {\bf mxshak} in the CONTROL file. You may also
consider reducing the tolerance of the SHAKE iteration using the
directive {\bf shake} in the CONTROL file. However it is probably
better to take a good look at the starting conditions!

\subsubsection*{\underline{Message 71}: error - too many metal potentials specified}

This should never happen!

\noindent \underline{\em Action}:

Report to authors.

\subsubsection*{\underline{Message 72}: error - too many tersoff potentials specified}

This should never happen!

\noindent \underline{\em Action}:

Report to authors.

\subsubsection*{\underline{Message 73}: error - too many inversion potentials specified}

This should never happen!

\noindent \underline{\em Action}:

Report to authors.

\subsubsection*{\underline{Message 74}: error - unidentified atom in tersoff potential list}

This shows that \D has encountered and erroneous entry for Tersoff
potentials in FIELD.

\noindent \underline{\em Action}:

Correct FIELD and resubmit.

\subsubsection*{\underline{Message 76}: error - duplicate tersoff potential specified}

This shows that \D has encountered and erroneous entry for Tersoff
potentials in FIELD.

\noindent \underline{\em Action}:

Correct FIELD and resubmit.

\subsubsection*{\underline{Message 77}: error - too many inversion angles per domain}

\D limits the number of inversion\index{potential!inversion} units
in the system to be simulated (actually, the number to be processed
by each node) and checks for the violation of this.  Termination will
result if the condition is violated.

\noindent \underline{\em Action}:

Use {\bf densvar} option in CONTROL to increase {\tt mxinv}
(alternatively, increase it by hand in {\sc set\_bounds} and
recompile) and resubmit.

\subsubsection*{\underline{Message 79}: error - tersoff potential cutoff undefined}

This shows that \D has encountered and erroneous entry for Tersoff
potentials in FIELD.

\noindent \underline{\em Action}:

Correct FIELD and resubmit.

\subsubsection*{\underline{Message 80}: error - too many pair potentials specified}

This should never happen!

\noindent \underline{\em Action}:

Report to authors.

\subsubsection*{\underline{Message 81}: error - unidentified atom in pair potential list}

This shows that \D has encountered and erroneous entry for vdw or
metal potentials in FIELD or cited TABle file.

\noindent \underline{\em Action}:

Correct FIELD and/or cited TABle file.

\subsubsection*{\underline{Message 82}: error - calculated pair potential index too large}

This should never happen!  In checking the vdw and metal
potentials specified in the FIELD file \D calculates a unique
integer indices that henceforth identify every specific potential
within the program.  If this index becomes too large, termination
of the program results.

\noindent \underline{\em Action}:

Report to authors.

\subsubsection*{\underline{Message 83}: error - too many three-body/angles potentials specified}

This should never happen!

\noindent \underline{\em Action}:

Report to authors.

\subsubsection*{\underline{Message 84}: error - unidentified atom in three-body/angles potential list}

This shows that \D has encountered and erroneous entry at three-body or angles
definitions in FIELD.

\noindent \underline{\em Action}:

Correct FIELD and resubmit.

\subsubsection*{\underline{Message 85}: error - required velocities not in CONFIG file}

If the user attempts to start up a \D simulation with any type of
{\bf restart} directive (see description of CONTROL file,) the
program will expect the CONFIG file to contain atomic velocities
as well as positions.  Termination results if these are not
present.

\noindent \underline{\em Action}:

Either replace the CONFIG file with one containing the velocities,
or if not available, remove the {\bf restart ...} directive
altogether and let \D create the velocities for itself.

\subsubsection*{\underline{Message 86}: error - calculated three-body potential index too large}

This should never happen!  \D has a permitted maximum for the
calculated index for any three-body\index{potential!three-body}
potential in the system (i.e. as defined in the FIELD file).  If
there are $m$ distinct types of atom in the system, the index can
possibly range from $1$ to $(m^{2}*(m-1))/2$.  If the internally
calculated index exceeds this number, this error reports results.

\noindent \underline{\em Action}:

Report to authors.

\subsubsection*{\underline{Message 88}: error - legend array exceeded in build\_book\_intra}

The second dimension of a legend array has been exceeded.

\noindent \underline{\em Action}:

If you have an intra-molecular (like) interaction present in abundance
in your model that you suspect is driving this out of bound error
increase its legend bound value, {\tt mxf}interaction, at the end of
{\sc scan\_field}, recompile and resubmit.  If the error persists
contact authors.

\subsubsection*{\underline{Message 89}: error - too many four-body/dihedrals/inversions potentials specified}

This should never happen!

\noindent \underline{\em Action}:

Report to authors.

\subsubsection*{\underline{Message 90}: error - specified tersoff potentials have different types'}

This is not allowed!  Only one general type of tersoff potential is allowed in FIELD as there are no mixing rules between different tersoff potentials!

\noindent \underline{\em Action}:

Correct your model representation in FIELD and try again.

\subsubsection*{\underline{Message 91}: error - unidentified atom in four-body/dihedrals/inversions potential list}

The specification of a four-body\index{potential!four-body} or
dihedrals or inversions potential in the FIELD file has referenced
an atom type that is unknown.

\noindent \underline{\em Action}:

Locate the errant atom type in the four-body/dihedrals/inversions
potential definition in the FIELD file and correct.  Make sure this
atom type is specified by an {\tt atoms} directive earlier in the file.

\subsubsection*{\underline{Message 92}: error - specified metal potentials have different types}

The specified metal\index{potential!metal} interactions in the
FIELD file are referencing more than one generic type of metal
potentials.  Only one such type is allowed in the system.

\noindent \underline{\em Action}:

Locate the errant metal type in the metal potential definition
in the FIELD file and correct.  Make sure only one metal type is
specified for all relevan atom interactions in the file.

\subsubsection*{\underline{Message 93}: error - PMFs mixing with rigid bodies not allowed}

\noindent \underline{\em Action}:

Correct FIELD and resubmit.

\subsubsection*{\underline{Message 95}: error - error - rcut or (rcut+rpad)~$>$~minimum of all half-cell widths}

In order for the minimum image convention to work correctly within \D,
it is necessary to ensure that the major cutoff, plus its possible padding
distance, applied to the pair interactions does not exceed half the
perpendicular width of the simulation cell.  (The perpendicular width is the
shortest distance between opposing cell faces.)  Termination results if this
is detected.  In NVE and NVT simulations this can only happen at the start of
a simulation, but in NPT and N$\mat{\sigma}$T, it may occur at any time.

\noindent \underline{\em Action}:

Supply a cutoff that is less than half the cell width.  If running
constant pressure calculations, use a cutoff that will accommodate
the fluctuations in the simulation cell.  Study the fluctuations
in the OUTPUT file to help you with this.

\subsubsection*{\underline{Message 96}: error - incorrect atom totals assignments in metal\_ld\_set\_halo}

This should never happen!

\noindent \underline{\em Action}:

Big trouble.  Report to authors.

\subsubsection*{\underline{Message 97}: error - constraints mixing with rigid bodies not allowed}

\noindent \underline{\em Action}:

Correct FIELD and resubmit.

\subsubsection*{\underline{Message 99}: error - cannot have shells as part of a constraint, rigid body or tether}

\noindent \underline{\em Action}:

Correct FIELD and resubmit.

\subsubsection*{\underline{Message 100}: error - core-shell unit separation $>$ rcut (the system cutoff)}

This could only happen if FIELD and CONFIG do not match each other
or CONFIG is damaged.

\noindent \underline{\em Action}:

Regenerate CONFIG (and FIELD) and resubmit.

\subsubsection*{\underline{Message 101}: error - calculated four-body potential index too large}

This should never happen!  \D has a permitted maximum for the
calculated index for any four-body potential in the system (i.e.
as defined in the FIELD file).  If there are $m$ distinct types of
atom in the system, the index can possibly range from $1$ to
$(m^{2}*(m+1)*(m+2))/6$.  If the internally calculated index
exceeds this number, this error report results.

\noindent \underline{\em Action}:

Report to authors.

\subsubsection*{\underline{Message 102}: error - rcut~$<$~2*rcter (maximum cutoff for tersoff potentials)}

The nature of the Tersoff interaction requires they have at least
twice shorter cutoff than the standard pair interctions (or the
major system cutoff).

\noindent \underline{\em Action}:

Decrease Tersoff cutoffs in FIELD or increase cutoff in CONTROL
and resubmit.

\subsubsection*{\underline{Message 103}: error - parameter mxlshp exceeded in pass\_shared\_units}

Various algorithms (constraint and core-shell ones) require that
information about `shared' atoms be passed between nodes.  If
there are too many such atoms, the arrays holding the information
will be exceeded and \D will terminate execution.

\noindent \underline{\em Action}:

Use {\bf densvar} option in CONTROL to increase {\tt mxlshp}
(alternatively, increase it by hand in {\sc set\_bounds} and
recompile) and resubmit.

\subsubsection*{\underline{Message 104}: error - arrays listme and lstout exceeded in pass\_shared\_units}

This should not happen!  Dimensions of indicated arrays have been
exceeded.

\noindent \underline{\em Action}:

Consider using {\tt densvar} option in CONTROL for extremely
non-equilibrium simulations.

\subsubsection*{\underline{Message 105}: error - shake algorithm (constraints\_shake) failed to converge}

The SHAKE\index{algorithm!SHAKE} algorithm for bond
constraints\index{constraints!bond} is iterative.  If the maximum
number of permitted iterations is exceeded, the program
terminates.  Possible causes include: a bad starting
configuration; too large a time step used; incorrect force
field\index{force field} specification; too high a temperature;
inconsistent constraints (over-constraint) etc..

\noindent \underline{\em Action}:

You may try to increase the limit of iteration cycles in the
constraint subroutines by using the directive {\bf mxshak} and/or
decrease the constraint precision by using the directive {\bf shake}
in CONTROL.  But the trouble may be much more likely to be cured by
careful consideration of the physical system being simulated.  For
example, is the system stressed in some way? Too far from
equilibrium?

\subsubsection*{\underline{Message 106}: error - neighbour list array too small in link\_cell\_pairs}

Construction of the Verlet neighbour list\index{Verlet neighbour
list} in subroutine {\sc link\_cell\_pairs} non-bonded (pair) force
has exceeded the neighbour list array dimensions.

\noindent \underline{\em Action}:

Consider using {\tt densvar} option in CONTROL for extremely
non-equilibrium simulations or increase by hand {\tt mxlist} in
{\sc set\_bounds}.

\subsubsection*{\underline{Message 107}: error - too many pairs for rdf look up specified}

This should never happen!  A possible reason is corruption in
FIELD or/and \D executable.

\noindent \underline{\em Action}:

See action notes on message 14 above.

\subsubsection*{\underline{Message 108}: error - unidentified atom in rdf look up list}

During reading of RDF look up pairs in FIELD \D has found an
unlisted previously atom type.

\noindent \underline{\em Action}:

Correct FIELD by either defining the new atom type or changing it
to an already defined one in the erroneous line.  Resubmit.

\subsubsection*{\underline{Message 109}: error - calculated pair rdf index too large}

This should never happen!  In checking the RDF pairs specified in
the FIELD file \D calculates a unique integer index that
henceforth identify every RDF pair within the program.  If this
index becomes too large, termination of the program results.

\noindent \underline{\em Action}:

Report to authors.

\subsubsection*{\underline{Message 108}: error - duplicate rdf look up pair specified}

During reading of RDF look up pairs in FIELD \D has found a
duplicate entry in the list.

\noindent \underline{\em Action}:

Delete the duplicate line and resubmit.

\subsubsection*{\underline{Message 111}: error - bond constraint unit separation $>$ rcut (the system cutoff)}

This should never happen!  \D has not been able to find an atom in
a processor domain or its bordering neighbours.

\noindent \underline{\em Action}:

Probable cause: link cells too small.  Use larger potential cutoff.
Contact \D authors.

\subsubsection*{\underline{Message 112}: error - only one *constraints* directive per molecule is allowed}

\D has found more than one constraints entry per molecule in FIELD.

\noindent \underline{\em Action}:

Correct the erroneous part in FIELD and resubmit.

\subsubsection*{\underline{Message 113}: error - intra-molecular bookkeeping arrays exceeded in deport\_atomic\_data}

One or more bookkeeping arrays for site-related interactions have
been exceeded.

\noindent \underline{\em Action}:

Consider using {\tt densvar} option in CONTROL for extremely
non-equilibrium simulations.  Alternatively, you will need to
print extra diagnostic data from the {\sc deport\_atomic\_data}
subroutine to find which boded-like contribution has exceeded
its assumed limit and then correct for it in {\sc set\_bounds},
recompile and resubmit.

\subsubsection*{\underline{Message 114}: error - legend array exceeded in deport\_atomic\_data}

The array {\tt legend} has been exceeded.

\noindent \underline{\em Action}:

Try increasing parameter {\tt mxfix} in {\sc set\_bounds},
recompile and resubmit.  Contact \D authors if the problem
persists.

\subsubsection*{\underline{Message 115}: error - transfer buffer exceeded in update\_shared\_units}

The transfer buffer has been exceeded.

\noindent \underline{\em Action}:

Consider increasing parameter {\tt mxbfsh} in {\sc set\_bounds},
recompile and resubmit.  Contact \D authors if the problem
persists.

\subsubsection*{\underline{Message 116}: error - incorrect atom transfer in update\_shared\_units}

An atom has become misplaced during transfer between nodes.

\noindent \underline{\em Action}:

This happens when the simulation is very numerically unstable.
Consider carefully the physical grounds of your simulation, i.e.
are you using the adiabatic shell model for accounting polarisation
with too big a timestep or too large control distances for the
variable timestep, is the ensemble type NPT or N$\mat{\sigma}$T and
the system target temperature too close to the melting temperature?

\subsubsection*{\underline{Message 118}: error - construction error in pass\_shared\_units}

This should not happen.

\noindent \underline{\em Action}:

Report to authors.

\subsubsection*{\underline{Message 120}: error - invalid determinant in matrix inversion}

\D occasionally needs to calculate matrix inverses (usually the
inverse of the matrix of cell vectors, which is of size 3 $\times$
3).  For safety's sake a check on the determinant is made, to
prevent inadvertent use of a singular matrix.

\noindent \underline{\em Action}:

Locate the incorrect matrix and fix it - e.g. are cell vectors
correct?

\subsubsection*{\underline{Message 122}: error - FIELD file not found}

\D failed to find a FIELD file in your directory.

\noindent \underline{\em Action}:

Supply a valid FIELD file before you start a simulation

\subsubsection*{\underline{Message 124}: error - CONFIG file not found}

\D failed to find a CONFIG file in your directory.

\noindent \underline{\em Action}:

Supply a valid CONFIG file before you start a simulation

\subsubsection*{\underline{Message 126}: error - CONTROL file not found}

\D failed to find a CONTROL file in your directory.

\noindent \underline{\em Action}:

Supply a valid CONTROL file before you start a simulation

\subsubsection*{\underline{Message 128}: error - chemical bond unit separation $>$ rcut (the system cutoff)}

This could only happen if FIELD and CONFIG do not match each other
or if the instantaneous configuration is ill defined because of
generation of large forces on bonded particles.  This may be due
to having a badly defined force-field and/or starting form a configuration
which is too much away from equilibrium.

\noindent \underline{\em Action}:

Regenerate CONFIG (and FIELD) and resubmit.  Try topology verification
by using {\tt nfold 1 1 1} in CONTROL.  Try using options as {\tt scale},
{\tt cap}, {\tt zero} and {\tt optimise}.  Try using smaller SHAKE tolerance
if constraints are present in the system.  You may as well try using the
{\tt variable timestep} option.

\subsubsection*{\underline{Message 130}: error - bond angle unit diameter $>$ rcut (the system cutoff)}

See action notes on message 128 above.

\noindent \underline{\em Action}:

See action notes on message 128 above.

\subsubsection*{\underline{Message 132}: error - dihedral angle unit diameter $>$ rcut (the system cutoff)}

See notes on message 128 above.

\noindent \underline{\em Action}:

See action notes on message 128 above.

\subsubsection*{\underline{Message 134}: error - inversion angle unit diameter $>$ rcut (the system cutoff)}

See notes on message 128 above.

\noindent \underline{\em Action}:

See action notes on message 128 above.

\subsubsection*{\underline{Message 138}: error - incorrect atom totals assignments in refresh\_halo\_positions}

This should never happen although, sometimes, it could due
to ill defined force field and/or and/or starting form a
configuration which is too much away from equilibrium.

\noindent \underline{\em Action}:

Try using the {\tt variable timestep} option and/or running in serial
to determine if particles gain too much speed and leave domains.

\subsubsection*{\underline{Message 141}: error - duplicate metal potential specified}

During reading of metal potentials (pairs of atom types) in FIELD
\D has found a duplicate pair of atoms in the list.

\noindent \underline{\em Action}:

Delete one of the duplicate entries and resubmit.

\subsubsection*{\underline{Message 145}: error - no two-body like interactions specified}

This error arises when there are no two-body like interactions
specified in FIELD and CONTROL.  I.e. none of the following
interactions exists or if does, it has been switched off;
any coulombic, vdw, metal, tersoff.  In \D expects that particles will
be kept apparat, stay separated and never go through each other
due to one of the fore-specified interactions.

\noindent \underline{\em Action}:

Users must alone take measures to prevent such outcome.

\subsubsection*{\underline{Message 150}: error - unknown van der waals potential selected}

\D checks when constructing the interpolation tables for the short
ranged potentials that the potential function requested is one
which is of a form known to the program.  If the requested
potential form is unknown, termination of the program results. The
most probable cause of this is the incorrect choice of the
potential keyword in the FIELD file.

\noindent \underline{\em Action}:

Read the \D documentation and find the potential keyword for the
potential desired.

\subsubsection*{\underline{Message 151}: error - unknown EAM keyword in TABEAM}

\D checks when constructing the interpolation tables for the EAM metal
potentials that the potential function requested is one which is
of a form known to the program.  If the requested potential form
is unknown, termination of the program results.  The most probable
cause of this is the incorrect choice of the potential keyword in
the FIELD file.

\noindent \underline{\em Action}:

Read the \D documentation and find the potential keyword for the
potential desired.

\subsubsection*{\underline{Message 152}: error - undefined direction passed to dpd\_v\_export}

This should never happen!

\noindent \underline{\em Action}:

Report to authors.

\subsubsection*{\underline{Message 154}: error - outgoing transfer buffer size exceeded in dpd\_v\_export}

See notes on message 38 above.

\noindent \underline{\em Action}:

See action notes on message 38 above.

\subsubsection*{\underline{Message 156}: error - incoming data transfer size exceeds limit in dpd\_v\_export}

See notes on message 38 above.

\noindent \underline{\em Action}:

See action notes on message 38 above.

\subsubsection*{\underline{Message 158}: error - incorrect atom totals assignments in  dpd\_v\_set\_halo}

This should never happen!

\noindent \underline{\em Action}:

Big trouble.  Report to authors.

\subsubsection*{\underline{Message 160}: error - undefined direction passed to statistics\_connect\_spread}

This should never happen!

\subsubsection*{\underline{Message 163}: error - outgoing transfer buffer size exceeded in statistics\_connect\_spread}

The transfer buffer has been exceeded.

\noindent \underline{\em Action}:

Consider using {\tt densvar} option in CONTROL for extremely
non-equilibrium simulations.  Alternatively, increase {\tt mxbfss}
parameters in {\sc set\_bounds} recompile and resubmit.

\subsubsection*{\underline{Message 164}: error - incoming data transfer size exceeds limit in statistics\_connect\_spread}

See notes on message 163 above.

\noindent \underline{\em Action}:

See action notes on message 163 above.

\subsubsection*{\underline{Message 170}: error - too many variables for statistics array}

This error means the statistics arrays appearing in subroutine
{\sc statistics\_collect} are too small.  This should never
happen!

\noindent \underline{\em Action}:

Contact \D authors.

\subsubsection*{\underline{Message 172}: error - duplicate intra-molecular entries specified in TABBND\textbar\textbar TABANG\textbar\textbar TABDIH\textbar\textbar TABINV}

A duplicate entry has been encountered in the intra-molecular table file.

\noindent \underline{\em Action}:

Contact \D authors.

\subsubsection*{\underline{Message 200}: error - rdf/z-density buffer array too small in system\_revive}

This error indicates that a global summation buffer array in
subroutine {\sc system\_revive} is too small, i.e {\tt
mxbuff}~$<$~{\tt mxgrdf}.  This should never happen!

\noindent \underline{\em Action}:

Contact \D authors.

\subsubsection*{\underline{Message 210}: error - only one *angles* directive per molecule is allowed}

\D has found more than one angles entry per molecule in FIELD.

\noindent \underline{\em Action}:

Correct the erroneous part in FIELD and resubmit.

\subsubsection*{\underline{Message 220}: error - only one *dihedrals* directive per molecule is allowed}

\D has found more than one dihedrals entry per molecule in FIELD.

\noindent \underline{\em Action}:

Correct the erroneous part in FIELD and resubmit.

\subsubsection*{\underline{Message 230}: error - only one *inversions* directive per molecule is allowed}

\D has found more than one inversions entry per molecule in FIELD.

\noindent \underline{\em Action}:

Correct the erroneous part in FIELD and resubmit.

\subsubsection*{\underline{Message 240}: error - only one *tethers* directive per molecule is allowed}

\D has found more than one tethers entry per molecule in FIELD.

\noindent \underline{\em Action}:

Correct the erroneous part in FIELD and resubmit.

\subsubsection*{\underline{Message 300}: error - incorrect boundary condition for link-cell algorithms}

The use of link cells in \D implies the use of appropriate
boundary conditions.  This error results if the user specifies
octahedral or dodecahedral boundary conditions, which are only
available in \C.

\noindent \underline{\em Action}:

Correct your boundary condition or consider using \C.

\subsubsection*{\underline{Message 305}: error - too few link cells per dimension for many-body and tersoff forces subroutines.}

The link cells algorithms for many-body and tersoff forces in \D
cannot work with less than 3 (secondary) link cells per dimension.
This depends on the cell size widths (as supplied in CONFIG) and the
largest system cut-off (as specified in CONTROL although it may be
drawn or overridden by cutoffs specified as part of some potentials'
parameter sets in FIELD).

\noindent \underline{\em Action}:

Decrease many-body and tersoff potentials cutoffs or/and number of
nodes or/and increase system size.

\subsubsection*{\underline{Message 307}: error - link cell algorithm violation}

\D does not like what you are asking it to do.  Probable cause:
the cutoff is too large to use link cells in this case.

\noindent \underline{\em Action}:

Rethink the simulation model; reduce the cutoff or/and number of
nodes or/and increase system size.

\subsubsection*{\underline{Message 308}: error - link cell algorithm in contention with SPME sum precision}

\D does not like what you are asking it to do.  Probable cause:
you ask for SPME precision that is not achievable by the current
settings of the link cell algorithm.

\noindent \underline{\em Action}:

Rethink the simulation model; reduce number of nodes or/and SPME sum
precision or/and increase cutoff.

\subsubsection*{\underline{Message 321}: error - LFV quaternion integrator failed}

This indicates unstable integration but may be due to many reasons.

\noindent \underline{\em Action}:

Rethink the simulation model.  Increase mxquat in CONTROL and resubmit or use VV integration to check system stability.

\subsubsection*{\underline{Message 340}: error - invalid integration option requested}

\D has detected an incompatibility in the simulation instructions,
namely that the requested integration algorithm is not compatible
with the physical model.  It {\em may} be possible to override
this error trap, but it is up to the user to establish if this is
sensible.

\noindent \underline{\em Action}:

This is a non-recoverable error, unless the user chooses to
override the restriction.

\subsubsection*{\underline{Message 350}: error - too few degrees of freedom}

This error can arise if a small system is being simulated and the
number of constraints applied is too large.

\noindent \underline{\em Action}:

Simulate a larger system or reduce the number of constraints.

\subsubsection*{\underline{Message 360}: error - degrees of freedom distribution problem}

This should never happen for a dynamically sensical system.  This
error arises if a model system contains one or more free, zero mass
particles.  Zero mass (mass-less) particles/sites are only allowed for
shells in core-shell units and as part of rigid bodies (mass-less but
charged RB sites).

\noindent \underline{\em Action}:

Inspect your FIELD to find and correct the erroneous entries,
and try again.

\subsubsection*{\underline{Message 380}: error - simulation temperature not specified or $< 1$ K}

\D has failed to find a {\bf temp} directive in the CONTROL file.

\noindent \underline{\em Action}:

Place a {\bf temp} directive in the CONTROL file, with the
required temperature specified.

\subsubsection*{\underline{Message 381}: error - simulation timestep not specified}

\D has failed to find a {\bf timestep} directive in the CONTROL
file.

\noindent \underline{\em Action}:

Place a {\bf timestep} directive in the CONTROL file, with the
required timestep specified.

\subsubsection*{\underline{Message 382}: error - simulation cutoff not specified}

\D has failed to find a {\bf cutoff} directive in the CONTROL
file.

\noindent \underline{\em Action}:

Place a {\bf cutoff} directive in the CONTROL file, with the
required forces cutoff specified.

\subsubsection*{\underline{Message 387}: error - system pressure not specified}

The target system pressure has not been specified in the CONTROL
file.  Applies to NPT simulations only.

\noindent \underline{\em Action}:

Insert a {\bf press} directive in the CONTROL file specifying the
required system pressure.

\subsubsection*{\underline{Message 390}: error - npt/nst ensemble requested in non-periodic system}

A non-periodic system has no defined volume, hence the NPT
algorithm cannot be applied.

\noindent \underline{\em Action}:

Either simulate the system with a periodic boundary, or use
another ensemble.

\subsubsection*{\underline{Message 402}: error - van der waals not specified}

The user has not set any cutoff in CONTROL, ({\tt rvdw}) - the van
der Waals\index{potential!van der Waals} potentials cutoff is
needed in order for \D to proceed.

\noindent \underline{\em Action}:

Supply a cutoff value for the van der Waals terms in the CONTROL
file using the directive {\tt rvdw}, and resubmit job.

\subsubsection*{\underline{Message 410}: error - cell not consistent with image convention}

The simulation cell vectors appearing in the CONFIG file are not
consistent with the specified image convention.

\noindent \underline{\em Action}:

Locate the variable {\tt imcon} in the CONFIG file and correct to
suit the cell vectors.

\subsubsection*{\underline{Message 414}: error - conflicting ensemble options in CONTROL file}

\D has found more than one {\bf ensemble} directive in the CONTROL
file.

\noindent \underline{\em Action}:

Locate extra {\bf ensemble} directives in CONTROL file and remove.

\subsubsection*{\underline{Message 416}: error - conflicting force options in CONTROL file}

\D has found incompatible directives in the CONTROL file
specifying the electrostatic interactions options.

\noindent \underline{\em Action}:

Locate the conflicting directives in the CONTROL file and correct.

\subsubsection*{\underline{Message 430}: error - integration routine not available}

A request for a non-existent ensemble\index{ensemble} has been
made or a request with conflicting options that \D cannot deal
with.

\noindent \underline{\em Action}:

Examine the CONTROL and FIELD files and remove inappropriate
specifications.

\subsubsection*{\underline{Message 432}: error -  undefined tersoff potential}

This shows that \D has encountered an unfamiliar entry for Tersoff
potentials in FIELD.

\noindent \underline{\em Action}:

Correct FIELD and resubmit.

\subsubsection*{\underline{Message 433}: error - rcut must be specified for the Ewald sum precision}

When specifying the desired precision for the Ewald
sum\index{Ewald!summation} in the CONTROL file, it is also
necessary to specify the real space cutoff {\tt rcut}.

\noindent \underline{\em Action}:

Place the {\bf cut} directive {\em before} the {\bf ewald
precision} directive in the CONTROL file and rerun.

\subsubsection*{\underline{Message 436}: error - unrecognised ensemble}

An unknown ensemble option has been specified in the CONTROL file.

\noindent \underline{\em Action}:

Locate {\bf ensemble} directive in the CONTROL file and amend
appropriately.

\subsubsection*{\underline{Message 440}: error - undefined angular potential}

A form of angular potential has been requested which \D does not
recognise.

\noindent \underline{\em Action}:

Locate the offending potential in the FIELD file and remove.
Replace with one acceptable to \D if this is possible.
Alternatively, you may consider defining the required potential in
the code yourself.  Amendments to subroutines {\sc read\_field}
and {\sc angles\_forces} will be required.

\subsubsection*{\underline{Message 442}: error - undefined three-body potential}

A form of three-body potential has been requested which \D does
not recognise.

\noindent \underline{\em Action}:

Locate the offending potential in the FIELD file and remove.
Replace with one acceptable to \D if this is reasonable.
Alternatively, you may consider defining the required potential in
the code yourself. Amendments to subroutines {\sc read\_field} and
{\sc three\_body\_forces} will be required.

\subsubsection*{\underline{Message 443}: error - undefined four-body potential}

\D has been requested to process a four-body
potential\index{potential!four-body} it does not recognise.

\noindent \underline{\em Action}:

Check the FIELD file and make sure the keyword is correctly
defined.  Make sure that subroutine {\sc three\_body\_forces}
contains the code necessary to deal with the requested potential.
Add the code required if necessary, by amending subroutines {\sc
read\_field} and {\sc three\_body\_forces}.

\subsubsection*{\underline{Message 444}: error - undefined bond potential}

\D has been requested to process a bond
potential\index{potential!bond} it does not recognise.

\noindent \underline{\em Action}:

Check the FIELD file and make sure the keyword is correctly
defined. Make sure that subroutine {\sc bonds\_forces} contains
the code necessary to deal with the requested potential.  Add the
code required if necessary, by amending subroutines {\sc
read\_field} and {\sc bonds\_forces}.

\subsubsection*{\underline{Message 445}: error - r\_14 $>$ rcut in dihedrals\_forces}

The 1-4 coulombic scaling for a dihedral angle bonding cannot be
performed since the 1-4 distance has exceeded the system short range
interaction cutoff, {\tt rcut}, in subroutine {\sc
dihedral\_forces}.

\noindent \underline{\em Action}:

To prevent this error occurring again increase {\tt rcut}.

\subsubsection*{\underline{Message 446}: error - undefined electrostatic key in
dihedral\_forces}

The subroutine {\sc dihedral\_forces} has been requested to process
a form of electrostatic potential\index{potential!electrostatics} it
does not recognise.

\noindent \underline{\em Action}:

The error arises because the integer key {\tt keyfrc} has an
inappropriate value (which should not happen in the standard
version of \D).  Check that the FIELD file correctly specifies the
potential.  Make sure the version of {\sc dihedral\_forces} does
contain the potential you are specifying.  Report the error to the
authors if these checks are correct.

\noindent \underline{\em Action}:

To prevent this error occurring again increase {\tt rvdw}.

\subsubsection*{\underline{Message 447}: error - only one *shells* directive per molecule is allowed}

\D has found more than one shells entry per molecule in FIELD.

\noindent \underline{\em Action}:

Correct the erroneous part in FIELD and resubmit.

\subsubsection*{\underline{Message 448}: error - undefined dihedral potential}

A form of dihedral potential\index{potential!dihedral} has been
requested which \D does not recognise.

\noindent \underline{\em Action}:

Locate the offending potential in the FIELD file and remove.
Replace with one acceptable to \D if this is reasonable.
Alternatively, you may consider defining the required potential in
the code yourself.  Amendments to subroutines {\sc read\_field}
and {\sc dihedral\_forces} (and its variants) will be required.

\subsubsection*{\underline{Message 449}: error - undefined inversion potential}

A form of inversion potential\index{potential!inversion} has been
encountered which \D does not recognise.

\noindent \underline{\em Action}:

Locate the offending potential in the FIELD file and remove.
Replace with one acceptable to \D if this is reasonable.
Alternatively, you may consider defining the required potential in
the code yourself. Amendments to subroutines {\sc read\_field} and
{\sc inversions\_forces} will be required.

\subsubsection*{\underline{Message 450}: error - undefined tethering potential}

A form of tethering potential\index{potential!tether} has been
requested which \D does not recognise.

\noindent \underline{\em Action}:

Locate the offending potential in the FIELD file and remove.
Replace with one acceptable to \D if this is reasonable.
Alternatively, you may consider defining the required potential in
the code yourself.  Amendments to subroutines {\sc read\_field}
and {\sc tethers\_forces} will be required.

\subsubsection*{\underline{Message 451}: error - three-body potential cutoff undefined}

The cutoff radius for a three-body
potential\index{potential!three-body} has not been defined in the
FIELD file.

\noindent \underline{\em Action}:

Locate the offending three-body force potential in the FIELD file
and add the required cutoff.  Resubmit the job.

\subsubsection*{\underline{Message 452}: error - undefined vdw potential}

A form of vdw potential has been requested which \D does not
recognise.

\noindent \underline{\em Action}:

Locate the offending potential in the FIELD file and remove.
Replace with one acceptable to \D if this is reasonable.
Alternatively, you may consider defining the required potential in
the code yourself.  Amendments to subroutines {\sc read\_field},
{\sc vdw\_generate}* and {\sc dihedrals\_14\_vdw} will be required.

\subsubsection*{\underline{Message 453}: error - four-body potential cutoff undefined}

The cutoff radius for a four-body\index{potential!four-body}
potential has not been defined in the FIELD file.

\noindent \underline{\em Action}:

Locate the offending four-body force potential in the FIELD file
and add the required cutoff.  Resubmit the job.

\subsubsection*{\underline{Message 454}: error - unknown external field}

A form of external field potential has been requested which \D
does not recognise.

\noindent \underline{\em Action}:

Locate the offending potential in the FIELD file and remove.
Replace with one acceptable to \D if this is reasonable.
Alternatively, you may consider defining the required potential in
the code yourself.  Amendments to subroutines {\sc read\_field}
and {\sc external\_field\_apply} will be required.

\subsubsection*{\underline{Message 456}: error - external field xpis-ton is applied to a layer with at least one frozen particle}

For a layer to emulate a piston no particle constituting it must be frozen.

\noindent \underline{\em Action}:

Locate the offending site(s) in the FIELD file and unfreeze the particles.

\subsubsection*{\underline{Message 461}: error - undefined metal potential}

A form of metal potential has been requested which \D does not
recognise.

\noindent \underline{\em Action}:

Locate erroneous entry in the FIELD file and correct the potental
interaction to one of the allowed ones for metals in \D.

\subsubsection*{\underline{Message 462}: error - thermostat friction constant must be$~>~0$}

A zero or negative value for the thermostat\index{thermostat!}
friction constant has been encountered in the CONTROL file.

\noindent \underline{\em Action}:

Locate the {\bf ensemble} directive in the CONTROL file and assign a
positive value to the time constant.

\subsubsection*{\underline{Message 463}: error - barostat friction constant must be$~>~0$}

A zero or negative value for the barostat\index{barostat!}
friction constant has been encountered in the CONTROL file.

\noindent \underline{\em Action}:

Locate the {\bf ensemble} directive in the CONTROL file and assign a
positive value to the time constant.

\subsubsection*{\underline{Message 464}: error - thermostat relaxation time constant must be$~>~0$}

A zero or negative value for the thermostat\index{thermostat!}
relaxation time constant has been encountered in the CONTROL file.

\noindent \underline{\em Action}:

Locate the {\bf ensemble} directive in the CONTROL file and assign
a positive value to the time constant.

\subsubsection*{\underline{Message 466}: error - barostat relaxation time constant must be$~>~0$}

A zero or negative value for the barostat\index{barostat} relaxation
time constant has been encountered in the CONTROL file.

\noindent \underline{\em Action}:

Locate the {\bf ensemble} directive in the CONTROL file and assign
a positive value to the time constant.

\subsubsection*{\underline{Message 467}: error - rho must not be zero in valid buckingham potential}

User specified vdw type buckingham potential has a non-zero force
and zero rho constants.  Only both zero or both non-zero are allowed.

\noindent \underline{\em Action}:

Inspect the FIELD file and change the values in question appropriately.

\subsubsection*{\underline{Message 468}: error - r0 too large for snm potential with current cutoff}

The specified location (r0) of the potential minimum for a shifted
n-m potential exceeds the specified potential cutoff. A potential
with the desired minimum cannot be created.

\noindent \underline{\em Action}:

To obtain a potential with the desired minimum it is necessary to
increase the van der Waals cutoff.  Locate the {\tt rvdw}
directive in the CONTROL file and reset to a magnitude greater
than r0.  Alternatively adjust the value of r0 in the FIELD file.
Check that the FIELD file is correctly formatted.

\subsubsection*{\underline{Message 470}: error - n~$<$~m in definition of n-m potential}

The specification of a n-m potential in the FIELD file implies
that the exponent m is larger than exponent n.  (Not all versions
of \D are affected by this.)

\noindent \underline{\em Action}:

Locate the n-m potential in the FIELD file and reverse the order
of the exponents.  Resubmit the job.

\subsubsection*{\underline{Message 471}: error - rcut~$<$~2*rctbp (maximum cutoff for three-body potentials)}

The cutoff for the pair interactions is smaller than twice that
for the three-body interactions. This is a bookkeeping requirement
for \D.

\noindent \underline{\em Action}:

Either use a smaller three-body cutoff, or a larger pair potential
cutoff.

\subsubsection*{\underline{Message 472}: error - rcut~$<$~ 2*rcfbp (maximum cutoff for four-body potentials)}

The cutoff for the pair interactions is smaller than twice that
for the four-body interactions. This is a bookkeeping requirement
for \D.

\noindent \underline{\em Action}:

Either use a smaller four-body cutoff, or a larger pair potential
cutoff.

\subsubsection*{\underline{Message 474}: error - conjugate gradient mimimiser cycle limit exceeded}

The conjugate gradient minimiser exceeded the iteration limit (100 for the relaxed shell model,
1000 for the configuration minimiser).

\noindent \underline{\em Action}:

Decrease the respective convergence criterion.  Alternatively,
you may try to increase the limit by hand in {\sc core\_shell\_relax}
or in {\sc minimise\_relax} respectively and recompile.  However,
it is unlikely that such measures will cure the problem as it is
more likely to lay in the physical description of the system being
simulated.  For example, are the core-shell spring constants well
defined?  Is the system being too far from equilibrium?

\subsubsection*{\underline{Message 476}: error - shells MUST all HAVE either zero or non-zero masses}

The polarisation of ions is accounted via a core-shell model as
the shell dynamics is either relaxed - shells have no mass, or
adiabatic - all shells have non-zero mass.

\noindent \underline{\em Action}:

Choose which model you would like to use in the simulated system
and adapt the shell masses in FIELD to comply with your choice.

\subsubsection*{\underline{Message 478}: error - shake algorithms (constraints \& pmf) failed to converge}

Your system has both bond and PMF constraints.  SHAKE (RATTLE\_VV1)
is done by combined application of both bond and PMF constraints
SHAKE (RATTLE\_VV1) in an iterative manner until the PMF constraint
virial converges to a constant.  No such convergence is achieved.

\noindent \underline{\em Action}:

See action notes on message 515 below.

\subsubsection*{\underline{Message 480}: error - PMF constraint length $>$ minimum of all half-cell widths}

The specified PMF length has exceeded the minimum of all half-cell
widths.

\noindent \underline{\em Action}:

Specify shorter PMF length or increase MD cell dimensions.

\subsubsection*{\underline{Message 484}: error - only one potential of mean force permitted}

Only one potential of mean force is permitted in FIELD.

\noindent \underline{\em Action}:

Correct the erroneous entries in FIELD.

\subsubsection*{\underline{Message 486}: error - only one of the PMF units is permitted to have frozen atoms}

Only one of the PMF units is permitted to have frozen atoms.

\noindent \underline{\em Action}:

Correct the erroneous entries in FIELD.

\subsubsection*{\underline{Message 488}: error - too many PMF constraints per domain}

This should not happen.

\noindent \underline{\em Action}:

Is the use of PMF constraints in your system physically sound?

\subsubsection*{\underline{Message 490}: error - local PMF constraint not found locally}

This should not happen.

\noindent \underline{\em Action}:

Is your system physically sound, is your system equilibrated?

\subsubsection*{\underline{Message 492}: error - a diameter of a PMF unit $>$ minimum of all half cell widths}

The diameter of a PMF unit has exceeded the minimum of all half-cell
widths.

\noindent \underline{\em Action}:

Consider the physical concept you are trying to imply in the
simulation.  Increase MD cell dimensions.

\subsubsection*{\underline{Message 494}: error - overconstrained PMF units}

PMF units are oveconstrained.

\noindent \underline{\em Action}:

\D algorithms cannot handle overconstrained PMF units.
Decrease the number of constraints on the PMFs.

\subsubsection*{\underline{Message 497}: error - pmf\_quench failure}

\noindent \underline{\em Action}:

See notes on message 515 below.

\subsubsection*{\underline{Message 498}: error - shake algorithm (pmf\_shake) failed to converge}

\noindent \underline{\em Action}:

See action notes on message 515 below.

\subsubsection*{\underline{Message 499}: error - rattle algorithm (pmf\_rattle) failed to converge}

See notes on message 515 below.

\noindent \underline{\em Action}:

See action notes on message 515 below.

\subsubsection*{\underline{Message 500}: error - PMF unit of zero length is not permitted}

PMF unit of zero length is found in FIELD.  PMF units are either a
single atom or a group of atoms usually forming a chemical molecule.

\noindent \underline{\em Action}:

Correct the erroneous entries in FIELD.

\subsubsection*{\underline{Message 501}: error - coincidence of particles in PMF unit}

A PMF unit must be constituted of non-repeating particles!

\noindent \underline{\em Action}:

Correct the erroneous entries in FIELD.

\subsubsection*{\underline{Message 502}: error - PMF unit member found to be present more than once}

A PMF unit is a group of unique (distingushed) atoms/sites. No
repetition of a site is allowed in a PMF unit.

\noindent \underline{\em Action}:

Correct the erroneous entries in FIELD.

\subsubsection*{\underline{Message 504}: error - cutoff too large for TABLE\textbar\textbar TABBND file}

The requested cutoff exceeds the information in the TABLE file or
the TABBND cutoff is larger than half the system cutoff {\tt rcut}.

\noindent \underline{\em Action}:

In the case when this is received while reading TABLE, reduce
the value of the vdw cutoff ({\tt rvdw}) in the CONTROL file or
reconstruct the TABLE file.  In the case when this is received
while reading TABBND then specify a larger {\tt rcut} in CONTROL.

\subsubsection*{\underline{Message 505}: error - EAM metal densities or pair crossfunctions out of range}

The resulting densities or pair crossfunctions are not defined in the TABEAM file.

\noindent \underline{\em Action}:

Recreate a TABEAM file with wider interval of defined densities and pair cross functions.

\subsubsection*{\underline{Message 506}: error - EAM or MBPC metal densities out of range}

The resulting densities are not defined in the TABEAM file if EAM
is used or ill defined due to atoms nearly overlapping when MBPC
metal potential is in use..

\noindent \underline{\em Action}:

Recreate a TABEAM file with wider range of densities.

\subsubsection*{\underline{Message 507}: error - metal density embedding out of range}

In the case of EAM type of metal interactions this indicates
that the electron density of a particle in the system has
exceeded the limits for which the embedding function for
this particle's type is defined (as supplied in TABEAM.  In
the case of Finnis-Sinclair type of metal interactions, this
indicates that the density has become negative.

\noindent \underline{\em Action}:

Reconsider the physical sanity and validity of the metal
interactions in your system and this type of simulation.
You MUST change the interactions' parameters and/or the
way the physical base of your investigation is handled in
MD terms.

\subsubsection*{\underline{Message 508}: error - EAM metal interaction entry in TABEAM unspecified in FIELD}

The specified EAM metal interaction entry found in TABEAM is not specified in FIELD.

\noindent \underline{\em Action}:

For $N$ metal atom types there are $(5N+N^{2})/2$ EAM
functions in the TABEAM file.  One density ($N$) and one
embedding ($N$) function for each atom type and $(N+N^{2})/2$
cross-interaction functions.  Fix the table entries and resubmit.

\subsubsection*{\underline{Message 509}: error - duplicate entry for a pair interaction detected in TABEAM}

A duplicate cross-interaction function entry is detected
in the TABEAM file.

\noindent \underline{\em Action}:

Remove all duplicate entries in the TABEAM file and resubmit.

\subsubsection*{\underline{Message 510}: error - duplicate entry for a density function detected in TABEAM}

A duplicate density function entry is detected
in the TABEAM file.

\noindent \underline{\em Action}:

Remove all duplicate entries in the TABEAM file and resubmit.

\subsubsection*{\underline{Message 511}: error - duplicate entry for an embedding function detected in TABEAM}

A duplicate embedding function entry is detected
in the TABEAM file.

\noindent \underline{\em Action}:

Remove all duplicate entries in the TABEAM file and resubmit.

\subsubsection*{\underline{Message 512}: error - non-definable vdw/dpd interactions detected in FIELD}

A VDW corss-interaction was uncpecified and recovering it by using a mixing rule proved impossible due to type difference of the single species potentials.

\noindent \underline{\em Action}:

Rethink your FIELD file interactions before restarting the job with a new compatible FIELD and possibly CONTROL file.

\subsubsection*{\underline{Message 513}: error - particle assigned to non-existent domain in read\_config}

This can only happen if particle coordinates do not match the cell
parameters in CONFIG.  Probably, due to negligence or numerical
inaccuracy inaccuracy in generation of big supercell from a small
one.

\noindent \underline{\em Action}:

Make sure lattice parameters and particle coordinates marry each
other.  Increase accuracy when generating a supercell.

\subsubsection*{\underline{Message 514}: error - allowed image conventions are: 0, 1, 2, 3 and 6}

\D has found unsupported boundary condition specified in CONFIG.

\noindent \underline{\em Action}:

Correct your boundary condition or consider using \C.

\subsubsection*{\underline{Message 515}: error - rattle algorithm (constraints\_rattle) failed to converge}

The RATTLE\index{algorithm!RATTLE} algorithm for bond
constraints\index{constraints!bond} is iterative.  If the maximum
number of permitted iterations is exceeded, the program
terminates.  Possible causes include: incorrect force
field\index{force field} specification; too high a temperature;
inconsistent constraints (over-constraint) etc..

\noindent \underline{\em Action}:

You may try to increase the limit of iteration cycles in the
constraint subroutines by using the directive {\bf mxshak} and/or
decrease the constraint precision by using the directive {\bf shake}
in CONTROL.  But the trouble may be much more likely to be cured by
careful consideration of the physical system being simulated.  For
example, is the system stressed in some way? Too far from
equilibrium?

%\subsubsection*{\underline{Message 516}: error - the number nodes MUST be a power of 2 series number}
%
%The DaFT development for the SMPE evaluation in \D supports
%only k-vector grids with a power of two size in any dimension.
%Therefore this restricts the types of domain decomposition to
%$2^{x}~\times~2^{y}~\times~2^{z}$ processor grids.
%
%\noindent \underline{\em Action}:
%
%You must ensure \D execution on number of processors represented as
%$2^{x}~\times~2^{y}~\times~2^{z}$.

\subsubsection*{\underline{Message 517}: error - allowed configuration information levels are: 0, 1 and 2}

\D has found an erroneous configuration information level, $l: 0 .le. l .le. 2$,
(i) for the trajectory option in CONTROL or (ii) in the header of CONFIG.

\noindent \underline{\em Action}:

Correct the error in CONFIG and rerun.

\subsubsection*{\underline{Message 518}: error - control distances for variable timestep not intact}

\D has found the control distances for the variable timestep
algorithm to be in contention with each other.

\noindent \underline{\em Action}:

{\tt mxdis} MUST BE$~>~2.5 \times$ {\tt mndis}.  Correct in
CONTROL and rerun.

\subsubsection*{\underline{Message 519}: error - REVOLD is incompatible or does not exist}

Either REVOLD does not exist or its formatting is incompatible.

\noindent \underline{\em Action}:

Change the {\tt restart} option in CONTROL and rerun.

\subsubsection*{\underline{Message 520}: error - domain decomposition failed}

A \D check during the domain decomposition mapping has been violated.
The number of nodes allowed for imcon = 0 is only 1,2,4 and 8!
The number of nodes allowed for imcon = 6 is restricted to 2 along the
z direction!  The number of nodes should not be a prime number since
these are not factorisable/decomposable!

\noindent \underline{\em Action}:

You must ensure \D execution on a number of processors that complies
with the advise above.

\subsubsection*{\underline{Message 530}: error - pseudo thermostat thickness MUST comply with: 2 Angs~$<=$~thickness~$<$~a quarter of the minimum MD cell width}

\D has found a check violated while reading CONTROL.

\noindent \underline{\em Action}:

Correct accordingly in CONTROL and resubmit.

\subsubsection*{\underline{Message 540}: error - pseudo thermostat MUST only be used in bulk simulations, i.e. imcon MUST be 1, 2 or 3}

\D has found a check violated while reading CONTROL.

\noindent \underline{\em Action}:

Correct accordingly in CONTROL {nve} or in CONFIG ({\tt imcon})
and resubmit.

\subsubsection*{\underline{Message 551}: error - REFERENCE not found !!!}

The {\tt defect} detection option is used in conjunction with {\tt restart}
but no REFERENCE file is found.

\noindent \underline{\em Action}:

Supply a REFERENCE configuration.

\subsubsection*{\underline{Message 552}: error - REFERENCE must contain cell parameters !!!}

REFERENCE MUST contain cell parameters i.e. image convention MUST be {\bf imcon}~$=$~1, 2, 3 or 6.

\noindent \underline{\em Action}:

Supply a properly formatted REFERENCE configuration.

\subsubsection*{\underline{Message 553}: error - REFERENCE is inconsistent !!!}

An atom has been lost in transfer between nodes.  This should
never happen!

\noindent \underline{\em Action}:

Big trouble.  Report problem to authors immediately.

\subsubsection*{\underline{Message 554}: error - REFERENCE's format different from CONFIG's !!!}

REFERENCE complies to the same rules as CONFIG with the exception
that image convention MUST be {\bf imcon}~$=$~1, 2, 3 or 6.

\noindent \underline{\em Action}:

Supply a properly formatted REFERENCE configuartion.

\subsubsection*{\underline{Message 555}: error - particle assigned to non-existent domain in defects\_read\_reference}

See notes on message 513 above.

\noindent \underline{\em Action}:

See action notes on message 513 above.

\subsubsection*{\underline{Message 556}: error - too many atoms in REFERENCE file}

See notes on message 45 above.

\noindent \underline{\em Action}:

See action notes on message 45 above.

\subsubsection*{\underline{Message 557}: error - undefined direction passed to defects\_reference\_export}

See notes on message 42 above.

\noindent \underline{\em Action}:

See action notes on message 42 above.

\subsubsection*{\underline{Message 558}: error - outgoing transfer buffer exceeded in defects\_reference\_export}

See notes on message 38 above.

\noindent \underline{\em Action}:

See action notes on message 38 above.

\subsubsection*{\underline{Message 559}: error - incoming data transfer size exceeds limit in defects\_reference\_export}

See notes on message 38 above.

\noindent \underline{\em Action}:

See action notes on message 38 above.

\subsubsection*{\underline{Message 560}: error - rdef found to be $>$ half the shortest interatomic distance in REFERENCE}

The defect detection option relies on a cutoff, rdef, to define the
vicinity around a site (defined in REFERENCES) in which a particle
can claim to occupy the site.  Evidently, rdef MUST be $<$ half the
shortest interatomic distance in REFERENCE.

\noindent \underline{\em Action}:

Decrease the value of rdef at directive {\bf defect} in CONTROL.

\subsubsection*{\underline{Message 570}: error - unsupported image convention (0) for system expansion option nfold}

System expansion is possible only for system with periodicity on
their boundaries.

\noindent \underline{\em Action}:

Change the image convention in CONFIG to any other suitable periodic
boundary condition.

\subsubsection*{\underline{Message 580}: error - replay (HISTORY) option can only be used for structural property recalculation}

No structural property has been specified for this option to
activate itself.

\noindent \underline{\em Action}:

In CONTROL specify properties for recalculation (RDFs,z-density
profiles, defect detection) or alternatively remove the option.

\subsubsection*{\underline{Message 585}: error - end of file encountered in HISTORY file}

This means that the HISTORY file is incomplete in some way: Either
should you abort the replay (HISTORY) option or provide a fresh
HISTORY file before restart.

\noindent \underline{\em Action}:

In CONTROL specify properties for recalculation (RDFs,z-density
profiles, defect detection) or alternatively remove the option.

\subsubsection*{\underline{Message 590}: error - uknown minimisation type, only "force", "energy" and "distance" are recognised}

Configuration minimisation can take only these three criteria.

\noindent \underline{\em Action}:

In CONTROL specify the criterion you like followed by the needed arguments.

\subsubsection*{\underline{Message 600}: error - "impact" option specified more than once in CONTROL}

Only one instance of the "impact" option is allowed in CONTROL.

\noindent \underline{\em Action}:

Remove any extra instances of the "impact" option in CONTROL.

\subsubsection*{\underline{Message 610}: error - "impact" applied on particle that is either frozen, or the shell of a core-shell unit or part of a RB}

It is the user's responsibility to ensure that impact is
initiated on a "valid" particle.

\noindent \underline{\em Action}:

In CONTROL remove the "impact" directive or correct the particle
identity in it so that it complies with the requirements.

\subsubsection*{\underline{Message 620}: error - duplicate or mixed intra-molecular entries specified in FIELD}

The FIELD parser has detected an inconsistency in the
description of bonding interactions.  It is the user's
responsibility to ensure that no duplicate or mixed-up
intra-molecular entries are specified in FIELD.

\noindent \underline{\em Action}:

Look at the preceding warning message in OUTPUT and find out
which entry of what intra-molecular-like interaction is at
fault.  Correct the bonding description and try running again.

\subsubsection*{\underline{Message 625}: error - only one *rigid* directive per molecule is allowed}

\D has found more than one rigids entry per molecule in FIELD.

\noindent \underline{\em Action}:

Correct the erroneous part in FIELD and resubmit.

\subsubsection*{\underline{Message 630}: error - too many rigid body units specified}

This should never happen!  This indicates an erroneous FIELD file
or corrupted \D executable.  Unlike \C, \D does not have a
set limit on the number of rigid body types it can handle in any
simulation (this is not the same as the total number of RBs in the
system or per domain).

\noindent \underline{\em Action}:

Examine FIELD for erroneous directives, correct and resubmit.

\subsubsection*{\underline{Message 632}: error - rigid body unit MUST have at least 2 sites}

This is likely to be a corrupted FIELD file.

\noindent \underline{\em Action}:

Examine FIELD for erroneous directives, correct and resubmit.

\subsubsection*{\underline{Message 634}: error - rigid body unit MUST have at least one non-massless site}

No RB dynamics is possible if all sites of a body are massless as no rotational inertia can be defined!

\noindent \underline{\em Action}:

Examine FIELD for erroneous directives, correct and resubmit.

%\subsubsection*{\underline{Message 636}: error - rigid body unit MUST NOT have any frozen site}
%
%\D does not permit a site in a rigid body to be frozen,
%i.e. fixed in one location in space.
%
%\noindent \underline{\em Action}:
%
%Remove the �freeze� condition from the site concerned.
%Consider using a very high site mass to achieve a similar effect.

\subsubsection*{\underline{Message 638}: error - coincidence of particles in rigid body unit}

This indicates a corrupted FIELD file as all members of a RB unit
must be destinguishable from one another.

\noindent \underline{\em Action}:

Examine FIELD for erroneous directives, correct and resubmit.

\subsubsection*{\underline{Message 640}: error - too many rigid body units per domain}

\D limits the number of rigid body\index{rigid body} units
in the system to be simulated (actually, the number to be processed
by each node) and checks for the violation of this.  Termination will
result if the condition is violated.

\noindent \underline{\em Action}:

Use {\bf densvar} option in CONTROL to increase {\tt mxrgd}
(alternatively, increase it by hand in {\sc set\_bounds} and
recompile) and resubmit.

\subsubsection*{\underline{Message 642}: error - rigid body unit diameter $>$ rcut (the system cutoff)}

\D domain decomposition limits the size of a RB to a largest
diagonal $<$ system cutoff.  I.e. the largest RB type is still
within a linked cell volume.

\noindent \underline{\em Action}:

Increase cutoff.

\subsubsection*{\underline{Message 644}: error - overconstrained rigid body unit}

This is a very unlikely message which usually indicates a corrupted FIELD file
or unphysically overconstrained system.

\noindent \underline{\em Action}:

Decrease constraint on the system.  Examine FIELD for erroneous directives,
if any, correct and resubmit.

\subsubsection*{\underline{Message 646}: error - overconstrained constraint unit}

This is a very unlikely message which usually indicates a corrupted FIELD file
or unphysically overconstrained system.

\noindent \underline{\em Action}:

Decrease constraint on the system.  Examine FIELD for erroneous directives,
if any, correct and resubmit.

\subsubsection*{\underline{Message 648}: error - quaternion setup failed}

This error indicates that the routine {\sc q\_setup} has failed
in reproducing all the atomic positions in rigid units from the centre of mass
and quaternion vectors it has calculated.

\noindent \underline{\em Action}:

Check the contents of the CONFIG file.  \D builds its local body
description of a rigid unit type from the {\em first} occurrence of
such a unit in the CONFIG file.  The error most likely occurs
because subsequent occurrences were not sufficiently similar
to this reference structure.  If the problem persists increase
the value of {\tt tol} in {\sc q\_setup} and recompile.
If problems still persist double the value of {\tt dettest} in
{\sc rigid\_bodies\_setup} and recompile.  If you still
encounter problems contact the authors.

\subsubsection*{\underline{Message 650}: error - failed to find principal axis system}

This error indicates that the routine {\sc rigid\_bodies\_setup}
has failed to find the principal axis for a rigid unit.

\noindent \underline{\em Action}:

This is an unlikely error.  \D should correctly handle linear,
planar and 3-dimensional rigid units.  There is the remote
possibility that the unit has all of its mass-bearing particles
frozen while some of the massless are not or the unit has just
one mass-bearing particle.  Another, more likely, possibility,
in case of linear molecules is that the precision of the coordinates
of these linear molecules' constituentsi, as produced by the user,
is not good enough, which leads \D to accepting it as non-linear
while, in fact, it is and then failing at the current point.
It is quite possible, despite considered as wrong practice, that
the user defined system of linear RBs is, in fact, generated from
a system of CBs (3 per RB) which has not been run in a high enough
SHAKE/RATTLE tolerance accuracy (10\^-8 and higher may be needed).
Check the definition of the rigid unit in the CONFIG file,
if sensible report the error to the authors.

\subsubsection*{\underline{Message 655}: error - FENE bond breaking failure}

A FENE type bond was broken.

\noindent \underline{\em Action}:

Examine FIELD for erroneous directives, if any, correct and resubmit.

\subsubsection*{\underline{Message 660}: error - TABBND or PDF bond breaking failure}

A bond with potential defined in TABBND or for which intra-molecular
potential distribution is collected has exceeded its bondleghth limit.

\noindent \underline{\em Action}:

If there is a TABBND present, reconstruct TABBND with potentials defined
over larger cutoff and try again.  If bonds PDF are collected, increase
their cutoff value in CONTROL.

\subsubsection*{\underline{Message 1000}: error - working precision mismatch between FORTRAN90 and MPI implementation}

\D has failed to match the available modes of MPI precision for real
numbers to the defined in {sc kinds\_f90} FORTRAN90 working
precision {\tt wp} for real numbers.  {\tt wp} is a precompile
parameter.

\noindent \underline{\em Action}:

This simply mean that {\tt wp} must have been changed from its
original value to something else and the new value is not matched by
the {\tt mpi\_wp} variable in {\sc comms\_module}.  It is the user's
responsibility to ensure that {\tt wp} and {\tt mpi\_wp} are
compliant.  Make the necessary corrections to {sc kinds\_f90} and/or
{\sc comms\_module}.

\subsubsection*{\underline{Message 1001}: error - allocation failure in comms\_module $->$ gcheck\_vector}

\D has failed to find available memory to allocate an array or
arrays, i.e. there is lack of sufficient memory (per node) on the
execution machine.

\noindent \underline{\em Action}:

This may simply mean that your simulation is too large for the
machine you are running on.  Consider this before wasting time
trying a fix.  Try using more processing nodes if they are
available.  If this is not an option investigate the possibility
of increasing the heap size for your application.  Talk to your
systems support people for advice on how to do this.

\subsubsection*{\underline{Message 1002}: error - deallocation failure in comms\_module $->$ gcheck\_vector}

\D has failed to deallocate an array or arrays, i.e. to free
memory that is no longer in use.

\noindent \underline{\em Action}:

Talk to your systems support people for advice on how to manage
this.

\subsubsection*{\underline{Message 1003}: error - allocation failure in comms\_module $->$ gisum\_vector}

See notes on message 1001 above.

\noindent \underline{\em Action}:

See action notes on message 1001 above.

\subsubsection*{\underline{Message 1004}: error - deallocation failure in comms\_module $->$ gisum\_vector}

See notes on message 1002 above.

\noindent \underline{\em Action}:

See action notes on message 1002 above.

\subsubsection*{\underline{Message 1005}: error - allocation failure in comms\_module $->$ grsum\_vector}

See notes on message 1001 above.

\noindent \underline{\em Action}:

See action notes on message 1001 above.

\subsubsection*{\underline{Message 1006}: error - deallocation failure in comms\_module $->$ grsum\_vector}

See notes on message 1002 above.

\noindent \underline{\em Action}:

See action notes on message 1002 above.

\subsubsection*{\underline{Message 1007}: error - allocation failure in comms\_module $->$ gimax\_vector}

See notes on message 1001 above.

\noindent \underline{\em Action}:

See action notes on message 1001 above.

\subsubsection*{\underline{Message 1008}: error - deallocation failure in comms\_module $->$ gimax\_vector}

See notes on message 1002 above.

\noindent \underline{\em Action}:

See action notes on message 1002 above.

\subsubsection*{\underline{Message 1009}: error - allocation failure in comms\_module $->$ grmax\_vector}

See notes on message 1001 above.

\noindent \underline{\em Action}:

See action notes on message 1001 above.

\subsubsection*{\underline{Message 1010}: error - deallocation failure in comms\_module $->$ grmax\_vector}

See notes on message 1002 above.

\noindent \underline{\em Action}:

See action notes on message 1002 above.

\subsubsection*{\underline{Message 1011}: error - allocation failure in parse\_module $->$ get\_record}

See notes on message 1001 above.

\noindent \underline{\em Action}:

See action notes on message 1001 above.

\subsubsection*{\underline{Message 1012}: error - deallocation failure in parse\_module $->$ get\_record}

See notes on message 1002 above.

\noindent \underline{\em Action}:

See action notes on message 1002 above.

\subsubsection*{\underline{Message 1013}: error - allocation failure in angles\_module $->$ allocate\_angles\_arrays}

See notes on message 1001 above.

\noindent \underline{\em Action}:

See action notes on message 1001 above.

\subsubsection*{\underline{Message 1014}: error - allocation failure in bonds\_module $->$ allocate\_bonds\_arrays}

See notes on message 1001 above.

\noindent \underline{\em Action}:

See action notes on message 1001 above.

\subsubsection*{\underline{Message 1015}: error - allocation failure in core\_shell\_module $->$ \\
\noindent allocate\_core\_shell\_arrays}

See notes on message 1001 above.

\noindent \underline{\em Action}:

See action notes on message 1001 above.

\subsubsection*{\underline{Message 1016}: error - allocation failure in statistics\_module $->$ allocate\_statitics\_arrays}

See notes on message 1001 above.

\noindent \underline{\em Action}:

See action notes on message 1001 above.

\subsubsection*{\underline{Message 1017}: error - allocation failure in tethers\_module $->$ allocate\_tethers\_arrays}

See notes on message 1001 above.

\noindent \underline{\em Action}:

See action notes on message 1001 above.

\subsubsection*{\underline{Message 1018}: error - allocation failure in constraints\_module $->$ \\
\noindent allocate\_constraints\_arrays}

See notes on message 1001 above.

\noindent \underline{\em Action}:

See action notes on message 1001 above.

\subsubsection*{\underline{Message 1019}: error - allocation failure in external\_field\_module $->$ \\
\noindent allocate\_external\_field\_arrays}

See notes on message 1001 above.

\noindent \underline{\em Action}:

See action notes on message 1001 above.

\subsubsection*{\underline{Message 1020}: error - allocation failure in dihedrals\_module $->$ allocate\_dihedrals\_arrays}

See notes on message 1001 above.

\noindent \underline{\em Action}:

See action notes on message 1001 above.

\subsubsection*{\underline{Message 1021}: error - allocation failure in inversions\_module $->$ allocate\_inversion\_arrays}

See notes on message 1001 above.

\noindent \underline{\em Action}:

See action notes on message 1001 above.

\subsubsection*{\underline{Message 1022}: error - allocation failure in vdw\_module $->$ allocate\_vdw\_arrays}

See notes on message 1001 above.

\noindent \underline{\em Action}:

See action notes on message 1001 above.

\subsubsection*{\underline{Message 1023}: error - allocation failure in metal\_module $->$ allocate\_metal\_arrays}

See notes on message 1001 above.

\noindent \underline{\em Action}:

See action notes on message 1001 above.

\subsubsection*{\underline{Message 1024}: error - allocation failure in three\_body\_module $->$ \\
\noindent allocate\_three\_body\_arrays}

See notes on message 1001 above.

\noindent \underline{\em Action}:

See action notes on message 1001 above.

\subsubsection*{\underline{Message 1025}: error - allocation failure in config\_module $->$ allocate\_config\_arrays}

See notes on message 1001 above.

\noindent \underline{\em Action}:

See action notes on message 1001 above.

\subsubsection*{\underline{Message 1026}: error - allocation failure in site\_module $->$ allocate\_site\_arrays}

See notes on message 1001 above.

\noindent \underline{\em Action}:

See action notes on message 1001 above.

\subsubsection*{\underline{Message 1027}: error - allocation failure in tersoff\_module $->$ alocate\_tersoff\_arrays}

See notes on message 1001 above.

\noindent \underline{\em Action}:

See action notes on message 1001 above.

\subsubsection*{\underline{Message 1028}: error - deallocation failure in angles\_module $->$ deallocate\_angles\_arrays}

See notes on message 1002 above.

\noindent \underline{\em Action}:

See action notes on message 1002 above.

\subsubsection*{\underline{Message 1029}: error - deallocation failure in bonds\_module $->$ deallocate\_bonds\_arrays}

See notes on message 1002 above.

\noindent \underline{\em Action}:

See action notes on message 1002 above.

\subsubsection*{\underline{Message 1030}: error - deallocation failure in core\_shell\_module $->$ \\
\noindent deallocate\_core\_shell\_arrays}

See notes on message 1002 above.

\noindent \underline{\em Action}:

See action notes on message 1002 above.

\subsubsection*{\underline{Message 1031}: error - deallocation failure in tethers\_module $->$ \\
\noindent deallocate\_tethers\_arrays}

See notes on message 1002 above.

\noindent \underline{\em Action}:

See action notes on message 1002 above.

\subsubsection*{\underline{Message 1032}: error - deallocation failure in constraints\_module $->$ \\
\noindent deallocate\_constraints\_arrays}

See notes on message 1002 above.

\noindent \underline{\em Action}:

See action notes on message 1002 above.

\subsubsection*{\underline{Message 1033}: error - deallocation failure in dihedrals\_module $->$ \\
\noindent deallocate\_dihedrals\_arrays}

See notes on message 1002 above.

\noindent \underline{\em Action}:

See action notes on message 1002 above.

\subsubsection*{\underline{Message 1034}: error - deallocation failure in inversions\_module $->$ \\
\noindent deallocate\_inversions\_arrays}

See notes on message 1002 above.

\noindent \underline{\em Action}:

See action notes on message 1002 above.

\subsubsection*{\underline{Message 1035}: error - allocation failure in defects\_module $->$ allocate\_defects\_arrays}

See notes on message 1001 above.

\noindent \underline{\em Action}:

See action notes on message 1001 above.

\subsubsection*{\underline{Message 1036}: error - allocation failure in pmf\_module $->$ allocate\_pmf\_arrays}

See notes on message 1001 above.

\noindent \underline{\em Action}:

See action notes on message 1001 above.

\subsubsection*{\underline{Message 1037}: error - deallocation failure in pmf\_module $->$ deallocate\_pmf\_arrays}

See notes on message 1002 above.

\noindent \underline{\em Action}:

See action notes on message 1002 above.

\subsubsection*{\underline{Message 1038}: error - allocation failure in minimise\_module $->$ allocate\_minimise\_arrays}

See notes on message 1001 above.

\noindent \underline{\em Action}:

See action notes on message 1001 above.

\subsubsection*{\underline{Message 1039}: error - deallocation failure in minimise\_module $->$ \\
\noindent deallocate\_minimise\_arrays}

See notes on message 1002 above.

\noindent \underline{\em Action}:

See action notes on message 1002 above.

\subsubsection*{\underline{Message 1040}: error - allocation failure in ewald\_module $->$ ewald\_allocate\_kall\_arrays}

See notes on message 1001 above.

\noindent \underline{\em Action}:

See action notes on message 1001 above.

\subsubsection*{\underline{Message 1041}: error - allocation failure in langevin\_module $->$ \\
\noindent langevin\_allocate\_arrays}

See notes on message 1001 above.

\noindent \underline{\em Action}:

See action notes on message 1001 above.

\subsubsection*{\underline{Message 1042}: error - allocation failure in rigid\_bodies\_module $->$ \\
\noindent allocate\_rigid\_bodies\_arrays}

See notes on message 1001 above.

\noindent \underline{\em Action}:

See action notes on message 1001 above.

\subsubsection*{\underline{Message 1043}: error - deallocation failure in rigid\_bodies\_module $->$ \\
\noindent deallocate\_rigid\_bodies\_arrays}

See notes on message 1002 above.

\noindent \underline{\em Action}:

See action notes on message 1002 above.

\subsubsection*{\underline{Message 1044}: error - allocation failure in comms\_module $->$ gimin\_vector}

See notes on message 1001 above.

\noindent \underline{\em Action}:

See action notes on message 1001 above.

\subsubsection*{\underline{Message 1045}: error - deallocation failure in comms\_module $->$ gimin\_vector}

See notes on message 1002 above.

\noindent \underline{\em Action}:

See action notes on message 1002 above.

\subsubsection*{\underline{Message 1046}: error - allocation failure in comms\_module $->$ grmin\_vector}

See notes on message 1001 above.

\noindent \underline{\em Action}:

See action notes on message 1001 above.

\subsubsection*{\underline{Message 1047}: error - deallocation failure in comms\_module $->$ grmin\_vector}

See notes on message 1002 above.

\noindent \underline{\em Action}:

See action notes on message 1002 above.

\subsubsection*{\underline{Message 1048}: error - error - allocation failure in comms\_module $->$ grsum\_matrix}

See notes on message 1001 above.

\noindent \underline{\em Action}:

See action notes on message 1001 above.

\subsubsection*{\underline{Message 1049}: error - deallocation failure in comms\_module $->$ grsum\_matrix}

See notes on message 1002 above.

\noindent \underline{\em Action}:

See action notes on message 1002 above.

\subsubsection*{\underline{Message 1050}: error - sorted I/O base communicator not set}

Possible corruption if {\sc io\_module}.  This should never happen!

\noindent \underline{\em Action}:

Make sure you have a clean copy of \D, compiled without any suspicious
warning messages.  Contact authors if the problem persists.

\subsubsection*{\underline{Message 1053}: error - sorted I/O allocation error}

Your I/O buffer (and possibly batch) size is too big.

\noindent \underline{\em Action}:

Decrease the value of the I/O buffer (and possibly batch) size in
CONTROL and restart your job.

\subsubsection*{\underline{Message 1056}: error - unkown write option given to sorted I/O}

This should never happen!

\noindent \underline{\em Action}:

Contact authors if the problem persists.

\subsubsection*{\underline{Message 1059}: error -  unknown write level given to sorted I/O}

This should never happen!

\noindent \underline{\em Action}:

Contact authors if the problem persists.

\subsubsection*{\underline{Message 1060}: error - allocation failure in statistics\_module -> allocate\_statitics\_connect}

See notes on message 1001 above.

\noindent \underline{\em Action}:

See action notes on message 1001 above.

\subsubsection*{\underline{Message 1061}: error - allocation failure in statistics\_module -> deallocate\_statitics\_connect}

See notes on message 1001 above.

\noindent \underline{\em Action}:

See action notes on message 1001 above.

\subsubsection*{\underline{Message 1063}: error - allocation failure in vdw\_module $->$ allocate\_vdw\_table\_arrays}

See notes on message 1001 above.

\noindent \underline{\em Action}:

See action notes on message 1001 above.

\subsubsection*{\underline{Message 1066}: error - allocation failure in vdw\_module $->$ allocate\_vdw\_direct\_fs\_arrays}

See notes on message 1001 above.

\noindent \underline{\em Action}:

See action notes on message 1001 above.

\subsubsection*{\underline{Message 1069}: error - allocation failure in metal\_module $->$ allocate\_metal\_table\_arrays}

See notes on message 1001 above.

\noindent \underline{\em Action}:

See action notes on message 1001 above.

\subsubsection*{\underline{Message 1070}: error - allocation failure in ewald\_module $->$ ewald\_allocate\_kfrz\_arrays}

See notes on message 1001 above.

\noindent \underline{\em Action}:

See action notes on message 1001 above.

\subsubsection*{\underline{Message 1072}: error - allocation failure in bonds\_module -> allocate\_bond\_pot\_arrays}

See notes on message 1001 above.

\noindent \underline{\em Action}:

See action notes on message 1001 above.

\subsubsection*{\underline{Message 1073}: error - allocation failure in bonds\_module -> allocate\_bond\_dst\_arrays}

See notes on message 1001 above.

\noindent \underline{\em Action}:

See action notes on message 1001 above.

\subsubsection*{\underline{Message 1074}: error - allocation failure in angles\_module -> allocate\_angl\_pot\_arrays}

See notes on message 1001 above.

\noindent \underline{\em Action}:

See action notes on message 1001 above.

\subsubsection*{\underline{Message 1075}: error - allocation failure in angles\_module -> allocate\_angl\_dst\_arrays}

See notes on message 1001 above.

\noindent \underline{\em Action}:

See action notes on message 1001 above.

\subsubsection*{\underline{Message 1076}: error - allocation failure in dihedrals\_module -> allocate\_dihd\_pot\_arrays}

See notes on message 1001 above.

\noindent \underline{\em Action}:

See action notes on message 1001 above.

\subsubsection*{\underline{Message 1077}: error - allocation failure in dihedrals\_module -> allocate\_dihd\_dst\_arrays}

See notes on message 1001 above.

\noindent \underline{\em Action}:

See action notes on message 1001 above.

\subsubsection*{\underline{Message 1078}: error - allocation failure in inversions\_module -> allocate\_invr\_pot\_arrays}

See notes on message 1001 above.

\noindent \underline{\em Action}:

See action notes on message 1001 above.

\subsubsection*{\underline{Message 1079}: error - allocation failure in inversions\_module -> allocate\_invr\_dst\_arrays}

See notes on message 1001 above.

\noindent \underline{\em Action}:

See action notes on message 1001 above.

\subsubsection*{\underline{Message 1080}: error - allocation failure in greenkubo\_module -> allocate\_greenkubo\_arrays}

See notes on message 1001 above.

\noindent \underline{\em Action}:

See action notes on message 1001 above.

\subsubsection*{\underline{Message 1081}: error - allocation failure in dpd\_module -> allocate\_dpd\_arrays}

See notes on message 1001 above.

\noindent \underline{\em Action}:

See action notes on message 1001 above.
