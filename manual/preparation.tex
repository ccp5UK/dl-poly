\section{A Guide to Preparing Input Files}

The CONFIG file and the FIELD file can be quite large and unwieldy
particularly if a polymer or biological molecule is involved in the
simulation.  This section outlines the paths to follow when trying
to construct files for such systems.  The description of the \D
force field in Chapter \ref{force-field} is essential reading.  The
various utility routines mentioned in this section are described in
greater detail in the \C User Manual.  Many of these have been
incorporated into the DL\_POLY GUI\index{GUI} \cite{smith-gui} and may
be conveniently used from there.

\subsection {Inorganic Materials}

The utility {\sc genlat} can be used to construct the CONFIG file
for relatively simple lattice structures.  Input is interactive. The
FIELD file for such systems are normally small and can be
constructed by hand.  Otherwise, the input of force field data for
crystalline systems is particularly simple, if no angular forces are
required (notable exceptions to this are zeolites and silicate
glasses - see below).  Such systems require only the specification
of the atomic types and the necessary pair forces.  The reader is
referred to the description of the \D FIELD file for further details
(Section \ref{field-file}).

\D can simulate zeolites and silicate (or other) glasses.  Both
these materials require the use of angular forces to describe the
local structure correctly.  In both cases the angular terms are
included as {\em three-body terms}, the forms of which are described
in Chapter \ref{force-field}.  These terms are entered into the
FIELD file with the pair potentials.

An alternative way of handling zeolites is to treat the zeolite
framework as a kind of macromolecule (see below).  Specifying all
this is tedious and is best done computationally: what is required
is to determine the nearest image neighbours of all atoms and assign
appropriate bond and valence angle potentials.  What must be avoided
at all costs is specifying the angle potentials {\em without}
specifying bond potentials\index{potential!bond}.  In this case \D
will automatically cancel the non-bonded\index{potential!non-bonded}
forces between atoms linked via valence
angles\index{potential!valence angle} and the system will collapse.
The advantage of this method is that the calculation is likely to be
faster than using three-body\index{potential!three-body} forces.
This method is not recommended for amorphous systems.

\subsection{Macromolecules}

To set up force fields for macromolecules, or indeed any covalent
molecules, it is best to use \F ~- \WEF\index{WWW}.  It is a program
application tool developed to facilitate the construction of force
field models for biological molecules and other molecules with complex
geometries.  For instance proteins, carbohydrates, polymers and
networked molecules such as graphenes and organic cages.  {\bf Although
created to assist \D, \F is a separate program suite that requires
separate registration!}

The primary functions of \F are as follows:
\begin{enumerate}
\item {\bf Force field model converter:} \F converts the user’s atom
models, supplied in PDB file format, into input files that are
recognisable and ready to run with \C and \D programs with minimum
user’s intervention.  This basically involves the conversion of the
user’s atomic configuration in simple xyz coordinates into identifiable
atom types base on a particular user-selectable potential schemes and
then automatically generate the DL\_POLY configuration file (CONFIG),
the force field file (FIELD) and a generic control file (CONTROL).
\item {\bf Force field editor:} \F allows the user to edit or modify
parameters of a particular force field scheme in order to produce a
customised scheme that is specific to a particular simulation model.
In addition, the standard force field model framework can also be easily
modified.  For instance, introduction of pseudo points and rigid body
implementation to an otherwise standard potential scheme such as CHARMM
or AMBER, etc.
\item {\bf Force field library repertoire:} \F contains a range of
popular potential schemes (see below), all described in a single \F
format that are also easily recognisable by the user for maintenance
purposes.  Users can easily expand the existing library to include
other new molecules.
\end{enumerate}

\subsubsection*{Force Field Schemes}

The available force field schemes are as follows:

CHARMM - proteins, ethers, some lipids and carbohydrates.

AMBER - proteins and Glycam for carbohydrates.

OPLSAA - proteins

DREIDING - General force field for covalent molecules.

PCFF - Polyorganics and other covalent molecules.

\subsubsection*{Model Construction}

\F does not have feature to construct molecular models.  This can be
achieved by either using DL\_POLY GUI\index{GUI} \cite{smith-gui} or any
other standard molecular building packages.  The output files must be
converted into the PDB format.  In the case of proteins, these structures
are usually obtained from data banks such as PDB.  These ‘{\em raw} PBD’
files must first be preprocessed by the user before they are in a
‘{\em readable} format’ for \F.  To ensure this, it is advisable that
users take into consideration the following steps:
\begin{enumerate}
\item Decide on inclusion/exclusion of and if necessary manually delete
molecular residues that involve multiple occupancies in crystalline structures.
\item Usually, hydrogen atoms are not assigned in the raw PDB file.
The molecules must therefore be pre-filled with hydrogen atoms (protonated)
by using any standard packages available.  The user must ensure that proper
care is taken of terminal residues which must also be appropriately terminated.
\item Decide on the various charge states of some amino acids, such as
histidine (HIS), lysine (LYS), glutamic acid (GLU), etc., by adding or
deleting the appropriate hydrogen atoms.  Force field schemes such as CHARMM
will have different three-letter notations for amino acids of different charge
states, \F will automatically identify these differences and assign the
appropriate potential parameters accordingly.
\item For cysteine (CYS) molecules with disulphide bonds, thiolate hydrogen
atoms must be removed.  \F will automatically define disulphide bonds between
the molecules, provided the S-S distance is within a ‘sensible’ value.
\item \F does not solvate the molecules and it is the user's responsibility
to add water by using any standard package available (for example the
DL\_POLY GUI\index{GUI} \cite{smith-gui}).
\end{enumerate}

Fore more details or further information, please consult the \F manual and
website \\ \WEF\index{WWW}.

\subsection{Adding Solvent to a Structure}

The utility {\sc wateradd} adds water from an equilibrated
configuration of 256 SPC water molecules at 300 K to fill out the
MD cell.  The utility {\sc solvadd} fills out the MD box with
single-site solvent molecules from a fcc lattice.  The FIELD files
will then need to be edited to account for the solvent molecules
added to the file.

Hint: to save yourself some work in entering the non-bonded
interactions variables involving solvent sites to the FIELD file put
two bogus atoms of each solvent type at the end of the CONNECT\_DAT
file (for AMBER\index{AMBER} force-fields\index{force field}) the
utility {\sc ambforce} will then evaluate all the
non-bonded\index{potential!non-bonded} variables required by \D.
Remember to delete the bogus entries from the CONFIG file before
running \D.

\subsection{Analysing Results}

\D is not designed to calculate every conceivable property you
might wish from a simulation.  Apart from some obvious
thermodynamic quantities and radial distribution functions, it
does not calculate anything beyond the atomic trajectories.  You
must therefore be prepared to post-process the HISTORY file if you
want other information.  There are some utilities in the \D
package to help with this, but the list is far from exhaustive. In
time, we hope to have many more.  Our users are invited to submit
code to the \D {\em public} library to help with this.

The utilities available are described in the \C User
Manual.  Users should also be aware that many of these utilities are
incorporated into the DL\_POLY GUI\index{GUI} \cite{smith-gui}.
