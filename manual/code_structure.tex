\section{Source Code}
\label{source-code}

\subsection{Modularisation Principles}
\label{modularisation}

Modules in \D are constructed to define parameters and variables
(scalars and arrays) and/or develop methods that share much in
common.  The division is far from arbitrary and module
interdependence is reduced to minimum.  However, some dependencies
exist which leads to the following division by groups in
hierarchical order:
\begin{itemize}

\item {\bf precision module}:: {\sc kinds\_f90}

The precision module defines the working precision {\tt wp} of all
real variables and parameters in \D.  By default it is set to 64-bit
(double) precision.  If the precision is changed, the user must
check whether the specific platform supports it and make sure it is
allowed for in the MPI implementation.  If all is OK then the code
must be recompiled.

\item {\bf MPI module}:: {\sc mpi\_module}

The MPI module implements all MPI functional calls used in \D.  It
is only used when \D is to be compiled in serial mode.

\item {\bf communication module}:: {\sc comms\_module}
({\sc mpi\_module})

The communication module defines MPI related parameters and develops
MPI related functions and subroutines such as: initialisation and
exit; global synchronisation, sum, maximum and minimum;
node ID and number of nodes; simulation time.  It is dependent on
{\sc kinds\_f90} and on {\sc mpi\_module} if MPI is emulated for
\D compilation in serial mode.  The {\sc mpi\_module} implements
all MPI functional calls used in \D.

\item {\bf global parameters module}:: {\sc setup\_module}

The global parameters module holds all important global variables
and parameters (see above).  It is dependent on {\sc kinds\_f90}.

\item {\bf parse module}:: {\sc parse\_module}

The parse module develops several methods used to deal with
textual input: {\tt get\_line strip\_blanks lower\_case get\_word
word\_2\_real}.  Depending on the method dependencies on
{\sc kinds\_f90 comms\_module setup\_module domains\_module} are found.

\item {\bf development module}:: {\sc development\_module}

The development module contains several methods used to help with
testing and debugging \D.  Depending on the method dependencies on
{\sc kinds\_f90 comms\_module setup\_module domains\_module} are found.

\item {\bf netCDF module}:: {\sc netcdf\_module}

The netCDF module contains all important netCDF functions and global
variables in \D context.  It is dependent on {\sc kinds\_f90}.

\item {\bf I/O module}:: {\sc io\_module}

The I/O module contains all important global variables that define
the I/O methods and types used in the package and contains
basic routines essential for the I/O in \D.  It is dependent on
{\sc kinds\_f90}.

\item {\bf domains module}:: {\sc domains\_module}

The domains module defines DD parameters and maps the available
computer resources on a DD grid.  The module does not depend on
previous modules but its mapping subroutine is dependent on
{\sc kinds\_f90} and {\sc comms\_module}.

\item {\bf site module}:: {\sc site\_module}

The site module defines all site related arrays (FIELD) and is
dependent on {\sc kinds\_f90} only.  However, it also develops an
allocation method that is dependent on {\sc setup\_module}.

\item {\bf configuration module}:: {\sc config\_module}

The configuration module defines all configuration related arrays
(CONFIG) and is dependent on {\sc kinds\_f90} only.  However, it
also develops an allocation method that is dependent on
{\sc setup\_module}.

\item {\bf vnl module}:: {\sc vnl\_module}

The Verlet neighbour list (VNL) module defines all VNL related
control variables and arrays needed for the VNL conditional
update functionality, and is dependent on {\sc kinds\_f90} only.
However, it is assisted by a {\sc vnl\_check} routine that is
dependent on more modules.

\item {\bf defects modules}:: {\sc defects\_module defects1\_module}

The defects modules define all defects and configuration related
arrays (REFERENCE) and are dependent on {\sc kinds\_f90} only.
However, they also develop allocation methods that are dependent
on {\sc setup\_module}.

\item {\bf dpd module}:: {\sc dpd\_module}

The Dissipative Particle Dynamics (DPD) module defines all DPD related
control variables and arrays needed for the DPD functionality, and is
dependent on {\sc kinds\_f90} only.  However, it also develops an
allocation method that is dependent on {\sc setup\_module}.

\item {\bf electrostatic modules}:: {\sc ewald\_module mpoles\_module}

This modules define all variables and arrays needed for the
SPME (i) refreshment k-space driven properties (ii) and
multipola relectrostatics control variable and arrays in the
\D scope when.  They depend on {\sc kinds\_f90} and but their
allocation methods on {\sc setup\_module}.

\item {\bf {\em inter}-molecular interactions modules}:: {\sc
vdw\_module metal\_module \\
tersoff\_module three\_body\_module four\_body\_module}

The intermolecular modules define all variables and potential
arrays needed for the calculation of the particular interaction in
the \D scope.  They depend on {\sc kinds\_f90}.  Their allocation
methods depend on {\sc setup\_module}.

\item {\bf {\em extra}-molecular interactions modules}:: {\sc
kim\_module plumed\_module}

These modules define all variables, arrays and functions needed
OpenKIM and PLUMED integrable into \D plugins.  They depend on
{\sc kinds\_f90}.  Their allocation methods depend on {\sc setup\_module}.

\item {\bf {\em intra}-molecular interactions and site-related
modules}:: {\sc rdf\_module z\_density\_module \\ core\_shell\_module
constraints\_module pmf\_module rigid\_bodies\_module \\ tethers\_module
bonds\_module angles\_module dihedrals\_module inversions\_module}

These modules define all variables and potential or statistical
grid arrays needed for the calculation of the particular interaction
or distribution function in the \D scope.  They all depend on {\sc kinds\_f90}
with allocation methods depending on {\sc setup\_module}.

\item {\bf external field module}:: {\sc external\_field\_module}

This module defines all variables and potential arrays needed for
the application of an external field in the \D scope.  It depends
on {\sc kinds\_f90} and its allocation method on {\sc setup\_module}.

\item {\bf langevin module}:: {\sc langevin\_module}

This module defines all variables and arrays needed for the
application of NPT and N$\mat{\sigma}$T Langevin routines
in the \D scope.  It depends on {\sc kinds\_f90} and its
allocation method on {\sc setup\_module}.

\item {\bf minimise module}:: {\sc minimise\_module}

This module defines all variables and arrays needed for the
application of a Conjugate Gradient Method minimisation routine
in the \D scope.  It depends on {\sc kinds\_f90} and its
allocation method on {\sc setup\_module}.

\item {\bf msd module}:: {\sc msd\_module}

This module globalises a CONTROL variable.

\item {\bf statistics module}:: {\sc statistics\_module}

This module defines all variables and arrays needed for the
statistical accountancy of a simulation in \D.  It depends on
{\sc kinds\_f90} and its allocation methods on {\sc setup\_module}
and {\sc comms\_module}.

\item {\bf greenkubo module}:: {\sc greenkubo\_module}

This module defines all variables and arrays needed for calculation
of Green-Kubo relations during a simulation in \D.  It depends on
{\sc kinds\_f90} and its allocation methods on {\sc setup\_module}.

\item {\bf kinetic module}:: {\sc kinetic\_module}

The kinetic module contains a collection of routines for the
calculation of various kinetic properties.
It is dependent on {\sc kinds\_f90}.

\item {\bf DaFT module}:: {\sc gpfa\_module parallel\_fft}

These modules contain all necessary functionality for \D DaFT and
it GPFA 1D FFT dependence.  They have dependencies on {\sc kinds\_f90},
{\sc comms\_module.f90} and {\sc setup\_module.f90}.

\end{itemize}

\subsection{File Structure}
\label{file-structure}

Generally, the \D file structure can be divided into four groups
as follows:
\begin{itemize}
\item {\bf general files} in the {\em source} directory
\item {\bf SERIAL specific} files in the {\em source/SERIAL} directory
\end{itemize}
The files in each group are listed in hierarchal order as closely
as possible as examplified in the relevant \D Makefies in the
{\em build} subdirectory.  The further down the category the file,
the more dependent it is on the files listed above it.

\subsection{Module Files}

The \D module files contain all global variables (scalars and
arrays) and parameters as well as some general methods and generic
functions intrinsically related to the purpose or/and contents of
the specific module.  The file-names and the methods or/and
functions developed in them have self-explanatory names.  More
information of their purpose can be found in their headers.

The rest of files in \D are dependent on the module files in
various ways.  The dependency relation to a module file is
explicitly stated in the declaration part of the code.

\subsection{General Files}

The \D general files are common to both MPI and SERIAL version of
the code.  In most cases, they have self-explanatory names as their
order is matched as closely as possible to that occurring in the
main segment of the code - {\sc dl\_poly}.  Only the first five
files are exception of that rule; {\sc warning} and {\sc error} are
important reporting subroutines that have call points at various
places in the code, and {\sc numeric\_container}, and
{\sc spme\_container} are containers of simple functions and
subroutines related in some way to their purpose in the code.

\subsection{SERIAL Specific Files}

These implement an emulation of some general MPI calls used in \D
source code when compiling in serial mode as well as some modified
counterparts of the general files changed to allow for faster and/or
better memory optimised serial execution.  Names are
self-explanatory.

\subsection{Comments on MPI Handling}

Only a few files make explicit calls to MPI routines.

\subsection{Comments on {\sc setup\_module}}
\label{parameters}

The most important module, by far, is {\sc setup\_module}, which
holds the most important global parameters and variables (some of
which serve as ``parameters'' for global array bounds, set in {\sc
set\_bounds}). A brief account of these is given below:

\begin{tabbing}
X\=XXXXXXXXX\=XXXXXXXXXXXXXX\=\kill
\> {\bf parameter}   \> {\bf value}          \> {\bf function} \\
\>                   \>                      \> \\
\> {\tt DLP\_VERSION}\> string               \> version string - number \\
\> {\tt DLP\_RELEASE}\> string               \> release string - date \\
\>                   \>                      \> \\
\> {\tt pi}          \> 3.14159265358979312  \> $\pi$ constant \\
\> {\tt twopi}       \> 6.28318530717958623  \> $2 \pi$ constant \\
\> {\tt fourpi}      \> 12.56637061435917246 \> $4 \pi$ constant \\
\> {\tt sqrpi}       \> 1.772453850905588    \> $\sqrt[2]{\pi}$ constant \\
\> {\tt rtwopi}      \> 0.15915494309189535  \> $\frac{1}{2 \pi}$ constant \\
\> {\tt rt2}         \> 1.41421356237309515  \> $\sqrt[2]{2}$ constant \\
\> {\tt rt3}         \> 1.73205080756887719  \> $\sqrt[2]{3}$ constant \\
\> {\tt r4pie0}      \> 138935.4835          \> electrostatics conversion factor to internal units, i.e. $\frac{1}{4 \pi \epsilon_{o}}$ \\
\> {\tt boltz}       \> 0.831451115          \> Boltzmann constant in internal units \\
\>                   \>                      \> also used as Kelvin/Boltzmann energy unit (very rarely used) \\
\> {\tt engunit}     \> {\em variable}       \> the system energy unit \\
\> {\tt eu\_ev}      \> 9648.530821          \> eV energy unit (most used) \\
\> {\tt eu\_kcpm}    \> 418.4                \> kcal/mol energy unit (often used) \\
\> {\tt eu\_kjpm}    \> 100.0                \> kJouls/mol energy unit (rarely used) \\
\> {\tt prsunt}      \> 0.163882576          \> conversion factor for pressure from internal units to katms \\
\> {\tt tenunt}      \> 1.660540200          \> conversion factor for surface tension from internal units to dyn/cm \\
\>                   \>                      \> \\
\> {\tt del\_max}    \> 0.01                 \>  maximum bin sizes in Angstroms for distance grids \\
\> {\tt delth\_max}  \> 0.20                 \>  maximum bin sizes in degrees for angle grids \\
\>                   \>                      \> \\
\> {\tt nread}       \> 5                    \> main input channel \\
\> {\tt nconf}       \> 11                   \> configuration file input channel \\
\> {\tt nfield}      \> 12                   \> force field input channel \\
\> {\tt ntable}      \> 13                   \> tabulated potentials file input channel \\
\> {\tt nrefdt}      \> 14                   \> reference configuration input channel \\
\> {\tt nrite}       \> 6                    \> main output channel \\
\> {\tt nstats}      \> 21                   \> statistical data file output channel \\
\> {\tt nrest}       \> 22                   \> output channel accumulators restart dump file \\
\> {\tt nhist}       \> 23                   \> trajectory history file channel \\
\> {\tt ndefdt}      \> 24                   \> output channel for defects data file \\
\> {\tt nrdfdt}      \> 25                   \> output channel for RDF data \\
\> {\tt nzdfdt}      \> 26                   \> output channel for Z-density data file \\
\> {\tt nrsddt}      \> 27                   \> output channel for displacements data files \\
\> {\tt npdfdt}      \> 28                   \> output channel for raw PDF files \\
\> {\tt ngdfdt}      \> 29                   \> output channel for normalised RDF data files \\
\> {\tt nvafdt}      \> 28                   \> output channel for VAF files \\
\> {\tt nmpldt}      \> 29                   \> output channel for the PLOLES data file \\
\>                   \>                      \> \\
\> {\tt seed(1:3)}   \> {\em variable}       \> pair of seeds for the random number generator \\
\> {\tt lseed}       \> {\em variable}       \> logical swich on/off indicator for seeding \\
\>                   \>                      \> \\
\> {\tt mxsite}      \> {\em variable}       \> max number of molecular sites \\
\> {\tt mxatyp}      \> {\em variable}       \> max number of unique atomic types \\
\> {\tt mxtmls}      \> {\em variable}       \> max number of unique molecule types \\
\> {\tt mxexcl}      \> {\em variable}       \> max number of excluded interactions per atom \\
\> {\tt mxompl}      \> {\em variable}       \> max number of multipolar order specified \\
\> {\tt mximpl}      \> {\em variable}       \> max number of multipolar total momenta for this order \\
\> {\tt mxspl}       \> {\em variable}       \> SPME FFT B-spline order \\
\> {\tt mxspl1}      \> {\em variable}       \> SPME FFT B-spline possible extension when $r_{\rm pad}>0$ \\
\> {\tt kmaxa}       \> {\em variable}       \> SPME FFT amended array dimension (a direction) \\
\> {\tt kmaxb}       \> {\em variable}       \> SPME FFT amended array dimension (b direction) \\
\> {\tt kmaxc}       \> {\em variable}       \> SPME FFT amended array dimension (c direction) \\
\> {\tt kmaxa1}      \> {\em variable}       \> SPME FFT original array dimension (a direction) \\
\> {\tt kmaxb1}      \> {\em variable}       \> SPME FFT original array dimension (b direction) \\
\> {\tt kmaxc1}      \> {\em variable}       \> SPME FFT original array dimension (c direction) \\
\> {\tt mxtshl}      \> {\em variable}       \> max number of specified core-shell unit types in system \\
\> {\tt mxshl}       \> {\em variable}       \> max number of core-shell units per node \\
\> {\tt mxfshl}      \> {\em variable}       \> max number of related core-shell units (1+1) \\
\> {\tt mxtcon}      \> {\em variable}       \> max number of specified bond constraints in system \\
\> {\tt mxcons}      \> {\em variable}       \> max number of constraint bonds per a node \\
\> {\tt mxfcon}      \> {\em variable}       \> max number of related constraint units (6+1) \\
\> {\tt mxlshp}      \> {\em variable}       \> max number of shared particles per node \\
\>                   \>                      \> ${\tt Max} (2~\frac{\tt mxshl}{2},2~\frac{\tt mxcons}{2},\frac{{\tt mxlrgd}~*~{\tt mxrgd}}{2})$ \\
\> {\tt mxproc}      \> {\em variable}       \> number of neighbour nodes in DD hypercube (26) \\
\> {\tt mxtpmf(1:2)} \> {\em variable}       \> max number of specified particles in a PMF unit (1:2) \\
\> {\tt mxpmf}       \> {\em variable}       \> max number of PMF constraints per a node \\
\> {\tt mxfpmf}      \> {\em variable}       \> max number of related PMF units (1+1) \\
\> {\tt mxtrgd}      \> {\em variable}       \> max number of types RB units \\
\> {\tt mxrgd}       \> {\em variable}       \> max number of RB units per node \\
\> {\tt mxlrgd}      \> {\em variable}       \> max number of constituent particles of an RB unit \\
\> {\tt mxfrgd}      \> {\em variable}       \> max number of related RB units (1+1) \\
\> {\tt mxtteth}     \> {\em variable}       \> max number of specified tethered potentials in system \\
\> {\tt mxteth}      \> {\em variable}       \> max number of tethered atoms per node \\
\> {\tt mxftet}      \> {\em variable}       \> max number of related tether units (1+1) \\
\> {\tt mxpteth}     \> {\em variable}       \> max number of parameters for tethered potentials (3) \\
\> {\tt mxtbnd}      \> {\em variable}       \> max number of specified chemical bond potentials in system \\
\> {\tt mxbond}      \> {\em variable}       \> max number of chemical bonds per node \\
\> {\tt mxfbnd}      \> {\em variable}       \> max number of related chemical bonds (1+(6*(6+1))/2) \\
\> {\tt mxpbnd}      \> {\em variable}       \> max number of parameters for chemical bond potentials (4) \\
\> {\tt mxgbnd}      \> {\em variable}       \> max number of grid points in chemical bond pot. arrays ($> 1004$) \\
\> {\tt mxtang}      \> {\em variable}       \> max number of specified bond angle potentials in system \\
\> {\tt mxangl}      \> {\em variable}       \> max number of bond angles per node \\
\> {\tt mxfang}      \> {\em variable}       \> max number of related bond angles (1+(6*(6+1))/2) \\
\> {\tt mxpang}      \> {\em variable}       \> max number of parameters for bond angle potentials (6) \\
\> {\tt mxgang}      \> {\em variable}       \> max number of grid points in bond angle pot. arrays ($> 1004$) \\
\> {\tt mxtdih}      \> {\em variable}       \> max number of specified dihedral angle potentials in system \\
\> {\tt mxdihd}      \> {\em variable}       \> max number of dihedral angles per node \\
\> {\tt mxfdih}      \> {\em variable}       \> max number of related dihedral angles (1+((6-2)6*(6+1))/2) \\
\> {\tt mxpdih}      \> {\em variable}       \> max number of parameters for dihedral angle potentials (7) \\
\> {\tt mxgdih}      \> {\em variable}       \> max number of grid points in dihedral angle pot. arrays ($> 1004$) \\
\> {\tt mxtinv}      \> {\em variable}       \> max number of specified inversion angle potentials in system \\
\> {\tt mxinv}       \> {\em variable}       \> max number of inversion angles per node \\
\> {\tt mxfinv}      \> {\em variable}       \> max number of related inversion angles (1+(6*(6+1))/4) \\
\> {\tt mxpinv}      \> {\em variable}       \> max number of parameters for inversion angle potentials (3) \\
\> {\tt mxginv}      \> {\em variable}       \> max number of grid points in inversion angle pot. arrays ($> 1004$) \\
\> {\tt mxrdf }      \> {\em variable}       \> max number of pairwise RDF in system \\
\> {\tt mxgrdf}      \> {\em variable}       \> number of grid points for RDF and Z-density arrays ($> 1004$)\\
\> {\tt mxgele}      \> {\em variable}       \> max number of grid points for ewald exclusion potential arrays \\
\> {\tt mxgusr}      \> {\em variable}       \> number of grid points for umbrella sampling restraint's RDF ($> 1004$)\\
\> {\tt mxvdw}       \> {\em variable}       \> max number of van der Waals potentials in system \\
\> {\tt mxpvdw}      \> {\em variable}       \> max number of van der Waals potential parameters (5) \\
\> {\tt mxgvdw}      \> {\em variable}       \> max number of grid points in vdw potential arrays ($> 1004$) \\
\> {\tt mxmet}       \> {\em variable}       \> max number of metal potentials in system \\
\> {\tt mxmed}       \> {\em variable}       \> max number of metal density potentials in system \\
\> {\tt mxmds}       \> {\em variable}       \> max number of metal extra density potentials in system \\
\> {\tt mxpmet}      \> {\em variable}       \> max number of metal potential parameters (9) \\
\> {\tt mxgmet}      \> {\em variable}       \> max number of grid points in metal potential arrays ($> 1004$)\\
\> {\tt mxter}       \> {\em variable}       \> max number of Tersoff potentials in system \\
\> {\tt mxpter}      \> {\em variable}       \> max number of Tersoff potential parameters (11) \\
\> {\tt mxgter}      \> {\em variable}       \> max number of grid points in tersoff potential arrays ($> 1004$)\\
\> {\tt mxgrid}      \> {\em variable}       \> max number of grid points in potential arrays ($> 1004$)\\
\> {\tt mxtana}      \> {\em variable}       \> max number of PDFs per type \\
\> {\tt mxgana}      \> {\em variable}       \> max number of grid points for PDFs arrays \\
\> {\tt mxgbnd}      \> {\em variable}       \> max number of grid points for chemical bonds PDFs \\
\> {\tt mxgang}      \> {\em variable}       \> max number of grid points for bond angles PDFs \\
\> {\tt mxgdih}      \> {\em variable}       \> max number of grid points for dihedral angles PDFs \\
\> {\tt mxginv}      \> {\em variable}       \> max number of grid points for inversion angles PDFs \\
\> {\tt mxtbp}       \> {\em variable}       \> max number of three-body potentials in system \\
\> {\tt mx2tbp}      \> {\em variable}       \> array dimension of three-body potential parameters \\
\> {\tt mxptbp}      \> {\em variable}       \> max number of three-body potential parameters (5) \\
\> {\tt mxfbp}       \> {\em variable}       \> max number of four-body potentials in system \\
\> {\tt mx2fbp}      \> {\em variable}       \> array dimension of four-body potential parameters \\
\> {\tt mxpfbp}      \> {\em variable}       \> max number of four-body potential parameters (3) \\

\> {\tt mxpfld}      \> {\em variable}       \> max number of external field parameters (5) \\
\> {\tt mxstak}      \> {\em variable}       \> dimension of stack arrays for rolling averages \\
\> {\tt mxnstk}      \> {\em variable}       \> max number of stacked variables \\
\> {\tt mxlist}      \> {\em variable}       \> max number of atoms in the Verlet list on a node \\
\> {\tt mxcell}      \> {\em variable}       \> max number of link cells per node \\
\> {\tt mxatms}      \> {\em variable}       \> max number of local+halo atoms per node \\
\> {\tt mxatdm}      \> {\em variable}       \> max number of local atoms per node \\
\> {\tt mxbfdp}      \> {\em variable}       \> max dimension of the transfer buffer for deport functions \\
\> {\tt mxbfss}      \> {\em variable}       \> max dimension of the transfer buffer for statistics functions \\
\> {\tt mxbfxp}      \> {\em variable}       \> max dimension of the transfer buffer for export functions \\
\> {\tt mxbfsh}      \> {\em variable}       \> max dimension of the transfer buffer for shared units \\
\> {\tt mxbuff}      \> {\em variable}       \> max dimension of the principle transfer buffer \\
\>                   \>                      \> \\
\> {\tt zero\_plus}  \> {\em variable}       \> the machine representation of $+0$ at working precision \\
\> {\tt half\_plus}  \> {\em variable}       \> the machine representation of $+0.5\uparrow$ at working precision \\
\> {\tt half\_minus} \> {\em variable}       \> the machine representation of $+0.5\downarrow$ at working precision \\
\end{tabbing}
