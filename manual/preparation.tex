\section{A Guide to Preparing Input Files}

The CONFIG file and the FIELD file can be quite large and unwieldy
particularly if a polymer or biological molecule is involved in the
simulation.  This section outlines the paths to follow when trying
to construct files for such systems.  The description of the \D
force field in Chapter \ref{force-field} is essential reading.  The
various utility routines mentioned in this section are described in
greater detail in the DL\_POLY\_2 User Manual.  Many of these
have been incorporated into the \D Graphical User
Interface\index{Graphical User Interface} \cite{smith-gui} and may
be conveniently used from there.

\subsection {Inorganic Materials}

The utility {\sc genlat} can be used to construct the CONFIG file
for relatively simple lattice structures.  Input is interactive. The
FIELD file for such systems are normally small and can be
constructed by hand.  Otherwise, the input of force field data for
crystalline systems is particularly simple, if no angular forces are
required (notable exceptions to this are zeolites and silicate
glasses - see below).  Such systems require only the specification
of the atomic types and the necessary pair forces.  The reader is
referred to the description of the \D FIELD file for further details
(Section \ref{field-file}).

\D can simulate zeolites and silicate (or other) glasses.  Both
these materials require the use of angular forces to describe the
local structure correctly.  In both cases the angular terms are
included as {\em three-body terms}, the forms of which are described
in Chapter \ref{force-field}.  These terms are entered into the
FIELD file with the pair potentials.

An alternative way of handling zeolites is to treat the zeolite
framework as a kind of macromolecule (see below).  Specifying all
this is tedious and is best done computationally: what is required
is to determine the nearest image neighbours of all atoms and assign
appropriate bond and valence angle potentials.  What must be avoided
at all costs is specifying the angle potentials {\em without}
specifying bond potentials\index{potential!bond}.  In this case \D
will automatically cancel the non-bonded\index{potential!non-bonded}
forces between atoms linked via valence
angles\index{potential!valence angle} and the system will collapse.
The advantage of this method is that the calculation is likely to be
faster than using three-body\index{potential!three-body} forces.
This method is not recommended for amorphous systems.

\subsection{Macromolecules}

Simulations of proteins are best tackled using the package
DLPROTEIN \cite{dlprotein}\index{DLPROTEIN} which is an adaptation
of DL\_POLY specific to protein modelling.  However, you may
simulate proteins and other macromolecules with \D if you wish.
This is described below.

If you select a {\em protein} structure from a SEQNET file (e.g.
from the Brookhaven database), use the utility {\sc proseq} to
generate the file CONFIG.  This will then function as input for
\D.  Some caution is required here however, as the protein
structure may not be fully determined and atoms may be missing
from the CONFIG file.

If you have the ``edit.out'' file produced by AMBER\index{AMBER}
for your molecule use this as the CONNECT\_DAT input file for the
utility {\sc ambforce}.  {\sc ambforce} will produce the \D FIELD
and CONFIG files for your molecule.

If you do not have the ``edit.out'' file things are a little more
tricky, particularly in coming up with appropriate partial charges
for atomic sites.  However, there are a series of utilities that
will at least produce the CONNECT\_DAT file for use with {\sc
ambforce}.  We now outline these utilities and the order in which
they should be used.

If you have a structure from the Cambridge Structural database
(CSDB) then use the utility {\sc fraccon} to take fractional
coordinate data and produce a CONNECT\_DAT and ``ambforce.dat''
file for use with {\sc ambforce}.  Note that you will need to
modify {\sc fraccon} to get the AMBER\index{AMBER} names correct
for sites in your molecule.  The version of {\sc fraccon} supplied
with \D is specific to the valinomycin molecule.

If you require an all atom force field and the database file does
not contain hydrogen positions then use the utility {\sc fracfill}
in place of {\sc fraccon}.  {\sc fraccon} produces an output file
HFILL which should then be used as input for the utility {\sc
hfill}.  The {\sc hfill} utility fills out the structure with the
missing hydrogens.  (Note that you may need to know what the
atomic charges are in some systems, for example the AMBER charges
from the literature.)

Note: with minor modifications the utilities {\sc fracfill} and {\sc
fraccon} can be used on structures from databases other than the
Cambridge structural database.

\subsection{Adding Solvent to a Structure}

The utility {\sc wateradd} adds water from an equilibrated
configuration of 256 SPC water molecules at 300 K to fill out the
MD cell.  The utility {\sc solvadd} fills out the MD box with
single-site solvent molecules from a fcc lattice.  The FIELD files
will then need to be edited to account for the solvent molecules
added to the file.

Hint: to save yourself some work in entering the non-bonded
interactions variables involving solvent sites to the FIELD file put
two bogus atoms of each solvent type at the end of the CONNECT\_DAT
file (for AMBER\index{AMBER} force-fields\index{force field}) the
utility {\sc ambforce} will then evaluate all the
non-bonded\index{potential!non-bonded} variables required by \D.
Remember to delete the bogus entries from the CONFIG file before
running \D.

\subsection{Analysing Results}

\D is not designed to calculate every conceivable property you
might wish from a simulation.  Apart from some obvious
thermodynamic quantities and radial distribution functions, it
does not calculate anything beyond the atomic trajectories.  You
must therefore be prepared to post-process the HISTORY file if you
want other information.  There are some utilities in the \D
package to help with this, but the list is far from exhaustive. In
time, we hope to have many more.  Our users are invited to submit
code to the \D {\em public} library to help with this.

The utilities available are described in the DL\_POLY\_2 User
Manual.  Users should also be aware that many of these utilities are
incorporated into the DL\_POLY Graphical User
Interface\index{Graphical User Interface} \cite{smith-gui}.
