\documentclass[11pt,a4paper,twoside,dvipdfm,pdflatex]{report}

% Floats settings
% Alter some LaTeX defaults for better treatment of figures:
% See p.105 of "TeX Unbound" for suggested values.
% See pp. 199-200 of Lamport's "LaTeX" book for details.
% General parameters, for ALL pages:
\renewcommand{\topfraction}{0.9} % max fraction of floats at top
\renewcommand{\bottomfraction}{0.8} % max fraction of floats at bottom
% Parameters for TEXT pages (not float pages):
\setcounter{topnumber}{2}
\setcounter{bottomnumber}{2}
\setcounter{totalnumber}{4} % 2 may work better
\setcounter{dbltopnumber}{2} % for 2-column pages
\renewcommand{\dbltopfraction}{0.9} % fit big float above 2-col. text
\renewcommand{\textfraction}{0.07} % allow minimal text w. figs
% Parameters for FLOAT pages (not text pages):
\renewcommand{\floatpagefraction}{0.7} % require fuller float pages
% N.B.: floatpagefraction MUST be less than topfraction !!
\renewcommand{\dblfloatpagefraction}{0.7} % require fuller float pages
% remember to use [htp] or [htpb] for placement

\newcommand{\UnderscoreCommands}{\do\index}

% Layout settings
\usepackage[portrait]{geometry}
\setlength{\textheight}{236mm} \setlength{\textwidth}{165mm}
\setlength{\topmargin}{-7mm} \setlength{\headsep}{10mm}
\setlength{\headheight}{10mm} \setlength{\footskip}{10mm}
\setlength{\oddsidemargin}{-2mm} \setlength{\evensidemargin}{-2mm}

% Paragraph settings
\setlength{\parindent}{0.0cm}
\setlength{\parskip}{1ex}
\setcounter{secnumdepth}{5}

% General Packages
\usepackage{fix-cm,array,tabularx,multicol,dcolumn,makeidx,ifthen,enumitem}
% times - for TNR fonts

% General Language
\usepackage[english]{babel}

\usepackage[dvips]{graphicx}
\usepackage[usenames,dvipsnames]{color}
%pdfLATEX% \usepackage[update,prepend,pdftex]{epstopdf}
%pdfLATEX% \usepackage[pdftex]{graphicx,color}

% Define some eye-pleasing colors for this document
\definecolor{webgreen}{rgb}{0,0.5,0}
\definecolor{webbrown}{rgb}{0.6,0,0}
\definecolor{webblue}{rgb}{0,0,0.75}

\usepackage
% Driver
[dvipdfm, % pdftex,
% Configuration Options
raiselinks=true,
breaklinks=true,
plainpages=true,
backref=true,
pagebackref=true,
pdfpagelabels,
% Extension Options
encap,
colorlinks=true,
citecolor=webgreen, %defined above
linkcolor=webbrown, %defined above
urlcolor=webblue,   %defined above
% PDF-Specific Options
bookmarks=true,
bookmarksopen=true,
bookmarksopenlevel=1
% PDF Display & Information Options
pdfpagemode=UseOutlines,
pdfpagelayout=OneColumn, % TwoColumnRight
pdfstartpage=1,
pdfstartview=FitBV,
pdfdisplaydoctitle=true,
pdftitle={THE DL-POLY-4 USER MANUAL},
pdfauthor={I.T. Todorov},
pdfsubject={Guide to the DL-POLY-4 Package},
pdfcreator={I.T. Todorov},
pdfkeywords={General Purpose Molecular Dynamics (MD) Software;
Domain Decomposition (DD), Liked Cells (LC);
Symplectic Integration, Velocity Verlet (VV), Leapfrog Verlet (LFV),
Ensembles: NVE; NVEk - Evans; NVT - VV-Langevin Impuls \& LFV-Langevin,
Andersen, Berendsen, Nose-Hoover, NPT - Langevin, Berendesen,
Nose-Hoover, Martyna-Tuckerman-Klein, NsT (extended to constant
normal pressure \& surface area or surface tension) - Langevin,
Berendesen, Nose-Hoover, Martyna-Tuckerman-Klein;
User Defined Intra- \& Inter-molecular Force-field,
Frozen Particles, External Fields;
Constraint Bonds (CB) \& Potential of Mean Field (PMF) Dynamics
with SHAKE (LFV) \& RATTLE (VV),
DD Rigid Body (RB) Dynamics with Euler-Quaternions using
FIQA (LFV) \& NOSQUISH (VV) integration schemes;
Long-ranged Coulomb Interactions by Parallelised Smoothed
Particle Mesh Ewald (SPME) using Memory Distributed 3D FFT (DaFT),
Damped Force-shifted Coulomb, Ion Polarisation by Dinamical
(Adiabatic Shells) or Relaxed (Massless Shells) Shell Models;
Zero Temperature \& Conjugate Gradient Method (CGM) Minimisers;
Temperature rescaling, Boundary Pseudo Thermostats, Defect Detection Tool;
Optimised Parallel (ASCII \& netCDF Amber) I/O (includes MPI-I/O),
Verifiable \& User Extendable Source Code in Self-Contained \& Portable,
Modularised FORTRAN90, MPI Parallelisation \& Memory Distribution via DD},
]{hyperref}
%pdfLATEX% \DeclareGraphicsRule{.eps}{pdf}{.pdf}{`epstopdf #1}
%pdfLATEX% \pdfcompresslevel=9

\usepackage[all]{hypcap}

%General Pagestyle
\usepackage{fancyhdr}

%commands
\usepackage{xspace}
\newcommand{\D}{{DL\_POLY\_4}\xspace}
\newcommand{\WEB}{\href{http://www.ccp5.ac.uk/DL\_POLY/}{http://www.ccp5.ac.uk/DL\_POLY/}\xspace}
\newcommand{\FTP}{\href{ftp://ftp.dl.ac.uk/ccp5/DL\_POLY/}{ftp://ftp.dl.ac.uk/ccp5/DL\_POLY/}\xspace}
\newcommand{\vek}[1]{\mbox{$\underline{#1}$}}
\newcommand{\mat}[1]{\mbox{$\underline{\underline{\bf #1}}$}}

\makeindex

\begin{document}

%       T I T L E   P A G E

%pagestyle
\pagenumbering{alph} \setcounter{page}{0}
\pagestyle{empty}

\renewcommand{\chaptermark}[1]{\markboth{#1}{#1}} % remember chapter title
\renewcommand{\sectionmark}[1]{\markright{\thesection\ #1}} % section number and title

% The pdfdummystart counter is a workaround to get correct
% bookmarks for the pre-CHAPTERS in pdf files

\newcounter{pdfdummystart}

\refstepcounter{pdfdummystart}
\addcontentsline{toc}{chapter}{\underline{THE \D USER MANUAL}}
\begin{titlepage}
\title{\fontsize{20}{25} \selectfont THE~~\D~~USER~~MANUAL}
\author{{\large \bf I.T. Todorov \& W. Smith} \\ \\
STFC Daresbury Laboratory \\
Daresbury, Warrington WA4 4AD \\
Cheshire, England, United Kingdom}
\date{\bf Version 4.03.4,~~June 2012}
\clearpage
\end{titlepage}

\maketitle
\clearpage

%       P R E F A C E

%pagestyle
\pagestyle{fancy}
\lhead{\copyright STFC}
\chead{}
\rhead{Preface}
\lfoot{}
\cfoot{\thepage}
\rfoot{}
\pagenumbering{roman} \setcounter{page}{1}

\newpage
\refstepcounter{pdfdummystart}
\addcontentsline{toc}{section}{About \D}
\input about_dlpoly.tex

\newpage
\refstepcounter{pdfdummystart}
\addcontentsline{toc}{section}{Disclaimer}
\input disclaimer.tex

\newpage
\refstepcounter{pdfdummystart}
\addcontentsline{toc}{section}{Acknowledgements}
\input acknowledgements.tex

\newpage
\refstepcounter{pdfdummystart}
\addcontentsline{toc}{section}{Manual Notation}
\input manual_notation.tex

%       C O N T E N T S

%pagestyle
\rhead{Contents}

\newpage
\refstepcounter{pdfdummystart}
\addcontentsline{toc}{chapter}{\underline{Contents}}
\tableofcontents
\clearpage

\newpage
\refstepcounter{pdfdummystart}
\addcontentsline{toc}{chapter}{\underline{List of Tables}}
\listoftables
\clearpage

\refstepcounter{pdfdummystart}
\newpage
\addcontentsline{toc}{chapter}{\underline{List of Figures}}
\listoffigures
\clearpage

%       R E A L   M A N U A L

\definecolor{webbrown}{rgb}{0,0,0.75}

\cleardoublepage

%pagestyle
\pagenumbering{arabic} \setcounter{page}{1}
\rhead{Section \thesection}

\chapter{Introduction}
\label{introduction}
\vskip 12ex
\section*{Scope of Chapter}
This chapter describes the concept, design and directory structure
of \D and how to obtain a copy of the source code.
\setcounter{equation}{0}
\newpage

\input introduction.tex

\chapter{Force Fields}
\label{force-field}
\vskip 12ex
\section*{Scope of Chapter}
This chapter describes the variety of interaction potentials
available in \D.
\setcounter{equation}{0}
\newpage

\input force_fields.tex

\chapter{Integration Algorithms}
\label{integration-algorithms}
\vskip 12ex
\section*{Scope of Chapter}
This chapter describes the integration algorithms coded into \D.
\setcounter{equation}{0}
\newpage

\input integration_algorithms.tex

\chapter{Construction and Execution}
\label{conex}
\vskip 12ex
\section*{Scope of Chapter}
This chapter describes how to compile a working version of \D and
run it.
\setcounter{equation}{0}
\newpage

\input construction.tex
\input compilation.tex
\input preparation.tex
\input ewald_precision.tex
\input diagnostic.tex

\chapter{Data Files}
\label{files}
\vskip 12ex
\section*{Scope of Chapter}
This chapter describes all the input and output files for \D,
examples of which are to be found in the {\em data} sub-directory.
\setcounter{equation}{0}
\newpage

\input input_files.tex
\input output_files.tex

\chapter{The \D Parallelisation and Source Code}
\label{source}
\vskip 12ex
\section*{Scope of Chapter}
This chapter we discuss the \D parallelisation strategy, describe the
principles used in the \D modularisation of the source code and list
the file structure found in the {\em source} subdirectory.
\setcounter{equation}{0}
\newpage

\input parallelisation.tex
\input code_structure.tex

\chapter{Examples}
\label{data}
\vskip 12ex
\section*{Scope of Chapter}
This chapter describes the standard test cases for \D, the input
and output files for which are in the {\em data} sub-directory.
\setcounter{equation}{0}
\newpage

\input examples.tex

%       A P P E N D I C E S

%\cleardoublepage

%pagestyle
\rhead{Appendix \thechapter}

\appendix
\addcontentsline{toc}{chapter}{\underline{Appendices}}

\chapter{\D Periodic Boundary Conditions}
\input boundary_conditions.tex

\chapter{\D Macros}
\input macros.tex

\chapter{\D Makefiles}
\input makefiles.tex

\chapter{\D Error Messages and User Action}
\input error_messages.tex

\chapter{\D README}
\input readme.tex

% The pdfdummyend counter is a workaround to get correct
% bookmarks for the index & bibliography in pdf files

\newcounter{pdfdummyend}

%       B I B L I O G R A P H Y

\cleardoublepage

%pagestyle
\rhead{Bibliography}
%
\refstepcounter{pdfdummyend}
\addcontentsline{toc}{chapter}{Bibliography}

\bibliographystyle{dl_poly}
\bibliography{dl_poly}

%
%       I N D E X

\cleardoublepage
%
%pagestyle
\rhead{Index}

\refstepcounter{pdfdummyend}
\addcontentsline{toc}{chapter}{Index}

\renewcommand{\see}[2]{\mbox{} \mbox{\textit{see} #1}}
\printindex

\newpage

%pagestyle
\pagestyle{empty}
\cleardoublepage

\end{document}
