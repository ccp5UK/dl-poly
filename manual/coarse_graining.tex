\section{Using \D for preparing and simulating coarse-grain models}
\label{cg-intro}

One can use \D for preparing and running simulations of
(systematically) coarse-grained (CG) models by either
{\bf (i)} employing potentials of mean force (PMF); or
{\bf (ii)} using iteratively optimised effective force-fields
(FF).  These can be obtained by using \D as MD engine within
the framework of VOTCA package -
\href{http://www.votca.org/}{http://www.votca.org/}\index{WWW}.

Systematic coarse-graining (SCG) of an atomistic system
usually implies the application of methodological mapping
procedure for recovery of the average configurational
(often termed physical) and/or topological (often termed
chemical/force-field) related properties of the original
model system onto a new model system with significantly
reduced degrees of freedom.  Integrating out degrees of
freedom ultimately leads to loss of information.  However,
when this is done consistently and selectively so that
the important for the modeller information (intrinsic
information pertaining to interactions within the model
system that governs its behaviour towards phenomena of interest)
and properties are retained, then the much reduced
phase-space of the new model system allows for expansion to
large length-scales as well as reaching rapidly times-scales
for investigating microscopic phenomena by using the well-known
classical MD machinery for the new CG particles.

A SCG mapping in the configuration space can be carried out
by using {\bf (i)} the DL\_CGMAP utility -
\href{http://www.ccp5.ac.uk/software/DL\_CGMAP/}{http://www.ccp5.ac.uk/software/DL\_CGMAP/}\index{WWW},
or {\bf (ii)}  VOTCA tools -
\href{http://www.votca.org/}{http://www.votca.org/}\index{WWW}.
For more details about the mapping procedures, the user is
advised to consult with the manuals of these tools.

The mapping leads to the generation of an effective CG
force-field in terms of effective interaction potentials
between the CG particles, which are often tabulated numerically.

This section aims at describing how to use \D for two SCG tasks:

\begin{itemize}
\item Post-simulation analysis of the bonded mean-force
interactions, based on the calculation of the corresponding
intramolecular distributions.
\item Preparation and use of the tabulated potentials for the
intramolecular (bonded) interactions.
\end{itemize}

\noindent {\bf Note:} Although these steps are also
applicable to atomistic systems, below we shall assume
that the CG mapping has been done and the following data
files have been generated for the CG mapped system:
CONFIG, FIELD and, alternatively, HISTORY.

\section{Intramolecular Potential Distribution Function (PDF) Analysis}
\label{IPDF-analysis}

The analysis can be performed on the fly or on an already
existing trajectory (by invoking the {\bf replay history})
options, see Section~\ref{control-file}).
To trigger the PDF collection and subsequent analysis for
any combination of intarmolecular interactions; such as
bonds, angles, dihedrals and inversions; some additional
directives must be supplied in the CONTROL file as
demonstrated in the example below.

\begin{verbatim}
TITLE: EXAMPLE OF DL_POLY_4 PDF ANALYSIS DIRECTIVES SNIPPET

#DIRECTIVES TO INVOKE INTRAMOLECULAR PDF ANALYSIS BY TYPE
analyse  bonds       sample every 100  nbins 300  rmax 6.0
analyse  angles      sample every 100  nbins 314  # [ 0 : pi]
analyse  diherals    sample every 100  nbins 628  # [-pi: pi]
analyse  inversions  sample every 100  nbins 628  # [ 0 : pi]

#DIRECTIVES TO INVOKE INTRAMOLECULAR PDF ANALYSIS FOR ALL TYPES
analyse  all         sample every 100  nbins 300  rmax 6.0

#DIRECTIVES TO INVOKE INTRAMOLECULAR PDF ANALYSIS PRINTING
print analysis
\end{verbatim}

The {\bf analyse} directive acts in a similar manner to
that of the {\bf rdf} directive outlining {\bf (i)} the
frequency of sampling (in steps/frames) and {\bf (ii)} the
number of bins over the cutoff interval/size of the
specified interaction.  It is only for chemical bonds
that this cutoff needs specifying as for the rest of
the possible intramolecular interactions these are known
{\em a priory} by definition.  If no cutoff is supplied
for bonds then it defaults to 2~\AA.  It is worth noting
that the number of bins per PDF must be larger than
$N = {\tt Nint} \left[ \frac{\tt size~of~PDF~interval}{\tt minimum~bin~size} \right]$
or otherwise it will default to ${\tt Max}(N,{\tt grid})$, where
{\tt grid} is the grid size of the corresponding intramolecular
interaction if a potential of it is defined in a tabulated
form or otherwise $0$.  The minimum bin size for distance
dependent interactions is 0.01~\AA~and $\pi/1000$ for angle
dependent interactions.  When {\bf analysis all} is used in
conjunction with the specific directives then it will
trigger the analysis on all PDFs as well as enforce on the
specifically targeted for analysis PDFs {\bf (i)} its
collection interval only when it is smaller than and
{\bf (ii)} its grid size only when it is larger than
the corresponding entities of the targeted for analysis
PDFs.  In the case of bonds it will also enforce the
{\bf rmax} grid size only when larger than the one
specified in the {\bf analysis bonds} directive.

While statistics is collected for all opted PDFs, printing
it will only occur when opted for by the {\bf print analysis}
directive.  The result of it is the creation of two data files
for each of the targeted distributions: BNDDAT \& PMFBND -
for bonds, ANGDAT \& PMFANG - for angles, DIHDAT \& PMFDIH -
for dihedrals and INVDAT \& PMFINV - for inversions.

Partial examples of the *DAT files for bonds and angles PDFs
are given below.

\begin{verbatim}
[user@host]$ more BNDDAT
# TITLE: Hexane FA OPLSAA -> CG mapped with 3 beads (A-B-A)
# types, bins, cutoff, frames: 1       300      2192    3.000     12
#
# r(Angstroms)  Pn_bond(r)  PDF_bond(r)   @   dr_bin  =  0.02000
#
# id, type, totals: A       B            30
    0.01000  0.000000E+00  0.000000E+00
    0.03000  0.000000E+00  0.000000E+00
    0.05000  0.000000E+00  0.000000E+00
    0.07000  0.000000E+00  0.000000E+00
...

[user@host]$ more ANGDAT
# TITLE: Hexane FA OPLSAA -> CG mapped with 3 beads (A-B-A)
# types, bins, cutoff, frames: 1       314      2192    180     24
#
# Theta(degrees)  Pn_ang(Theta)  Pn_ang(Theta)/Sin(Theta)   @   dTheta_bin =  0.57325
#
# id, type, totals:  A       B       A                1         50
    0.28662  0.000000E+00  0.000000E+00
    0.85987  0.000000E+00  0.000000E+00
    1.43312  0.000000E+00  0.000000E+00
    2.00637  0.000000E+00  0.000000E+00
...
\end{verbatim}

One can see that all the header lines are commented out due
to starting with the hash symbol {\bf \#}.  Nonetheless, the
header contains some useful information about the sampling
procedure and the distributions.  The title is, as usual,
placed in the first line, followed by a line with the four
most important descriptors -- the numbers of unique {\em types}
(of bonds, angles, dihedrals or inversions) analysed, the
numbers of {\em bins} in the (histogram-like) grid, the
{\em cutoff} interval (absolute size) over which the
distributions are sampled, and the number of sampled
{\em frames} the PDF data is derived upon.  An empty line
follows after which a line specifies what is printed as
data further on below and gives the grid bin size.  Again,
after an empty line there is a line containing the unique
unit {\em id} specification in terms of the participating
atom names, the {\em type} counter for the unique types, and
the {\em totals}, which is the number of tracked units of
this type in the model system this analysis is carried out on.

In all PDFs *DAT files the first column contains the abscissa
bin-centered distance or angle values), the second column
is the distribution normalized so that it can be directly
used for the PMF calculation, whereas the third column is
the original histogram normalised to unity.

The OUTPUT file will also reprint copies of the first and
third columns of all PDFs *DAT files, plus a column
accumulating the normalised sum for each histogram.

In addition to the PDFs *DAT, \D also calculates the
potentials of mean force, which are stored in the PMF*
files.  They can be directly used as numerically defined
force-field (tables) upon renaming: PMFBND~$\to$~TABBND,
PMFANG~$\to$~TABANG, PMFDIH~$\to$~TABDIH and PMFINV~$\to$~TABINV.
{\bf The user is, however, strongly advised to check
the quality of the obtained tabulated force-field before
using those as input for a CG simulation.}

\section{Setting up Tabulated Intramolecular Force-Fields}
\label{bonded-tables}

As mentioned above, for a coarse-grained simulation the
effective potentials must be provided in a tabulated form.
For non-bonded (short-range) interactions the TABLE file
must be prepared as described in Section~\ref{table-file}.
However, the tabulated data format for intramolecular
interactions (chemical bonds, bond angles, dihedral and
inversion angles) differs from that of the TABLE file and
assumes three columns: abscissa (distance in \AA~or angle
in degrees), and then potential and virial or force data
(virial for distance dependent interactions - i.e. TABBND,
and force for angular dependent interactions - i.e. TABANG,
TABDIH or TABINV).  Shown below are examples of a TABBND
and a TABANG files.

\begin{verbatim}
[user@host]$ more TABBND
# TITLE: Hexane FA OPLSAA -> CG mapped with 3 beads (A-B-A)
# 10 1000
#
# A B
 1.00000e-02  -9.0906600e+02   1.3954000e+00
 2.00000e-02  -9.1046200e+02   2.7908000e+00
 3.00000e-02  -9.1185700e+02   4.1862000e+00
 4.00000e-02  -9.1325300e+02   5.5816000e+00
...

[user@host]$ more TABANG
# TITLE: Hexane FA OPLSAA -> CG mapped with 3 beads (A-B-A)
# 1000
#
# A B A
 1.80000e-01   8.8720627e+01   6.9119576e-01
 3.60000e-01   8.8596227e+01   6.9119227e-01
 5.40000e-01   8.8471827e+01   6.9118704e-01
 7.20002e-01   8.8347327e+01   6.9118180e-01
...
\end{verbatim}

The tables for bonds, angles, dihedrals and inversions
are named TABBND, TABANG, TABDIH and TABINV correspondingly.
The format of these files is fixed in terms of the line
(record) order.  In particular, the initial two header
lines must contain a title and a record with the grid
specification, as all following blocks of tabulated data
must be preceded by an empty line and a one-line header
record containing the (white-space delimited) names
of the atoms making up the given intermolecular interaction
unit.  All the header lines preceding the tabulated data
are commented out, which eases plotting of the data when
using [Xm]Grace software.

In the TABBND file the second line specifies first the
``bond cutoff'' (the maximum abscissa value $\le r_{\rm cut}$ in \AA)
before the number of grid bins.  Unlike in the TABLE file,
the bin size should not be specified as it is calculated
based on the provided two numbers.  {\bf Note} that all
potential and virial data are assumed to be provided in
the energy units as specified by the user in the FIELD
file (see Section~\ref{field-file}) with the abscissa units
of \AA~(distance).

In the TABANG, TABDIH and TABINV files the second line
specifies only the number of grid bins, while the angle
ranges covered by the data are assumed as $0 < {\rm TABANG \& TABINV} \le 180$
and  $-180 < {\rm TABDIH} \le 180$.  {\bf Note} that all
potential and virial data are assumed to be provided in
the energy units as specified by the user in the FIELD
file (see Section~\ref{field-file}) with the abscissa units
of {\em degrees} (angle), which are internally converted
to and then handled by \D in {\em radians}.

Finally, in order to instruct \D to use tabulated
intramolecular interactions from these files the user
has to specify in the FIELD file for the particular
intramolecular interaction potential (type and id) that
it is tabulated by using the keyword {\bf tab} or {\bf -tab}
(the latter is only valid for bonds and angles) in a similar
manner to how all other interaction types are specified
(see Section~\ref{field-file}).  The dash symbol (-) in
front of the keyword {\bf tab} is interpreted in the
same manner as in Table~\ref{bond-table} and Table~\ref{angle-table}.
