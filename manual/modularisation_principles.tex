\section{The \D Modularisation Principles}
\label{modularisation-principles}

Modules in \D are constructed to define parameters and variables
(scalars and arrays) and/or develop methods that share much in
common.  The division is far from arbitrary and module
interdependence is reduced to minimum.  However, some dependencies
exist which leads to the following division by groups in
hierarchical order:
\begin{itemize}

\item {\bf precision module}:: {\sc kinds\_f90}

The precision module defines the working precision {\\tt wp} of
all real variables and parameters in \D.  By default it is set to
KIND~$=~16$.  If the precision is changed the code must be
recompiled.  Before users change to higher precision they must
check wether the specific platform supports such.

\item {\bf communication module}:: {\sc comms\_module}

The communication module defines MPI related parameters ({\tt
mpi\_wp} dependent on {\tt wp} in {\sc kinds\_f90}) and develops
MPI related functions and subroutines such us: initialisation and
exit; global sum, maximum and minimum; node id and number of
nodes.  It is dependent on {\sc kinds\_f90}.

\item {\bf global parameters module}:: {\sc setup\_module}

The global parameters module holds all important global variables
and parameters (see above).  It is dependent on {\sc kinds\_f90}

\item {\bf domains module}:: {\sc domains\_module}

The domains module defines DD parameters and maps the available
computer resources on a DD grid.  The module does not depend on
previous modules but its mapping subroutine is dependent on all of
them.

\item {\bf parse module}:: {\sc parse\_module}

The parse module develops several methods used to deal with
textual input: {\tt get\_line strip\_blanks lower\_case get\_word
word\_2\_real}.  Depending on the method dependencies on {\sc
kinds\_f90 comms\_module setup\_module domains\_module} are found.

\item {\bf site module}:: {\sc site\_module}

The site module defines all site related arrays and scalars (FIELD)
and is dependent on {\sc kinds\_f90} only.  However, it also develops
an allocation method that is dependent on {\sc setup\_module}.

\item {\bf configuration module}:: {\sc config\_module}

The configuration module defines all configuration related arrays
and scalars (CONFIG) and is dependent on {\sc kinds\_f90} only.
However, it also develops an allocation method that is dependent
on {\sc setup\_module}.

\item {\bf defects module}:: {\sc defects\_module}

The defects module defines all reference configuration related
arrays and scalars (REFERENCE) used for defects tracking and is
dependent on {\sc kinds\_f90} only.  However, it also develops
an allocation method that is dependent on {\sc setup\_module}.

\item {\bf {\em inter}-molecular interactions modules}:: {\sc
vdw\_module metal\_module \\
tersoff\_module three\_body\_module four\_body\_module}

The intermolecular modules define all variables and potential
arrays needed for the calculation of the particular interaction in
the \D scope.  They depend on {\sc kinds\_f90}.  Their allocation
methods depend on {\sc setup\_module}.

\item {\bf {\em intra}-molecular and site-related interactions
modules}:: {\sc core\_shell\_module constraints\_module
tethers\_module bonds\_module angles\_module \\
dihedrals\_module inversions\_module}

These modules define all variables and potential arrays needed for
the calculation of the particular interaction in the \D scope.
They depend on {\sc kinds\_f90}.  Their allocation methods depend
on {\sc setup\_module}.

\item {\bf external field module}:: {\sc external\_field\_module}

This module defines all variables and potential arrays needed for
the application of an external field in the \D scope.  It depends
on {\sc kinds\_f90} and its allocation method on {\sc
setup\_module}.

\item {\bf statistics module}:: {\sc statistics\_module}

This module defines all variables and arrays needed for the
statistical accountancy of a simulation in \D.  It depends on {\sc
kinds\_f90} and its allocation method on {\sc setup\_module}.

\end{itemize}
