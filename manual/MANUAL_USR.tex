\documentclass[11pt,a4paper,dvipdfm]{report}
\setcounter{secnumdepth}{5}

%packages
\usepackage{makeidx}
%\usepackage{epstopdf} %if pdflatex $1.tex in makedoc
\usepackage[dvips]{graphicx}
\usepackage[usenames,dvipsnames]{color}
\usepackage
[dvipdfm,
%------------- Back referencing Info --------------------
pagebackref, %or backref
%------------ Doc View ----------------------------------
bookmarksopen=true,
bookmarks=true,
pdfpagemode=UseOutlines,
pdfstartpage=1,
pdfstartview=FitH,
%------------- Doc Colours ------------------------------
colorlinks=true,
filecolor=webbrown, %defined below
citecolor=webgreen, %defined below
linkcolor=webblue, %defined below
urlcolor=webblue, %defined below
%------------- Doc Info ---------------------------------
pdftitle={THE DL-POLY-4 USER MANUAL},
pdfauthor={I.T. Todorov},
pdfsubject={Guide to the DL-POLY-4 Package},
pdfkeywords={General Purpose Molecular Dynamics (MD) Software;
Domain Decomposition (DD), Liked Cells (LC);
Symplectic Integration, Velocity Verlet (VV), Leapfrog Verlet (LFV),
Ensembles: NVE; NVEk - Evans; NVT - VV-Langevin Impuls \& LFV-Langevin,
Andersen, Berendsen, Nose-Hoover, NPT - Langevin, Berendesen,
Nose-Hoover, Martyna-Tuckerman-Klein, NsT (extended to constant
normal pressure \& surface area or surface tension) - Langevin,
Berendesen, Nose-Hoover, Martyna-Tuckerman-Klein;
User Defined Intra- \& Inter-molecular Force-field,
Frozen Particles, External Fields;
Constraint Bonds (CB) \& Potential of Mean Field (PMF) Dynamics
with SHAKE (LFV) \& RATTLE (VV),
DD Rigid Body (RB) Dynamics with Euler-Quaternions (LFV) \& No-Squish (VV);
Long-ranged Coulomb Interactions by Parallelised Smoothed
Particle Mesh Ewald (SPME) using Memory Distributed 3D FFT (DaFT),
Damped Force-shifted Coulomb, Ion Polarisation by Dinamical
(Adiabatic Shells) or Relaxed (Massless Shells) Shell Models;
Zero Temperature \& Conjugate Gradient Method (CGM) Minimisers;
Temperature rescaling, Boundary Pseudo Thermostats, Defect Detection Tool;
Optimised Parallel (ASCII \& netCDF Amber) I/O (includes MPI-I/O),
Verifiable \& User Extendable Source Code in Self-Contained \& Portable,
Modularised FORTRAN90, MPI Parallelisation \& Memory Distribution via DD},
]{hyperref}
%
%Define some eye-pleasing colors for this document
%
\definecolor{webgreen}{rgb}{0,0.5,0}
\definecolor{webbrown}{rgb}{0.6,0,0}
\definecolor{webblue}{rgb}{0,0,0.75}

% Layout settings
\setlength{\textheight}{236mm} \setlength{\textwidth}{165mm}
\setlength{\topmargin}{-7mm} \setlength{\headsep}{10mm}
\setlength{\footskip}{10mm} \setlength{\oddsidemargin}{-2mm}
\setlength{\evensidemargin}{-2mm}

%pagestyle
\usepackage{fancyhdr}
\pagestyle{fancy}
\lhead{\copyright STFC}
\chead{}
\rhead{Preface}
\lfoot{}
\cfoot{\thepage}
\rfoot{}

%paragraph settings
\parindent 0ex
\parskip 1ex

%commands
\usepackage{xspace}
\newcommand{\D}{{DL\_POLY\_4}\xspace}
\newcommand{\WEB}{\href{http://www.ccp5.ac.uk/DL\_POLY/}{http://www.ccp5.ac.uk/DL\_POLY/}\xspace}
\newcommand{\FTP}{\href{ftp://ftp.dl.ac.uk/ccp5/DL\_POLY/}{ftp://ftp.dl.ac.uk/ccp5/DL\_POLY/}\xspace}
\newcommand{\vek}[1]{\mbox{$\underline{#1}$}}
\newcommand{\mat}[1]{\mbox{$\underline{\underline{\bf #1}}$}}

\makeindex

\begin{document}
\pagenumbering{alph} \setcounter{page}{0}
\addcontentsline{toc}{chapter}{\underline{THE \D USER MANUAL}}

\begin{titlepage}
\title{THE \D USER MANUAL}
\author{I.T. Todorov \& W. Smith \\ \\
STFC Daresbury Laboratory \\
Daresbury, Warrington WA4 4AD \\
Cheshire, UK}
\date{Version 4.02.1,~~August 2011}
\end{titlepage}

\maketitle
\clearpage

\pagenumbering{roman} \setcounter{page}{1}

\maketitle \addcontentsline{toc}{section}{About \D}
\input about_dlpoly.tex
\maketitle \addcontentsline{toc}{section}{Disclaimer}
\input disclaimer.tex
\maketitle \addcontentsline{toc}{section}{Acknowledgements}
\input acknowledgements.tex
\maketitle \addcontentsline{toc}{section}{Manual Notation}
\input manual_notation.tex

%pagestyle
\rhead{Contents}

\addcontentsline{toc}{chapter}{\underline{Contents}}
\tableofcontents
\clearpage
\addcontentsline{toc}{chapter}{\underline{List of Tables}}
\listoftables
\clearpage
\addcontentsline{toc}{chapter}{\underline{List of Figures}}
\listoffigures
\clearpage

\pagenumbering{arabic} \setcounter{page}{1}

%pagestyle
\rhead{Section \thesection}

\chapter{Introduction}
\label{introduction}
\vskip 12ex
\section*{Scope of Chapter}
This chapter describes the concept, design and directory structure
of \D and how to obtain a copy of the source code.
\setcounter{equation}{0}
\newpage

\input introduction.tex

\chapter{Force Fields}
\label{force-field}
\vskip 12ex
\section*{Scope of Chapter}
This chapter describes the variety of interaction potentials
available in \D.
\setcounter{equation}{0}
\newpage

\input force_fields.tex

\chapter{Integration Algorithms}
\label{integration-algorithms}
\vskip 12ex
\section*{Scope of Chapter}
This chapter describes the integration algorithms coded into \D.
\setcounter{equation}{0}
\newpage

\input integration_algorithms.tex

\chapter{Construction and Execution}
\label{conex}
\vskip 12ex
\section*{Scope of Chapter}
This chapter describes how to compile a working version of \D and
run it.
\setcounter{equation}{0}
\newpage

\input construction.tex
\input compilation.tex
\input preparation.tex
\input ewald_precision.tex
\input diagnostic.tex

\chapter{Data Files}
\label{files}
\vskip 12ex
\section*{Scope of Chapter}
This chapter describes all the input and output files for \D,
examples of which are to be found in the {\em data} sub-directory.
\setcounter{equation}{0}
\newpage

\input input_files.tex
\input output_files.tex

\chapter{The \D Parallelisation and Source Code}
\label{source}
\vskip 12ex
\section*{Scope of Chapter}
This chapter we discuss the \D parallelisation strategy, describe the
principles used in the \D modularisation of the source code and list
the file structure found in the {\em source} subdirectory.
\setcounter{equation}{0}
\newpage

\input parallelisation.tex
\input code_structure.tex

\chapter{Examples}
\label{data}
\vskip 12ex
\section*{Scope of Chapter}
This chapter describes the standard test cases for \D, the input
and output files for which are in the {\em data} sub-directory.
\setcounter{equation}{0}
\newpage

\input examples.tex

\clearpage
\appendix
\addcontentsline{toc}{chapter}{\underline{Appendices}}
\rhead{Appendix \thechapter}

\chapter{\D Periodic Boundary Conditions}
\input boundary_conditions.tex

\chapter{\D Macros}
\input macros.tex

\chapter{\D Makefiles}
\input makefiles.tex

\chapter{\D Error Messages and User Action}
\input error_messages.tex

\chapter{\D README}
\input readme.tex

\clearpage
\addcontentsline{toc}{chapter}{Bibliography}
\bibliographystyle{dl_poly}
\rhead{Bibliography}
\bibliography{dl_poly}

\clearpage
\addcontentsline{toc}{chapter}{Index}
\rhead{Index}
\printindex
\end{document}
