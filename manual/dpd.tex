\label{DPD-all}
\section{Introduction}

Although in a Molecular Dynamics sense Dissipative Particle Dynamics (DPD) is
regarded as a type of thermostat, on its own it is an off-lattice, discrete
particle method for modelling mesoscopic systems in a fluid state.  It has little
in common with Lattice Gas Automata and Lattice Boltzmann dynamics methods
\cite{hoogerbrugge-92a}, except in its application to systems of similar length
and time scales, and last but not least that it captures hydrodynamics behaviour.

The DPD method inherits its methodology from Brownian/Langevin Dynamics (BD).
However, it differs from BD in an important way: it is \emph{Galilean invariant}
and for this reason conserves hydrodynamic behaviour, while the BD method does not
(its microscopic behaviour is only diffusive.  Many systems in their fluid state
are crucially dependent on hydrodynamic interactions and it is essential to
retain this feature in their models.  DPD is particularly useful for simulating
coarse-grained systems on the near-molecular scale, such as polymers, biopolymers,
lipids, emulsions and surfactants -- systems in which large scale structure evolves
on a time scale that is too long to be modelled effectively by traditional MD.  The
particles in DPD \cite{espanol-95a} are not regarded as molecules in a fluid but as
lumps of molecules grouped to form a \emph{fluid particle} in much the same spirit
as the renormalisation group has been applied in polymer physics where lumps of monomers
are grouped to form a \emph{bead}.  Hence, the beads are regarded as carriers of momentum.

It is worth noting that DPD may also be used when such systems experience shear and flow gradients.

\section{Outline of Method}

Following \cite{groot-97a} the DPD algorithm can be summarised by the following:

\begin{itemize}
\item A condensed phase system may be modelled as a system of `free' particles
interacting directly through \emph{soft} forces.  Note that \D allows for the
application of the DPD thermostat beyond systems of free particles only.  Thus
it will be valid on systems with any inratamolecular like interactions.
\item The system is coupled to a heat bath via stochastic forces, which act on
the particles in a pairwise manner.
\item The particles also experience a damping or drag force, which also acts
in a pairwise manner.
\item Thermodynamic equilibrium is maintained through the balance of the stochastic
and drag forces, i.e. the method satisfies the fluctuation-dissipation theorem.
\item At equilibrium (or steady state) the properties of the system are calculated
as averages over the individual particles, as in traditional Molecular Dynamics.
\end{itemize}

Therefore, the equation of motion are the same as these for the microcanonical
ensemble (NVE) but force, $f_{i}$, on particle $i$ is now a sum of pair forces:
\begin{equation}
\vek{f}_i = \sum_{j \neq i}^N \left( \vek{f}_{ij}^{C} + \vek{f}_{ij}^{D} + \vek{f}_{ij}^{R} \right)~~, \label{DPD}
\end{equation}
in which $\vek{f}_{ij}^{C}$, $\vek{f}_{ij}^{D}$ and $\vek{f}_{ij}^{R}$ are
the \emph{conservative}, \emph{drag} and \emph{random} (or \emph{stochastic})
pair forces respectively.  Each represents the force exerted on particle $i$
due to the presence of particle $j$.

The conservative interactions are usually \emph{soft} (i.e. weakly interacting)
so that the particles can pass by each other (or even through each other) relatively
easily so that equilibrium is achieved quickly.  A common form of interaction
potential is an inverse parabola (part of VDW types of potentials, see Section~\ref{vdw})
\begin{equation}
V(r_{ij}) = \left\{ \begin{array} {l@{\quad:\quad}l}
\frac{A_{ij}}{2}~r_{c}~\left(1-\frac{r_{ij}}{r_{c}}\right)^{2} & r_{ij} < r_{c} \\
0 & r_{ij} \ge r_{c} \end{array} \right.~~, \label{DPDU}
\end{equation}
where $r_{ij} = |\vek{r}_{j}-\vek{r}_{i}|$, $r_{c}$ is a cutoff radius and $A_{ij}$
is the interaction strength (that may be the same for all particle pairs or may be
different for different particle types).

Equation~(\ref{DPD}) gives rise to a repulsive force of the form:
\begin{equation}
\vek{f}_{ij}^{C} = A_{ij}~w^{C}(r_{ij}) \frac{\vek{r}_{ij}}{r_{ij}} =
A_{ij} \left( 1 - \frac{r_{ij}}{r_{c}} \right) \frac{\vek{r}_{ij}}{r_{ij}}~~. \label{DPDF}
\end{equation}

This is the deterministic or \emph{conservative} force $\vek{f}_{ij}^{C}$ exerted
on particle $i$ by particle $j$.  Note the switching function:
\begin{equation}
w^{C}(r_{ij}) = \left\{ \begin{array} {l@{\quad:\quad}l}
\left(1-\frac{r_{ij}}{r_{c}}\right) & r_{ij} < r_{c} \\
0 & r_{ij} \ge r_{c} \end{array} \right.~~, \label{DPDS}
\end{equation}
and the force are zero when $r_{ij} \ge r_{c}$ and thus the particles have an effective
diameter of $1$ in units of the cutoff radius $r_{c}$.  In the \D context all inter-
and intra-molecular forces will fall into this category of force!

The stochastic forces experienced by the particles is again pairwise in nature
and takes the form:
\begin{equation}
\vek{f}_{ij}^{R} = \sigma_{ij} w^{R}(r_{ij}) \zeta_{ij} \Delta t^{-\frac{1}{2}} \frac{\vek{r}_{ij}}{r_{ij}}~~,
\end{equation}
in which $\Delta t$ is the time step and $w^{R}(r_{ij})$ is a switching function
which imposes a finite limit on the range of the stochastic force.  $\zeta_{ij}$
is a random number with zero mean and unit variance.  The constant $\sigma_{ij}$
is related to the temperature, as is understood from the role of the stochastic
force in representing a heat bath.

Finally, the particles are subject to a drag force, which depends on the relative
velocity between interacting pairs of particles:
\begin{equation}
\vek{f}_{ij}^{D} = -\gamma_{ij} w^{D}(r_{ij})
\left(\vek{r}_{ij} \cdot \vek{v}_{ij}\right) \frac{\vek{r}_{ij}}{r_{ij}^2}~~,
\end{equation}
where $w^{D}(r_{ij})$ is once again a switching function and
$\vek{v}_{ij} = \vek{v}_{j}-\vek{v}_{i}$ is the inter-particle relative velocity.
The constant $\gamma_{ij}$ is the drag coefficient.  It follows from the
fluctuation-dissipation theorem that for thermodynamic equilibrium to
result from this method the following relations must hold:
\begin{eqnarray}
\sigma_{ij}^2 &=& 2~\gamma_{ij} k_{B} T \label{DPDC1} \\
w^{D}(r_{ij}) &=& \left[w^{R}(r_{ij})\right]^{2}~~. \label{DPDC2}
\end{eqnarray}

In practice, the switching functions are defined through:
\begin{equation}
w^{R}(r_{ij}) = \left[w^{C}(r_{ij})\right]^{2}~~,
\end{equation}
which ensures that all interactions are switched off at the range $r_{ij} = r_{c}$.

In many DPD simulations, the stochastic and drag coefficients are often constant
for all interactions, i.e. $\sigma_{ij} \equiv \sigma$ and $\gamma_{ij} \equiv \gamma$,
although this assumption does not have to apply.  In \D the $\gamma_{ij}$ coefficients
may be supplied at the end of each specified vdw interaction potential as a parameter
further to the last one for the particular vdw potential form.  For a DPD thermostat to
work correctly all possible two body interactions must be defined and all $\gamma_{ij} \ne 0$.
What \D will attempt first, if a two body interaction is missing, is to derive it using
mixing rules (default may be overridden by user specification).  However, if any of
$\gamma_{ij} = 0$ then \D will check for the existence of a global $\gamma$ that may
be optionally supplied by the user on the {\bf ensemble dpd} line and if it is non-zero
a global override will occur.  Otherwise, when the requirements for a DPD thermostat are
not satisfied, everything else will result in a controlled termination.

\section{Equation of state and dynamic properties}

The form of the conservative force determines the equation of state for a DPD fluid,
which can be derived using the virial theorem to express system pressure as follows:
\begin{eqnarray}
{\cal P} &=& \rho k_{B}T + \frac{1}{3V} \left\langle \sum_{j>i} (\vek{r}_i-\vek{r}_j) \cdot \vek{f}_{ij}^{C} \right\rangle \\
  &=& \rho k_{B} T + \frac{2 \pi}{3} \rho^{2} \int_{0}^{r_{c}} A \left( 1 - \frac{r}{r_{c}} \right) r^{3} g(r)~dr~~,
\end{eqnarray}
where $g(r)$ is a radial distribution function for the soft sphere model \cite{groot-97a}
and $\rho$ is the DPD particle density.  For sufficiently large densities ($\rho > 2$),
$g(r)$ takes the same form and the equation of state can be well-approximated by:
\begin{equation}
{\cal P} = \rho k_{B}T + \alpha A \rho^{2}~~,
\end{equation}
where the parameter $\alpha \approx 0.101 \pm 0.001$ has units equivalent to $r_{c}^{4}$.
This expression permits the use of fluid compressibilities to obtain conservative force
parameters for bulk fluids, e.g. for water $A \approx 75 k_{B} T/\rho$.  Alternative
equations of state may be obtained by modifying the functional form of conservative
interactions to include localized densities (i.e. many-body DPD) \cite{pagonabarraga-01a,trofimov-02a}.

Transport coefficients for a DPD fluid can be derived using the expressions for
the drag and stochastic forces\cite{groot-97a,koelman-93a,marsh-97a}.
The kinematic viscosity can be found to be
\begin{equation}
\nu \approx \frac{45 k_{B} T}{4 \pi \gamma \rho r_{c}^{3}} + \frac{2 \pi \gamma \rho r_{c}^{5}}{1575}~~,
\end{equation}
while the self-diffusion coefficient is given as
\begin{equation}
D \approx \frac{45 k_{B} T}{2 \pi \gamma \rho r_{c}^{3}}.
\end{equation}
The ratio of these two properties, the Schmidt number ($\textnormal{Sc} = \nu / D$), is therefore:
\begin{equation}
\textnormal{Sc} \approx \frac{1}{2} + \frac{(2 \pi \gamma \rho r_{c}^{4})^{2}}{70875 k_{B} T}
\end{equation}
and for values of the drag coefficient and density frequently used in DPD simulations,
this value is of the order of unity, which is an appropriate magnitude for gases but
three orders of magnitude too small for liquids.

This property of standard DPD does \emph{not} rule it out for simulations of liquid phases
except when hydrodynamics are important.  It may also be argued that the self-diffusion of
DPD particles might not correspond to that of individual molecules and thus a Schmidt number
of the order $10^{3}$ is unnecessary for modelling liquids \cite{peters-04a}.  Alternative
thermostats are available in the \M \cite{seaton-13a} - \WEM\index{WWW} package, which can
model systems with higher Schmidt numbers \cite{lowe-99a,stoyanov-05a}.

\section{Derivation of Equilibrium}

The derivation of the DPD algorithm is based on the Fokker-Planck equation
\begin{equation}
\frac{\partial \rho}{\partial t} = \mathcal{L} \rho \label{FokkerPlanck}
\end{equation}

where $\rho$ is the equilibrium distribution function and $\mathcal{L}$ is the evolution
operator, which may be split into \emph{conservative} and \emph{stochastic+dissipative} parts:
\begin{equation}
\mathcal{L} = \mathcal{L}^{C} + \mathcal{L}^{R+D}
\end{equation}
with
\begin{eqnarray}
\mathcal{L}^{C} &=& -\sum_{i=1}^{N} \frac{\vek{p}_{i}}{m_{i}}
\frac{\partial}{\partial \vek{r}_{i}} - \sum_{i \neq j}^{N}
\vek{f}_{ij}^{C} \frac{\partial}{\partial \vek{p}_{i}} \\
\mathcal{L}^{R+D} &=& \sum_{i=1}^{N} \hat{e}_{ij} \cdot \frac{\partial}{\partial \vek{p}_{i}}
\left[ \frac{\sigma^{2}}{2} \left\{w^{R} \left(r_{ij} \right) \right\}^{2} \hat{e}_{ij} \cdot
\left\{ \frac{\partial}{\partial \vek{p}_{i}} - \frac{\partial}{\partial \vek{p}_{j}} \right\} +
\gamma w^{D} \left( \hat{e}_{ij} \cdot \vek{v}_{ij} \right) \right]~~, \label {DPDEvolution}
\end{eqnarray}
where $\hat{e}_{ij} = \frac{\vek{r_{ij}}}{r_{ij}}$.

When $\sigma = \gamma = 0$ then equation~(\ref{FokkerPlanck}) becomes
\begin{equation}
\frac{\partial \rho}{\partial t} = \mathcal{L}^{C} \rho~~,
\end{equation}
for which the equilibrium solution is evidently
\begin{equation}
\rho^{eq} = \frac{1}{Z} \exp \left( \frac{1}{k_{B} T} \left[ \sum_{i=1}^{N}
\frac{p_{i}^{2}}{2 m_{i}} + \frac{1}{2} \sum_{j \neq i}^{N} \phi (r_{ij}) \right] \right)
\end{equation}

which is, of course, the Boltzmann distribution function for an equilibrium system.
Thus it is apparent that for the simulation based on equation~(\ref{FokkerPlanck})
to maintain the same distribution function, the terms in the operator $\mathcal{L}^{R+D}$
of equation~(\ref{DPDEvolution}) must sum to zero.  It follows that the
conditions given in equations~(\ref{DPDC1}) and (\ref{DPDC2}) must apply.

\section{Summary of Dissipative Particle Dynamics}

DPD is a simple method that can be viewed as a novel thermostatting method for molecular
dynamics.  All that is required is a system of spherical particles enclosed in a periodic
box undergoing time evolution as a result of the above forces.  It should be noted that
all computed interactions are pairwise, which means that the principle of the conservation
of momentum in the system, or \emph{Galilean invariance}, is preserved.  The conservation
of momentum is required for the preservation of hydrodynamic forces.  Therefore, the DPD
method is an NVT method that \emph{preserves hydrodynamics}.  The presence of hydrodynamics
is important in annealing defects in ordered mesophases \cite{gonella-97a}.  Thus DPD has
an intrinsic advantage over other methods such as traditional molecular dynamics, dynamic
density functional theory (which are purely \emph{diffusive}!) or Monte Carlo methods, in
trying to evolve a system towards an ordered thermodynamic equilibrium state.
