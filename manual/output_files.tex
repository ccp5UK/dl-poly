\section{The OUTPUT Files}
\label{output-files}

\D may produce many output files.  However only OUTPUT
(an incremental summary file of the simulation), STATIS
(a statistical history file), REVCON (a restart configuration
file - final) and REVIVE (a restart statistics accumulators
file - final) are mandatory.  DUMP\_E (a restart electronic 
temperature grid file - final) is also produced if the 
\index{Two-Temperature Model} two-temperature model (TTM)
is in use. The existence of the remaining files is 
optional upon user specifications in CONTROL.  Some of
these optional files are HISTORY, DEFECTS, MSDTMP, CFGMIN,
RDFDAT, USRDAT, ZDNDAT, VDFDAT\_*, LATS\_E, LATS\_I,
PEAK\_E, PEAK\_I.  These respectively contain: an
incremental dump file of all atomic coordinates, velocities
and forces; an incremental dump file of atomic coordinates
of defected particles (interstitials) and sites (vacancies);
an incremental dump file of of individual atomic mean square
displacement and temperature; a dump file of all atomic
coordinates of a minimised structure; a radial distribution
function (RDF) data file; the RDF data file for the umbrella sampling
(harmonic restraint); Z-density distribution data file;
velocity autocorrelation function (VAF) data files (one file for each species);
electronic temperature profile data file; ionic temperature profile data file;
electronic temperature statistical data file; 
ionic temperature statistical data file.

\subsection{The HISTORY File}
\label{history-file}

The HISTORY file is the dump file of atomic coordinates, velocities
and forces.  Its principal use is for off-line analysis.  The file
is written by the subroutine {\sc trajectory\_write}.  The control
variables for this file are {\tt ltraj, nstraj, istraj} and {\tt
keytrj} which are created internally, based on information read from
the {\bf traj} directive in the CONTROL file (see Section~\ref{control-file}).
The HISTORY file will be created only if the directive {\bf traj}
appears in the CONTROL file.

The HISTORY file can become {\em very} large, especially if it is
formatted.  For serious simulation work it is recommended that the
file be written to a scratch disk capable of accommodating a large
data file.  Alternatively, the file may be written in netCDF format
instead of in ASCII (users must change ensure this functionality is
available), which has the additional advantage of speed.

The HISTORY has the following structure:
\begin{tabbing}
X\=XXXXXXXX\=XXXXXXXXXXXXXXXX\=\kill
{\bf record 1} \\
\> {\tt header}  \> a72     \> file header \\
{\bf record 2} \\
\> {\tt keytrj}  \> integer \> trajectory key (see Table~\ref{keytrj}) in last frame \\
\> {\tt imcon}   \> integer \> periodic boundary key (see Table~\ref{imcon}) in last frame \\
\> {\tt megatm}  \> integer \> number of atoms in simulation cell in last frame \\
\> {\tt frame}   \> integer \> number configuration frames in file \\
\> {\tt records} \> integer \> number of records in file
\end{tabbing}

For timesteps greater than {\tt nstraj} the HISTORY file is
appended at intervals specified by the {\bf traj} directive in the
CONTROL file, with the following information for each
configuration:
\begin{tabbing}
X\=XXXXXXXX\=XXXXXXXXXXXXXXXX\=\kill
{\bf record i} \\
\> {\tt timestep} \> a8      \> the character string ``timestep'' \\
\> {\tt nstep}    \> integer \> the current time-step \\
\> {\tt megatm}   \> integer \> number of atoms in simulation cell (again) \\
\> {\tt keytrj}   \> integer \> trajectory key (again) \\
\> {\tt imcon}    \> integer \> periodic boundary key (again) \\
\> {\tt tstep}    \> real    \> integration timestep (ps) \\
\> {\tt time}     \> real    \> elapsed simulation time (ps) \\
{\bf record ii} \\
\> {\tt cell(1)}  \> real    \> x component of $a$ cell vector in \AA \\
\> {\tt cell(2)}  \> real    \> y component of $a$ cell vector in \AA \\
\> {\tt cell(3)}  \> real    \> z component of $a$ cell vector in \AA \\
{\bf record iii} \\
\> {\tt cell(4)}  \> real    \> x component of $b$ cell vector in \AA \\
\> {\tt cell(5)}  \> real    \> y component of $b$ cell vector in \AA \\
\> {\tt cell(6)}  \> real    \> z component of $b$ cell vector in \AA \\
{\bf record iv} \\
\> {\tt cell(7)}  \> real    \> x component of $c$ cell vector in \AA \\
\> {\tt cell(8)}  \> real    \> y component of $c$ cell vector in \AA \\
\> {\tt cell(9)}  \> real    \> z component of $c$ cell vector in \AA
\end{tabbing}
This is followed by the configuration for the current timestep. i.e.
for each atom in the system the following data are included:
\begin{tabbing}
X\=XXXXXXXX\=XXXXXXXXXXXXXXXX\=\kill
{\bf record a} \\
\> {\tt atmnam} \> a8      \> atomic label \\
\> {\tt iatm}   \> integer \> atom index \\
\> {\tt weight} \> real    \> atomic mass (a.m.u.) \\
\> {\tt charge} \> real    \> atomic charge (e) \\
\> {\tt rsd}    \> real    \> displacement from position at $t=0$ in \AA \\
{\bf record b} \\
\> {\tt xxx}    \> real    \> x coordinate \\
\> {\tt yyy}    \> real    \> y coordinate \\
\> {\tt zzz}    \> real    \> z coordinate \\
{\bf record c} only for {\tt keytrj} $>$ 0 \\
\> {\tt vxx}    \> real    \> x component of velocity in \AA/picosecond \\
\> {\tt vyy}    \> real    \> y component of velocity in \AA/picosecond \\
\> {\tt vzz}    \> real    \> z component of velocity in \AA/picosecond \\
{\bf record d} only for {\tt keytrj} $>$ 1 \\
\> {\tt fxx}    \> real    \> x component of force in \AA$\cdot$Dalton/picosecond$^{2}$ \\
\> {\tt fyy}    \> real    \> y component of force in \AA$\cdot$Dalton/picosecond$^{2}$ \\
\> {\tt fzz}    \> real    \> z component of force in \AA$\cdot$Dalton/picosecond$^{2}$
\end{tabbing}
Thus the data for each atom is a minimum of two records and a maximum of 4.

\subsection{The MSDTMP File}
\label{msdtmp-file}

The MSDTMP file is the dump file of individual atomic mean square
displacements (square roots in \AA) and mean square temperature
(square roots in Kelvin).  Its principal use is for off-line analysis.
The file is written by the subroutine {\sc msd\_write}.  The control
variables for this file are {\tt l\_msd, nstmsd, istmsd} which are
created internally, based on information read from the {\bf msdtmp}
directive in the CONTROL file (see Section~\ref{control-file}).
The MSDTMP file will be created only if the directive {\bf msdtmp}
appears in the CONTROL file.

The MSDTMP file can become {\em very} large, especially if it is
formatted.  For serious simulation work it is recommended that the
file be written to a scratch disk capable of accommodating a large
data file.

The MSDTMP has the following structure:
\begin{tabbing}
X\=XXXXXXXX\=XXXXXXXXXXXXXXXX\=\kill
{\bf record 1} \\
\> {\tt header}  \> a52     \> file header \\
{\bf record 2} \\
\> {\tt megatm}  \> integer \> number of atoms in simulation cell in last frame \\
\> {\tt frame}   \> integer \> number configuration frames in file \\
\> {\tt records} \> integer \> number of records in file
\end{tabbing}

For timesteps greater than {\tt nstmsd} the MSDTMP file is
appended at intervals specified by the {\bf msdtmp} directive in the
CONTROL file, with the following information for each
configuration:
\begin{tabbing}
X\=XXXXXXXX\=XXXXXXXXXXXXXXXX\=\kill
{\bf record i} \\
\> {\tt timestep} \> a8      \> the character string ``timestep'' \\
\> {\tt nstep}    \> integer \> the current time-step \\
\> {\tt megatm}   \> integer \> number of atoms in simulation cell (again) \\
\> {\tt tstep}    \> real    \> integration timestep (ps) \\
\> {\tt time}     \> real    \> elapsed simulation time (ps)
\end{tabbing}
This is followed by the configuration for the current timestep. i.e.
for each atom in the system the following data are included:
\begin{tabbing}
X\=XXXXXXXX\=XXXXXXXXXXXXXXXX\=\kill
{\bf record a} \\
\> {\tt atmnam}           \> a8      \> atomic label \\
\> {\tt iatm}             \> integer \> atom index \\
\> $\sqrt{{\tt MSD}(t)}$  \> real    \> square root of the atomic mean square displacements (in \AA) \\
\> T$_{mean}$             \> real    \> atomic mean temperature (in Kelvin)
\end{tabbing}

\subsection{The DEFECTS File}
\label{defects-file}

The DEFECTS file is the dump file of atomic coordinates of defects
(see Section~\ref{reference-file}).  Its principal use is for
off-line analysis.  The file is written by the subroutine
{\sc defects\_write}.  The control
variables for this file are {\tt ldef, nsdef, isdef} and {\tt rdef}
which are created internally, based on information read from
the {\bf defects} directive in the CONTROL file (see Section~\ref{control-file}).
The DEFECTS file will be created only if the directive {\bf defects}
appears in the CONTROL file.

The DEFECTS file may become {\em very} large, especially if it is
formatted.  For serious simulation work it is recommended that the
file be written to a scratch disk capable of accommodating a large
data file.

The DEFECTS has the following structure:
\begin{tabbing}
X\=XXXXXXXX\=XXXXXXXXXXXXXXXX\=\kill
{\bf record 1} \\
\> {\tt header}  \> a72     \> file header \\
{\bf record 2} \\
\> {\tt rdef}    \> real    \> site-interstitial cutoff (\AA) in last frame \\
\> {\tt frame}   \> integer \> number configuration frames in file \\
\> {\tt records} \> integer \> number of records in file
\end{tabbing}

For timesteps greater than {\tt nsdef} the DEFECTS file is
appended at intervals specified by the {\bf defects} directive in the
CONTROL file, with the following information for each
configuration:
\begin{tabbing}
X\=XXXXXXXXXXXX\=XXXXXXXXXXXXXXXX\=\kill
{\bf record i} \\
\> {\tt timestep}      \> a8      \> the character string ``timestep'' \\
\> {\tt nstep}         \> integer \> the current time-step \\
\> {\tt tstep}         \> real    \> integration timestep (ps) \\
\> {\tt time}          \> real    \> elapsed simulation time (ps) \\
\> {\tt imcon}         \> integer \> periodic boundary key (see Table~\ref{imcon}) \\
\> {\tt rdef}          \> real    \> site-interstitial cutoff (\AA) \\
{\bf record ii} \\
\> {\tt defects}       \> a7      \> the character string ``defects'' \\
\> {\tt ndefs}         \> integer \> the total number of defects \\
\> {\tt interstitials} \> a13     \> the character string ``interstitials'' \\
\> {\tt ni}            \> integer \> the total number of interstitials \\
\> {\tt vacancies}     \> a9      \> the character string ``vacancies'' \\
\> {\tt nv}            \> integer \> the total number of vacancies \\
{\bf record iii} \\
\> {\tt cell(1)}       \> real    \> x component of $a$ cell vector \\
\> {\tt cell(2)}       \> real    \> y component of $a$ cell vector \\
\> {\tt cell(3)}       \> real    \> z component of $a$ cell vector \\
{\bf record iv} \\
\> {\tt cell(4)}       \> real    \> x component of $b$ cell vector \\
\> {\tt cell(5)}       \> real    \> y component of $b$ cell vector \\
\> {\tt cell(6)}       \> real    \> z component of $b$ cell vector \\
{\bf record v} \\
\> {\tt cell(7)}       \> real    \> x component of $c$ cell vector \\
\> {\tt cell(8)}       \> real    \> y component of $c$ cell vector \\
\> {\tt cell(9)}       \> real    \> z component of $c$ cell vector
\end{tabbing}
This is followed by the {\tt ni} interstitials for the current
timestep, as each interstitial has the following data lines:
\begin{tabbing}
X\=XXXXXXXX\=XXXXXXXXXXXXXXXX\=\kill
{\bf record a}  \\
\> {\tt atmnam} \> a10     \> i\_atomic label from CONFIG \\
\> {\tt iatm}   \> integer \> atom index from CONFIG \\
{\bf record b} \\
\> {\tt xxx}    \> real    \> x coordinate \\
\> {\tt yyy}    \> real    \> y coordinate \\
\> {\tt zzz}    \> real    \> z coordinate
\end{tabbing}
This is followed by the {\tt nv} vacancies for the current
timestep, as each vacancy has the following data lines:
\begin{tabbing}
X\=XXXXXXXX\=XXXXXXXXXXXXXXXX\=\kill
{\bf record a}  \\
\> {\tt atmnam} \> a10     \> v\_atomic label from REFERENCE \\
\> {\tt iatm}   \> integer \> atom index from REFERENCE \\
{\bf record b} \\
\> {\tt xxx}    \> real    \> x coordinate from REFERENCE \\
\> {\tt yyy}    \> real    \> y coordinate from REFERENCE \\
\> {\tt zzz}    \> real    \> z coordinate from REFERENCE
\end{tabbing}

\subsection{The RSDDAT File}
\label{rsddat-file}

The RSDDAT file is the dump file of atomic coordinates of atoms
that are displaced from their original position at $t~=~0$ farther
than a preset cutoff.  Its principal use is for off-line analysis.
The file is written by the subroutine {\sc rsd\_write}.  The control
variables for this file are {\tt lrsd, nsrsd, isrsd} and {\tt rrsd}
which are created internally, based on information read from
the {\bf displacements} directive in the CONTROL file (see
Section~\ref{control-file}).  The RSDDAT file will be created only
if the directive {\bf defects} appears in the CONTROL file.

The RSDDAT file may become {\em very} large, especially if it is
formatted.  For serious simulation work it is recommended that the
file be written to a scratch disk capable of accommodating a large
data file.

The RSDDAT has the following structure:
\begin{tabbing}
X\=XXXXXXXX\=XXXXXXXXXXXXXXXX\=\kill
{\bf record 1} \\
\> {\tt header}  \> a72     \> file header \\
{\bf record 2} \\
\> {\tt rdef}    \> real    \> displacement qualifying cutoff (\AA) in last frame \\
\> {\tt frame}   \> integer \> number configuration frames in file \\
\> {\tt records} \> integer \> number of records in file
\end{tabbing}

For timesteps greater than {\tt nsrsd} the RSDDAT file is
appended at intervals specified by the {\bf displacements} directive
in the CONTROL file, with the following information for each
configuration:
\begin{tabbing}
X\=XXXXXXXXXXXX\=XXXXXXXXXXXXXXXX\=\kill
{\bf record i} \\
\> {\tt timestep}      \> a8      \> the character string ``timestep'' \\
\> {\tt nstep}         \> integer \> the current time-step \\
\> {\tt tstep}         \> real    \> integration timestep (ps) \\
\> {\tt time}          \> real    \> elapsed simulation time (ps) \\
\> {\tt imcon}         \> integer \> periodic boundary key (see Table~\ref{imcon}) \\
\> {\tt rrsd}          \> real    \> displacement qualifying cutoff (\AA) \\
{\bf record ii} \\
\> {\tt displacements} \> a13     \> the character string ``displacements'' \\
\> {\tt nrsd}          \> integer \> the total number of displacements \\
{\bf record iii} \\
\> {\tt cell(1)}       \> real    \> x component of $a$ cell vector \\
\> {\tt cell(2)}       \> real    \> y component of $a$ cell vector \\
\> {\tt cell(3)}       \> real    \> z component of $a$ cell vector \\
{\bf record iv} \\
\> {\tt cell(4)}       \> real    \> x component of $b$ cell vector \\
\> {\tt cell(5)}       \> real    \> y component of $b$ cell vector \\
\> {\tt cell(6)}       \> real    \> z component of $b$ cell vector \\
{\bf record v} \\
\> {\tt cell(7)}       \> real    \> x component of $c$ cell vector \\
\> {\tt cell(8)}       \> real    \> y component of $c$ cell vector \\
\> {\tt cell(9)}       \> real    \> z component of $c$ cell vector
\end{tabbing}
This is followed by the {\tt nrsd} displacements for the current
timestep, as each atom has the following data lines:
\begin{tabbing}
X\=XXXXXXXX\=XXXXXXXXXXXXXXXX\=\kill
{\bf record a}  \\
\> {\tt atmnam} \> a10     \> atomic label from CONFIG \\
\> {\tt iatm}   \> integer \> atom index from CONFIG \\
\> {\tt ratm}   \> real    \> atom displacement from its position at $t~=~0$ \\
{\bf record b} \\
\> {\tt xxx}    \> real    \> x coordinate \\
\> {\tt yyy}    \> real    \> y coordinate \\
\> {\tt zzz}    \> real    \> z coordinate
\end{tabbing}

\subsection{The CFGMIN File}
\label{cfgminfile}

The CFGMIN file only appears if the user has selected the programmed
minimisation option (directive {\bf minim}ise (or {\bf optim}ise)
in the CONTROL file).  Its contents have the same format as the
CONFIG file (see Section~\ref{config-file}), but contains only atomic
position data and will never contain either velocity or force data
(i.e. parameter {\tt levcfg} is always zero).  In addition, three
extra numbers appear on the end of the second line of the file:
\begin{enumerate}
\item an integer indicating the number of minimisation cycles
required to obtain the structure,
\item the configuration energy of the minimised configuration expressed
in \D units (Section~\ref{units}), and
\item the configuration energy of the initial structure expressed
in \D units (Section~\ref{units}).
\end{enumerate}

\subsection{The OUTPUT File}
\label{output-file}

The job output consists of 7 sections: Header; Simulation control
specifications; Force field specification; System specification;
Summary of the initial configuration; Simulation progress; Sample of
the final configuration; Summary of statistical data; and Radial
distribution functions and Z-density profile.  These sections are
written by different subroutines at various stages of a job.
Creation of the OUTPUT file {\em always} results from running \D. It
is meant to be a human readable file, destined for hardcopy output.

\subsubsection{Header}

Gives the \D version number, the number of processors in use, the
link-cell algorithm in use and a title for the job as given in the
header line of the input file CONTROL.  This part of the file is
written from the subroutines {\sc dl\_poly}, {\sc set\_bounds}
and {\sc read\_control}.

\subsubsection{Simulation Control Specifications}

Echoes the input from the CONTROL file.  Some variables may be
reset if illegal values were specified in the CONTROL file.  This
part of the file is written from the subroutine {\sc
read\_control}.

\subsubsection{Force Field Specification}

Echoes the FIELD file.  A warning line will be printed if the
system is not electrically neutral.  This warning will appear
immediately before the non-bonded short-range potential
specifications.  This part of the file is written from the
subroutine {\sc read\_field}.

\subsubsection{System Specification}

Echoes system name, periodic boundary specification, the cell
vectors and volume, some initial estimates of long-ranged
corrections the energy and pressure (if appropriate), some
concise information on topology and degrees of freedom
break-down list.  This part of the file is written from the
subroutines {\sc scan\_config}, {\sc check\_config},
{\sc system\_init}, {\sc report\_topology} and
{\sc set\_temperature}.

\subsubsection{Summary of the Initial Configuration}

This part of the file is written from the main subroutine {\sc
dl\_poly\_}.  It states the initial configuration of (a maximum
of) 20 atoms in the system.  The configuration information given
is based on the value of {\tt levcfg} in the CONFIG file.  If {\tt
levcfg} is 0 (or 1) positions (and velocities) of the 20 atoms are
listed.  If {\tt levcfg} is 2 forces are also written out.

\subsubsection{Simulation Progress}

This part of the file is written by the \D root segment
{\sc dl\_poly}.  The header line is printed at the top of each page as:

\begin{lstlisting}[basicstyle=\small]
--------------------------------------------------------------------------------------------------

    step   eng_tot  temp_tot   eng_cfg   eng_src   eng_cou   eng_bnd   eng_ang   eng_dih   eng_tet
time(ps)    eng_pv  temp_rot   vir_cfg   vir_src   vir_cou   vir_bnd   vir_ang   vir_con   vir_tet
cpu  (s)    volume  temp_shl   eng_shl   vir_shl     alpha      beta     gamma   vir_pmf     press

--------------------------------------------------------------------------------------------------
\end{lstlisting}

\noindent The labels refer to :
\begin{tabbing}
X\=XXXXXXXX\=XXXXXXXXXXXXXXXX\=\kill
{\bf line 1} \\
\> {\tt step}      \> MD step number \\
\> {\tt eng\_tot}  \> total internal energy of the system \\
\> {\tt temp\_tot} \> system temperature (in Kelvin) \\
\> {\tt eng\_cfg}  \> configurational energy of the system \\
\> {\tt eng\_src}  \> configurational energy due to short-range potential contributions \\
\> {\tt eng\_cou}  \> configurational energy due to electrostatic potential\index{potential!electrostatics} \\
\> {\tt eng\_bnd}  \> configurational energy due to chemical bond\index{potential!bond} potentials \\
\> {\tt eng\_ang}  \> configurational energy due to valence angle\index{potential!valence angle} and three-body\index{potential!three-body} potentials \\
\> {\tt eng\_dih}  \> configurational energy due to dihedral\index{potential!dihedral} inversion and four-body\index{potential!four-body} potentials \\
\> {\tt eng\_tet}  \> configurational energy due to tethering potentials\index{potential!tether} \\
{\bf line 2} \\
\> {\tt time(ps)}  \> elapsed simulation time (in pico-seconds) since the beginning of the job \\
\> {\tt eng\_pv}   \> enthalpy of system \\
\> {\tt temp\_rot} \> rotational temperature (in Kelvin) \\
\> {\tt vir\_cfg}  \> total configurational contribution to the virial \\
\> {\tt vir\_src}  \> short range potential contribution to the virial \\
\> {\tt vir\_cou}  \> electrostatic potential\index{potential!electrostatics} contribution to the virial \\
\> {\tt vir\_bnd}  \> chemical bond contribution to the virial \\
\> {\tt vir\_ang}  \> angular and three-body\index{potential!three-body} potentials contribution to the virial \\
\> {\tt vir\_con}  \> constraint bond\index{constraints!bond} contribution to the virial \\
\> {\tt vir\_tet}  \> tethering potential\index{potential!tether} contribution to the virial \\
{\bf line 3} \\
\> {\tt cpu (s)}   \> elapsed cpu time (in seconds) since the beginning of the job \\
\> {\tt volume}    \> system volume (in \AA$^{3}$) \\
\> {\tt temp\_shl} \> core-shell temperature (in Kelvin) \\
\> {\tt eng\_shl}  \> configurational energy due to core-shell potentials \\
\> {\tt vir\_shl}  \> core-shell potential contribution to the virial \\
\> {\tt alpha }    \> angle between $b$ and $c$ cell vectors (in degrees) \\
\> {\tt beta }     \> angle between $c$ and $a$ cell vectors (in degrees) \\
\> {\tt gamma }    \> angle between $a$ and $b$ cell vectors (in degrees) \\
\> {\tt vir\_pmf}  \> PMF constraint\index{constraints!PMF} contribution to the virial \\
\> {\tt press}     \> pressure (in kilo-atmospheres)
\end{tabbing}

\noindent {\bf Note:} The total internal energy of the system
(variable {\tt tot\_energy}) includes all contributions to the
energy (including system extensions due to thermostats etc.).  It
is nominally the {\em conserved variable} of the system, and is
not to be confused with conventional system energy, which is a sum
of the kinetic and configuration energies.

The interval for printing out these data is determined by the
directive {\bf print} in the CONTROL file.  At each time-step that
printout is requested the instantaneous values of the above
statistical variables are given in the appropriate columns.
Immediately below these three lines of output the rolling averages
of the same variables are also given.  The maximum number of
time-steps used to calculate the rolling averages is controlled by
the directive {\bf stack} in file CONTROL (see above) and listed
as parameter {\tt mxstak} in the {\sc setup\_module} file (see
Section~\ref{file-structure}).  The default value is {\tt mxstak}
$=~100$.

\subsubsection*{Energy Units}

The energy unit for the energy and virial data appearing in the
OUTPUT is defined by the {\bf units} directive appearing in the
FIELD file.  System energies are therefore read in {\bf units}
per MD cell.

\subsubsection*{Pressure Units}

The unit of pressure\index{units!pressure} is katms, irrespective
of what energy unit is chosen.

\subsubsection*{Two-Temperature Model}

If the \index{Two-Temperature Model}two-temperature model is
in use, information about the timestep sizes used for electronic 
thermal diffusivity is written immediately prior to each report of 
statistical variables at each molecular dynamics timestep for which
printout is requested. The optimum diffusive timestep size is given 
in pico-seconds, along with the chosen value and the corresponding 
number of divisions of the MD timestep. If dynamic calculation of
the average atomic density in active cells is requested, this value is
included along with the number of active ionic temperature cells.
Reports are also given when energy deposition starts and finishes.

\subsubsection{Sample of Final Configuration}

The positions, velocities and forces of the 20 atoms used for the
sample of the initial configuration (see above) are given.  This
is written by the main subroutine {\sc dl\_poly}.

\subsubsection{Summary of Statistical Data}

This portion of the OUTPUT file is written from the subroutine
{\sc statistics\_result}.  The number of time-steps used in the
collection of statistics is given.  Then the averages over the
production portion of the run are given for the variables
described in the previous section.  The root mean square variation
in these variables follow on the next two lines.  The
energy\index{units!DL\_POLY} and pressure
units\index{units!pressure} are as for the preceding section.

Also provided in this section are estimates of the diffusion
coefficient and the mean square displacement for the different
atomic species in the simulation.  These are determined from a {\em
single time origin} and are therefore approximate.  Accurate
determinations of the diffusion coefficients can be obtained using
the {\sc msd} utility program, which processes the HISTORY file
(see \C User Manual).

If an NPT (N$\mat{\sigma}$T) simulation is performed the OUTPUT file
also provides the mean pressure (and stress tensor in pressure units
as density) and mean simulation cell vectors.  In case when extended
N$\mat{\sigma}$T ensembles are used then further mean $(x,y)$
plain area and mean surface tension are also displayed in the OUTPUT file.

\subsubsection{Radial Distribution Functions}

If both calculation and printing of radial distribution functions
have been requested (by selecting directives {\bf rdf} and {\bf
print rdf} in the CONTROL file) radial distribution functions are
printed out.  This is written from the subroutine {\sc
rdf\_compute}.  First the number of time-steps used for the
collection of the histograms is stated.  Then each pre-requested
function is given in turn.  For each function a header line states
the atom types (`a' and `b') represented by the function.  Then
$r,~g(r)$ and $n(r)$ are given in tabular form.  Output is given
from 2 entries before the first non-zero entry in the $g(r)$
histogram.  $n(r)$ is the average number of atoms of type `b' within
a sphere of radius $r$ around an atom of type `a'.
Note that a readable version of these data is provided by the
RDFDAT file (below).

\subsubsection{Umbrella Sampling Restraint RDF}

If an umbrella sampling harmonic restraint is defined in the FIELD file
(by selecting the {\bf ushr} external field sectione) the RDF of the
two restraint objects/fragments is printed out.  This is written from
the subroutine {\sc usr\_compute} in {\sc rdf\_compute}.
Note that a readable version of these data is provided by the
USRDAT file (below).

\subsubsection{Z-density Profile}

If both calculation and printing of Z-density profiles have been
requested (by selecting directives {\bf zden} and {\bf print zden}
in the CONTROL file Z-density profiles are printed out as the last
part of the OUTPUT file.  This is written by the subroutine {\sc
z\_density\_compute}.  First the number of time-steps used for the
collection of the histograms is stated.  Then each function is given
in turn.  For each function a header line states the atom type
represented by the function.  Then $z,~\rho(z)$ and $n(z)$ are given
in tabular form. Output is given from $Z = [-L/2,L/2]$ where L is
the length of the MD cell in the Z direction and $\rho(z)$ is the
mean number density.  $n(z)$ is the running integral from $-L/2$ to
$z$ of $({\rm xy~cell~area}) \times \rho(s)~ds$.
Note that a readable version of these data is provided by the
ZDNDAT file (below).

\subsubsection{Velocity Autocorrelation Functions}

If both calculation and printing of velocity autocorrelation functions
have been requested (by selecting directives {\bf vaf} and {\bf print vaf}
in the CONTROL file the velocity autocorrelation function for the system
(either time-averaged or the last complete sample) is printed out as the
last part of the OUTPUT file.  This is written by the subroutine
{\sc vaf\_compute}.  First the details of the calculations are stated:
either the number of samples used to give a time-averaged profile
or the number of the last completed sample with its starting time.
The absolute value of the velocity autocorrelation function for the system
at $t=0$, $C(0)$, is then stated. Then $t$ and $Z(t)$ are given in tabular form.
$Z(t)=C(t)/C(0)$ is the value of the velocity autocorrelation function,
$C(t)=\langle \vek{v}_{i}(0) \cdot  \vek{v}_{i}(t) \rangle$,
scaled by $C(0) \equiv 3k_B T/m$.  Note that a readable version of these data
for individual species is provided by the VAFDAT files (below).

\subsection{The REVCON File}
\label{revcon-file}

This file is formatted and written by the subroutine {\sc revive}.
REVCON is the restart configuration file.  The file is written every
{\tt ndump} time steps in case of a system crash during execution
and at the termination of the job.  A successful run of \D will
always produce a REVCON file, but a failed job may not produce the
file if an insufficient number of timesteps have elapsed.  {\tt
ndump} is controlled by the directive {\bf dump} in file CONTROL
(see above) and listed as parameter {\tt ndump} in the {\sc
setup\_module} file (see Section~\ref{file-structure}). The default
value is {\tt ndump} $=1000$.  REVCON is identical in format to the
CONFIG input file (see Section~\ref{config-file}).  REVCON should be
renamed CONFIG to continue a simulation from one job to the next.
This is done for you by the {\sl copy} macro supplied in the {\em
execute} directory of \D.

\subsection{The REVIVE File}
\label{revive-file}

This file is unformatted and written by the subroutine {\sc
system\_revive}.  It contains the accumulated statistical data. It
is updated whenever the file REVCON is updated (see previous
section).  REVIVE should be renamed REVOLD to continue a
simulation from one job to the next.  This is done by the {\sl
copy} macro supplied in the {\em execute} directory of \D.  In
addition, to continue a simulation from a previous job the {\bf
restart} keyword must be included in the CONTROL file.

The format of the REVIVE file is identical to the REVOLD file
described in Section~\ref{revold-file}.

\subsection{The DUMP\_E File}

This file is formatted and written by the subroutine
{\sc ttm\_system\_revive} every {\tt ndump} time steps.  It 
contains the electronic temperatures of all coarse-grained 
electronic temperature (CET) cells and can be used to 
restart a simulation using the \index{Two-Temperature Model}
two-temperature model without renaming the file.

The format of the DUMP\_E is described in 
Section~\ref{dumpe-file}.
 
\subsection{The RDFDAT File}
\label{rdf-file}

This is a formatted file containing {\em Radial Distribution
Function} (RDF) data.  Its contents are as follows:
\begin{tabbing}
X\=XXXXXXXX\=XXXXXXXXXXXX\=XXXXXXXXXX\=\kill
{\bf record 1} \\
\> {\tt cfgname} \> a72     \> configuration name \\
{\bf record 2} \\
\> {\tt ntprdf}  \> integer \> number of different RDF pairs tabulated in file \\
\> {\tt mxgrdf}  \> integer \> number of grid points for each RDF pair
\end{tabbing}
There follow the data for each individual RDF, i.e. {\tt ntprdf}
times.  The data supplied are as follows:
\begin{tabbing}
X\=XXXXXXXX\=XXXXXXXXXXXX\=XXXXXXXXXX\=\kill
{\bf first record} \\
\> {\tt atname 1} \> a8   \> first atom name \\
\> {\tt atname 2} \> a8   \> second atom name \\
{\bf following records} ({\em mxgrdf} records) \\
\> {\tt radius}   \> real \> interatomic distance (\AA) \\
\> {\tt g(r)}     \> real \> RDF at given radius
\end{tabbing}

{\bf Note 1.} The RDFDAT file is optional and appears when the
{\bf print rdf} option is specified in the CONTROL file.

{\bf Note 2.} Along with the RDFDAT file, two other files will
be created whenever the {\bf print ana}lysis directive is invoked:
VDWPMF~\&~VDWTAB, both containing the data for potentials of mean
force and the corresponding virials calculated based on the obtained
RDF:s, i.e. PMF $\sim -\ln({\rm RDF})$ (in the energy units specified
in the FIELD file).  These files have a simple three column format,
the same as that used for *PMF files in the case of bonded units,
see Section~\ref{IPDF-analysis}.  The purpose of these files is to
provide the user with means of setting up a PMF-based force-field,
for example in the case of initial coarse-graining of
an atomistic system.  In particular, one can convert the VDWTAB file
into a correctly formatted TABLE file (Section~\ref{table-file})
by using the utility called {\tt pmf2tab.f} (subject to compilation;
found in \D directory {\tt utility}) as follows,

{\tt [user@host]\$ pmf2tab.exe < VDWTAB}

%\begin{lstlisting}
%[user@host]$ pmf2tab.exe < VDWTAB
%\end{lstlisting}

see Section~\ref{cg-intro} for completeness.

\subsection{The USRDAT File}
\label{usr-file}

\begin{tabbing}
X\=XXXXXXXX\=XXXXXXXX\=\kill
{\bf record 1} \\
\> {\tt \# title}  \> a100    \> file header title \\
{\bf record 2} \\
\> {\tt \# header} \> a100    \> file information header \\
{\bf record 3} \\
\> {\tt \# info}   \> a30     \> information to follow string \\
{\bf record 3} \\
\> {\tt bins}      \> integer \> number of bins \\
\> {\tt cutoff}    \> real    \> cutoff in \AA \\
\> {\tt frames}    \> integer \> number of sampled configurations \\
\> {\tt volume}    \> real    \> average cell volue \AA$^{3}$ \\
{\bf record 4} \\
\> {\tt \#}        \> a1      \> a hash (\#) symbol \\
{\bf following records} ({\em mxgusr} records) \\
\> {\tt radius}    \> real    \> interatomic distance (\AA) \\
\> {\tt g(r)}      \> real    \> RDF at given radius
\end{tabbing}

\subsection{The ZDNDAT File}
\label{zdn-file}

This is a formatted file containing the Z-density data.  Its
contents are as follows:
\begin{tabbing}
X\=XXXXXXXX\=XXXXXXXXXXXX\=XXXXXXXXXX\=\kill
{\bf record 1} \\
\> {\tt cfgname} \> a72    \> configuration name \\
{\bf record 2} \\
\> {\tt ntpatm}  \> integer \> number of unique atom types profiled in file \\
\> {\tt mxgrdf}  \> integer \> number of grid points in the Z-density function\
\end{tabbing}
There follow the data for each individual Z-density function, i.e.
{\tt ntpatm} times. The data supplied are as follows:
\begin{tabbing}
X\=XXXXXXXX\=XXXXXXXXXXXX\=XXXXXXXXXX\=\kill
{\bf first record} \\
\> {\tt atname}  \> a8      \> unique atom name \\
{\bf following records} ({\em mxgrdf} records) \\
\> {\tt z}       \> real    \> distance in z direction (\AA) \\
\> $\rho(z)$     \> real    \> Z-density at given height {\tt z}
\end{tabbing}

{\bf Note} the ZDNDAT file is optional and appears when the {\bf print
rdf} option is specified in the CONTROL file.

\subsection{The VAFDAT Files}
\label{vaf-files}

These are formatted files containing {\em Velocity Autocorrelation
Function} (VAF) data.  An individual file is created for each atomic
species, i.e. {\tt VAFDAT}\_{\em atname}. Their contents are as follows:
\begin{tabbing}
X\=XXXXXXXX\=XXXXXXXXXXXX\=XXXXXXXXXX\=\kill
{\bf record} \\
\> {\tt cfgname} \> a72    \> configuration name \\
\end{tabbing}
There follow the data for the VAF, either a single time-averaged
profile or successive profiles separated by two blank lines.
The data supplied are as follows:
\begin{tabbing}
X\=XXXXXXXX\=XXXXXXXXXXXX\=XXXXXXXXXX\=\kill
{\bf first record} \\
\> {\tt atname}    \> a8      \> atom name \\
\> {\tt binvaf}    \> integer \> number of data points in VAF profile, {\em excluding} $t=0$ \\
\> {\tt vaforigin} \> real    \> absolute value of VAF at $t=0$ ($C(0) \equiv 3k_B T/m$) \\
\> {\tt vaftime0}  \> real    \> simulation time (ps) at beginning of (last) VAF profile ($t=0$) \\
{\bf following records} ({\em binvaf}+1 records) \\
\> {\tt t}         \> real    \> time (ps) \\
\> $\tt Z(t)$      \> real    \> scaled velocity autocorrelation function ($C(t)/C(0)$) at given time $t$
\end{tabbing}

{\bf Note} the VAFDAT files are optional and appear when the
{\bf print vaf} option is specified in the CONTROL file.

\subsection{The \textrm{\textit{\textbf{INT}}}DAT,
\textrm{\textit{\textbf{INT}}}PMF \& \textrm{\textit{\textbf{INT}}}TAB Files}
\label{bonded-files}

These files, where \textrm{\textit{\textbf{INT}}} is referring
to \textrm{\textit{\textbf{INT}}}ra-molecular interactions and
\textrm{\textit{\textbf{VDW}}}(RDF derived inter-molecular),
have very similar formatting rules with some examples shown in
Section~\ref{IPDF-analysis}.  Refer to Section~\ref{IPDF-analysis}
for their meaning and usage in coarse grained model systems.

\begin{tabbing}
X\=XXXXXXXX\=XXXXXXXX\=\kill
{\bf record 1} \\
\> {\tt \# title}  \> a100    \> file header title \\
{\bf record 2} \\
\> {\tt \# header} \> a100    \> file information header \\
{\bf record 3} \\
\> {\tt \# info}   \> a30     \> information to follow string \\
\> {\tt bins}   \> integer \> number of bins for all PDFs \\
\> {\tt cutoff} \> real    \> cutoff in \AA~for bonds and RDFs or degrees \\
\>                 \>         \> for angular intramolecular interactions \\
\> {\tt frames} \> integer \> number of sampled configurations \\
\> {\tt types}  \> integer \> number of unique types of these interactions \\
{\bf record 4} \\
\> {\tt \#}        \> a1      \> a hash (\#) symbol \\
{\bf record 5} \\
\> {\tt \# info 1} \> a100    \> information to follow string \\
{\bf record 6} \\
\> {\tt \#}        \> a1      \> a hash (\#) symbol
\end{tabbing}

The subsequent records define each PDF potential in turn,
in the order indicated by the specification in the FIELD file.
Each potential is defined by a header record and a set of data
records with the potential-like and force-like tables.

\begin{tabbing}
X\=XXXXXXXX\=XXXXXXXX\=\kill
{\bf empty record:} \\
{\bf id record:} \\
\> {\tt \# info}   \> a25     \> information to follow string \\
\> {\tt atom 1}    \> a8      \> first atom type \\
\> {\tt atom 2}    \> a8      \> second atom type \\
\> {\tt atom 3}    \> a8      \> third atom type - only available in ANG* files \\
\> {\tt atom 4}    \> a8      \> forth atom type - only available in DIH* \& INV* files \\
\> {\tt index}     \> integer \> unique index of PDF in file \\
\> {\tt instances} \> integer \> instances of this unique type of PDF \\
{\bf interaction data records 1--bins:} \\
\> {\tt abscissa}  \> real \> consecutive value over the full cutoff/range in \\
\>                 \>      \> \AA~for BNDTAB \& VDWTAB and degrees for ANGTAB, DIHTAB \& INVTAB \\
\> {\tt potential} \> real \> potential at the abscissa grid point in {\bf units} as specified in FIELD \\
\> {\tt force}     \> real \> complementary force (virial for BNDTAB \& VDWTAB) value
\end{tabbing}

\subsection{The STATIS File}
\label{statis-file}

The file is formatted, with integers as ``i10'' and reals as
``e14.6''.  It is written by the subroutine {\sc
statistics\_collect}.  It consists of two header records followed
by many data records of statistical data.
\begin{tabbing}
X\=XXXXXXXX\=XXXXXXXXXXXX\=XXXXXXXXXX\=\kill
{\bf record 1} \\
\> {\tt cfgname} \> a72 \> configuration name \\
{\bf record 2} \\
\> {\tt string}  \> a8  \> energy units
\end{tabbing}

{\bf Data records}\\ Subsequent lines contain the instantaneous
values of statistical variables dumped from the array {\tt
stpval}.  A specified number of entries of {\tt stpval} are
written in the format ``(1p,5e14.6)''.  The number of array
elements required (determined by the parameter {\tt mxnstk} in the
{\sc setup\_module} file) is
\begin{eqnarray}
{\tt mxnstk} & \ge & 28 + {\tt ntpatm}~(\rm number~of~unique~atomic~sites) ~+ \nonumber \\
             &     & 9~(\rm stress~tensor~elements) ~+ \nonumber \\
             &     & 10~(\rm if~constant~pressure~simulation~requested) ~+ \nonumber \\
             &     & 2~(\rm if~iso~>~0~requested) + 2~(\rm if~iso~>~1~requested) ~+ \nonumber \\
             &     & 2*mxatdm~(\rm if~msdtmp~option~is~used) \nonumber
\end{eqnarray}
The STATIS file is appended at intervals determined by the {\bf
stats} directive in the CONTROL file.  The energy unit is as
specified in the FIELD file with the {\bf units} directive, and
are compatible with the data appearing in the OUTPUT file.  The
contents of the appended information of calculated
{\em instantaneous} observables is:
\begin{tabbing}
X\=XXXXXXXXXXXX\=XXXXXXXXXXXX\=XXXXXXXXXXXX\=\kill
{\bf record i} \\
\> {\tt nstep}  \> integer \> current MD time-step \\
\> {\tt time }  \> real    \> elapsed simulation time \\
\> {\tt nument} \> integer \> number of array elements to follow \\
{\bf record ii} {\tt stpval}(1) -- {\tt stpval}(5) \\
\> {\tt engcns} \> real    \> total extended system energy, $E^{x}_{tot}=(E_{kin}+E_{rot})+E_{conf}+E_{consv}$ \\
\>              \>         \> (i.e. including the conserved quantity, $E_{consv}$) \\
\> {\tt temp}   \> real    \> system temperature, $2\frac{E_{kin}+E_{rot}}{f k_{B}}$ \\
\> {\tt engcfg} \> real    \> configurational energy, $E_{conf}$ \\
\> {\tt engsrc} \> real    \> short range potential energy \\
\> {\tt engcpe} \> real    \> electrostatic energy \\
{\bf record iii} {\tt stpval}(6) -- {\tt stpval}(10) \\
\> {\tt engbnd} \> real    \> chemical bond energy \\
\> {\tt engang} \> real    \> valence angle and 3-body potential energy \\
\> {\tt engdih} \> real    \> dihedral, inversion, and 4-body potential energy \\
\> {\tt engtet} \> real    \> tethering energy \\
\> {\tt enthal} \> real    \> enthalpy ($E^{x}_{tot} + {\cal P} \cdot V$) for NVE/T/E$_{kin}$ ensembles \\
\>              \>         \> enthalpy ($E^{x}_{tot} + P \cdot {\cal V}$) for NP/$\sigma$T or NP$_{n}$A/$\gamma$ ensembles \\
{\bf record iv} {\tt stpval}(11) -- {\tt stpval}(15) \\
\> {\tt tmprot} \> real    \> rotational temperature, $E_{rot}$ \\
\> {\tt vir}    \> real    \> total virial \\
\> {\tt virsrc} \> real    \> short-range virial \\
\> {\tt vircpe} \> real    \> electrostatic virial \\
\> {\tt virbnd} \> real    \> bond virial \\
{\bf record v} {\tt stpval}(16) -- {\tt stpval}(20) \\
\> {\tt virang} \> real    \> valence angle and 3-body virial \\
\> {\tt vircon} \> real    \> constraint bond virial \\
\> {\tt virtet} \> real    \> tethering virial \\
\> {\tt volume} \> real    \> volume, ${\cal V}$ \\
\> {\tt tmpshl} \> real    \> core-shell temperature \\
{\bf record vi} {\tt stpval}(21) -- {\tt stpval}(25) \\
\> {\tt engshl} \> real    \> core-shell potential energy \\
\> {\tt virshl} \> real    \> core-shell virial \\
\> {\tt alpha } \> real    \> MD cell angle $\alpha$ \\
\> {\tt beta }  \> real    \> MD cell angle $\beta$ \\
\> {\tt gamma } \> real    \> MD cell angle $\gamma$ \\
{\bf record vii} {\tt stpval}(26), {\tt stpval}(27), {\tt stpval}(0) \\
\> {\tt virpmf} \> real    \> PMF constraint virial \\
\> {\tt press}  \> real    \> pressure, ${\cal P}$ \\
\> {\tt consv}  \> real    \> extended DoF energy, $E_{consv}$ \\
{\bf the next {\tt ntpatm} entries} \\
\> {\tt amsd(1)} \> real   \> mean squared displacement of first atom types \\
\> {\tt amsd(2)} \> real   \> mean squared displacement of second atom types \\
\> {\tt ...} \> ... \> ... \\
\> {\tt amsd(ntpatm)} \> real \> mean squared displacement of last atom types \\
{\bf the next 9 entries for the stress tensor in pressure units} \\
\> {\tt stress(1)} \> real \> xx component of stress tensor \\
\> {\tt stress(2)} \> real \> xy component of stress tensor \\
\> {\tt stress(3)} \> real \> xz component of stress tensor \\
\> {\tt stress(4)} \> real \> yx component of stress tensor \\
\> {\tt ...} \> real \> ... \\
\> {\tt stress(9)} \> real \> zz component of stress tensor \\
{\bf the next 10 entries - {\em if} a NPT or N$\mat{\sigma}$T simulation is undertaken} \\
\> {\tt cell(1)} \> real   \> x component of $a$ cell vector \\
\> {\tt cell(2)} \> real   \> y component of $a$ cell vector \\
\> {\tt cell(3)} \> real   \> z component of $a$ cell vector \\
\> {\tt cell(4)} \> real   \> x component of $b$ cell vector \\
\> {\tt ...} \> real \> ... \\
\> {\tt cell(9)} \> real   \> z component of $c$ cell vector \\
\> {\tt stpipv} \> real    \> pressure, ${\cal P} \cdot {\cal V}$ \\
{\bf the next 2 entries - {\em if} a NPT or NP$_{n}$AT simulation is undertaken} \\
\> {\tt h\_z}   \> real    \> MD cell height $h_{z}$ to normal surface ${\cal A}\perp{z}$ \\
\> {\tt A$\perp${z}} \> real \> MD cell normal surface ${\cal A}\perp{z}={\cal V}/h_{z}$ \\
{\bf the next 2 entries - {\em if} a NPT or N$\gamma_{n}$AT simulation is undertaken} \\
\> {\tt gamma\_x} \> real \> surface tension $\gamma_{n_{x}}$ on normal surface ${\cal A}\perp{z}$ \\
\> {\tt gamma\_y} \> real \> surface tension $\gamma_{n_{y}}$ on normal surface ${\cal A}\perp{z}$
\end{tabbing}

\subsection{The LATS\_E and LATS\_I Files}
\label{latsei-files}

These are formatted files containing electronic (LATS\_E)
and ionic (LATS\_I)temperatures at user-requested intervals
along the y-direction in the centre of the system's xz-plane 
from \index{Two-Temperature Model} two-temperature model
calculations.

Each line in these files consists of a series of electronic or ionic 
temperatures along the y-direction -- {\tt -eltsys(2)/2} $\le y \le$ {\tt +eltsys(2)/2} 
and {\tt -ntsys(2)/2} $\le y \le$ {\tt +ntsys(2)/2} at $x=z=0$ -- 
corresponding to a requested timestep. The number of values in
each line will depend on the number of electronic or ionic temperature
cells requested by the user. 

\subsection{The PEAK\_E and PEAK\_I Files}
\label{peakei-files}

These are formatted files containing statistics from 
\index{Two-Temperature Model} two-temperature model
calculations at user-requested intervals. Each line in these 
files corresponds to a requested time step and the data is
based upon active coarse-grained electronic (CET) and 
ionic (CIT) temperature grid cells. 

In the PEAK\_E file, the data are formatted as follows:
\begin{tabbing}
X\=XXXXXXXXXXXX\=XXXXXXXXXXXX\=XXXXXXXXXXXX\=\kill
{\bf record i} \\
\> {\tt nstep}  \> integer \> current MD time-step \\
\> {\tt time }  \> real    \> elapsed simulation time \\
\> {\tt eltemp\_min} \> real \> minimum value of electronic temperature in system (K) \\
\> {\tt eltemp\_max} \> real \> maximum value of electronic temperature in system (K) \\
\> {\tt eltemp\_mean} \> real \> mean value of electronic temperature in system (K) \\
\> {\tt eltemp\_sum} \> real \> sum of electronic temperatures in system (K) \\
\> {\tt Ue} \> real \> total electronic energy in system (eV) \\
\end{tabbing}
The PEAK\_I file is formatted in a similar fashion, as follows:
\begin{tabbing}
X\=XXXXXXXXXXXX\=XXXXXXXXXXXX\=XXXXXXXXXXXX\=\kill
{\bf record i} \\
\> {\tt nstep}  \> integer \> current MD time-step \\
\> {\tt time }  \> real    \> elapsed simulation time \\
\> {\tt tempion\_min} \> real \> minimum value of ionic temperature in system (K) \\
\> {\tt tempion\_max} \> real \> maximum value of ionic temperature in system (K) \\
\> {\tt tempion\_mean} \> real \> mean value of ionic temperature in system (K) \\
\> {\tt tempion\_sum} \> real \> sum of ionic temperatures in system (K) \\
\end{tabbing}
