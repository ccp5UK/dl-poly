\section{Parallelisation}

\D is a distributed parallel molecular dynamics package based on
the Domain Decomposition parallelisation strategy
\cite{todorov-04a,todorov-06a,pinches-91a,rapaport-91b,smith-91a,smith-93a}.
In this section we briefly outline the basic methodology.  Users
wishing to add new features \D will need to be familiar with the
underlying techniques as they are described in the above
references.

\subsection{The Domain Decomposition Strategy}\label{parallelisation}

The Domain Decomposition (DD) strategy
\cite{todorov-04a,todorov-06a,smith-91a} is one of several ways to achieve
parallelisation\index{parallelisation} in MD.  Its name derives
from the division of the simulated system into equi-geometrical
spatial blocks or domains, each of which is allocated to a
specific processor of a parallel computer.  I.e. the arrays
defining the atomic coordinates $\vek{r}_{i}$, velocities
$\vek{v}_{i}$ and forces $\vek{f}_{i}$, for all $N$ atoms in the
simulated system, are divided in to sub-arrays of approximate size
$N/P$, where $P$ is the number of processors, and allocated to
specific processors.  In \D the domain allocation is handled by
the routine {\sc domains\_module} and the decision of approximate
sizes of various bookkeeping arrays in {\sc set\_bounds}.  The
division of the configuration data in this way is based on the
location of the atoms in the simulation cell, such a {\em
geometric} allocation of system data is the hallmark of DD
algorithms.  Note that in order for this strategy to work
efficiently, the simulated system must possess a reasonably
uniform density, so that each processor is allocated almost an equal
portion of atom data (as much as possible).  Through this approach
the forces computation and integration of the equations of motion
are shared (reasonably) equally between processors and to a large
extent can be computed independently on each processor.  The
method is conceptually simple though tricky to program and is
particularly suited to large scale simulations, where efficiency
is highest.

The DD strategy underpinning \D is based on the link cell
algorithm of Hockney and Eastwood \cite{hockney-81a} as
implemented by various authors (e.g. Pinches {\em et al.}
\cite{pinches-91a} and Rapaport \cite{rapaport-91b}).  This
requires that the cutoff applied to the interatomic potentials is
relatively short ranged.  In \D the link-cell list is build by the
routine {\sc link\_cell\_pairs}.  As with all DD algorithms, there
is a need for the processors to exchange `halo data', which in the
context of link-cells means sending the contents of the link cells
at the boundaries of each domain, to the neighbouring processors,
so that each may have all necessary information to compute the
pair forces acting on the atoms belonging to its allotted domain.
This in \D is handled by the {\sc set\_halo\_particles} routine.

Systems containing complex molecules present several difficulties.
They often contain ionic species, which usually require Ewald
summation\index{Ewald!summation} methods \cite{allen-89a,smith-92b},
and {\em intra}-molecular interactions in addition to
{\em inter}-molecular forces.  Intramolecular interactions are
handled in the same way as in \C, where each processor
is allocated a subset of intramolecular bonds to deal with.  The
allocation in this case is based on the atoms present in the
processor's domain.  The SHAKE\index{algorithm!SHAKE} and
RATTLE\index{algorithm!RATTLE} algorithms \cite{ryckaert-77a,andersen-83a}
require significant modification.  Each processor must deal with
the constraint bonds present in its own domain, but it must also
deal with bonds it effectively shares with its neighbouring
processors.  This requires each processor to inform its neighbours
whenever it updates the position of a shared atom during every
SHAKE (RATTLE\_VV1) cycle (RATTLE\_VV2 updates the velocities),
so that all relevant processors may incorporate this update into
its own iterations.  In the case of the DD strategy the SHAKE
(RATTLE) algorithm is simpler than for the Replicated Data method
of \C, where global updates of the atom positions
(merging and splicing) are required \cite{smith-93b}.  The
absence of the merge requirement means that the DD tailored
SHAKE and RATTLE are less communications dependent and thus more
efficient, particularly with large processor counts.

The DD strategy is applied to complex molecular systems as follows:
\begin{enumerate}
\item Using the atomic coordinates $\vek{r}_{i}$, each processor
calculates the forces acting between the atoms in its domain -
this requires additional information in the form of the halo data,
which must be passed from the neighbouring processors beforehand.
The forces are usually comprised of:
\begin{enumerate}
\item All common forms of non-bonded\index{potential!non-bonded} atom-atom (van der Waals) forces
\item Atom-atom (and site-site) coulombic\index{potential!electrostatics} forces
\item Metal-metal\index{potential!metal} (local density dependent) forces
\item Tersoff\index{potential!Tersoff} (local density dependent) forces (for hydro-carbons) \cite{tersoff-89a}
\item Three-body\index{potential!three-body} valence angle\index{potential!valence angle} and hydrogen bond\index{potential!bond} forces
\item Four-body\index{potential!four-body} inversion\index{potential!inversion} forces
\item Ion core-shell polarasation\index{polarisation!shell model}
\item Tether\index{potential!tether} forces
\item Chemical bond\index{potential!chemical bond} forces
\item Valence angle\index{potential!valence angle} forces
\item Dihedral angle\index{potential!dihedral} (and improper dihedral angle\index{potential!improper dihedral}) forces
\item Inversion angle\index{potential!inversion} forces
\item External field\index{potential!external field} forces.
\end{enumerate}
\item The computed forces are accumulated in atomic
force arrays $\vek{f}_{i}$ independently on each processor
\item The force arrays are used to update the atomic
velocities and positions  of all the atoms in the domain
\item Any atom which effectively moves from one domain to another, is
relocated to the neighbouring processor responsible for that domain.
\end{enumerate}
It is important to note that load balancing (i.e. equal and
concurrent use of all processors) is an essential requirement of
the overall algorithm.  In \D this is accomplished quite naturally
through the DD partitioning of the simulated system.  Note that
this will only work efficiently if the density of the system is
reasonably uniform.  {\sc There are no load balancing algorithms
in \D to compensate for a bad density distribution.}

\subsection{Distributing the Intramolecular Bonded Terms}

The intramolecular\index{parallelisation!intramolecular terms}
terms in \D are managed through bookkeeping arrays which
list all atoms (sites) involved in a particular interaction
and point to the appropriate arrays of parameters that define
the potential.  Distribution of the forces calculations
is accomplished by the following scheme:
\begin{enumerate}
\item Every atom (site) in the simulated system is assigned a
unique `global' index number from $1$ to $N$.
\item Every processor maintains a list of the local indices of
the atoms in its domain.  (This is the local atom list.)
\item Every processor also maintains a sorted (in ascending order)
local list of global atom indices of the atoms in its domain.
(This is the local sorted atom list.)
\item Every intramolecular\index{parallelisation!intramolecular
terms} bonded term $U_{type}$ in the system has a unique index
number $i_{type}$: from $1$ to $N_{type}$ where $type$ represents a
bond\index{potential!chemical bond}, angle\index{potential!valence angle},
dihedral\index{potential!dihedral}, or inversion\index{potential!inversion}.
Also attached there with unique index numbers are
core-shell\index{polarisation!shell model} units,
bond constraint\index{constraints!bond} units,
PMF constraint\index{constraints!PMF} units,
rigid body\index{rigid body} units and
tethered\index{potential!tether} atoms,
their definition by site rather than by chemical type.
\item On each processor a pointer array
$key_{type}(n_{type},i_{type})$ carries the indices of the
specific atoms involved in the potential term labelled
$i_{type}$~.  The dimension $n_{type}$ will be $1$ if the
term represents a tether\index{potential!tether}, $1,~2$
for a core-shell\index{polarisation!shell model} unit or
a bond constraint\index{constraints!bond} unit or a
bond\index{potential!chemical bond}, $1,~2,~3$ for a valence
angle\index{potential!valence angle} and $1,~2,~3,~4$ for a
dihedral\index{potential!dihedral} or an
inversion\index{potential!inversion}, $1,..,n_{\tt PMF~unit_{1~or~2}}+1$
for a PMF constraint\index{constraints!PMF} unit, or
$-1,~0,~1,..,n_{\tt RB~unit}$ for a rigid body\index{rigid body} unit.
\item Using the $key$ array, each processor can identify the global
indices of the atoms in the bond term and can use this in
conjunction with the local sorted atoms list {\em and a binary
search algorithm} to find the atoms in local atom list.
\item Using the local atom identity, the potential energy and force
can be calculated.
\end{enumerate}

It is worth mentioning that although rigid body\index{rigid body}
units are not bearing any potential parameters, their definition
requires that their topology is distributed in the same manner as
the rest of the intra-molecular like interactions.

Note that, at the start of a simulation \D allocates individual
bonded interactions to specific processors, based on the domains
of the relevant atoms (\D routine {\sc build\_book\_intra}).
This means that each processor does not have to handle every
possible bond term to find those relevant to its domain.  Also
this allocation is updated as atoms move from domain to domain
i.e. during the {\em relocation} process that follows the
integration of the equations of motion (\D routine {\sc
relocate\_particles}). Thus the allocation of bonded terms is
effectively dynamic, changing in response to local changes.

\subsection{Distributing the Non-bonded Terms}

\D calculates the non-bonded\index{potential!non-bonded} pair
interactions using the link cell algorithm due to Hockney and
Eastwood \cite{hockney-81a}.  In this algorithm a relatively short
ranged potential cutoff ($r_{\rm cut}$) is assumed.  The simulation cell
is logically divided into so-called link cells, which have a width
not less than (or equal to) the cutoff distance.  It is easy to
determine the identities of the atoms in each link cell.  When the
pair interactions are calculated it is already known that atom
pairs can only interact if they are in the same link cell, or are
in link cells that share a common face.  Thus using the link cell
`address' of each atom, interacting pairs are located easily and
efficiently via the `link list' that identifies the atoms in each
link cell.  So efficient is this process that the link list
can be recreated every time step at negligible cost.

For reasons, partly historical, the link list is used to construct
a Verlet\index{algorithm!Verlet neighbour list} neighbour list
\cite{allen-89a}.  The Verlet list records the indices of all
atoms within the cutoff radius ($r_{\rm cut}$) of a given atom.  The
use of a neighbour list is not strictly necessary in the context
of link-cells, but it has the advantage here of allowing a neat
solution to the problem of `excluded' pair interactions arising
from the intramolecular terms and frozen atoms (see below).

In \D, the neighbour list\index{Verlet neighbour list} is
constructed {\em simultaneously} on each node, using the DD
adaptation of the link cell algorithm to share the total burden of
the work reasonably equally between nodes.  Each node is thus
responsible for a unique set of non-bonded interactions and the
neighbour list is therefore different on each node.

A feature in the construction of the Verlet\index{algorithm!Verlet}
neighbour list for macromolecules is the concept of
{\em excluded atoms}, which arises from the need to exclude
certain atom pairs from the overall list.  Which atom pairs need
to be excluded is dependent on the precise nature of the force
field\index{force field} model, but as a minimum atom pairs linked
via extensible bonds\index{potential!chemical bond} or
constraints\index{constraints!bond} and atoms (grouped in pairs)
linked via valence\index{potential!valence angle} angles are
probable candidates.  The assumption behind this requirement is
that atoms that are formally bonded\index{potential!bonded} in a
chemical sense, should not participate in
non-bonded\index{potential!non-bonded} interactions.  (However,
this is not a universal requirement of all force fields\index{force field}.)
The same considerations are needed in
dealing with charged excluded atoms.

The modifications necessary to handle the excluded and frozen atoms
are as follows.  A distributed {\em excluded atoms list} is
constructed by the \D routine {\sc build\_excl\_intra} at the start
of the simulation and is then used in conjunction with the Verlet
neighbour list builder {\sc link\_cell\_pairs} to ensure that
excluded interactions are left out of the pair force calculations.
Note that, completely frozen pairs of atoms are excluded in the same
manner.  The excluded atoms list is updated during the atom relocation
process described above (\D routine {\sc exchange\_particles}).

Once the neighbour list has been constructed, each node of the
parallel computer may proceed independently to calculate the pair
force contributions to the atomic forces (see routine {\sc
two\_body\_forces}).

The potential energy and forces arising from the
non-bonded\index{potential!non-bonded} interactions, as well as
metal\index{potential!metal} and Tersoff\index{potential!Tersoff}
interactions are calculated using interpolation tables.  These
are generated in the following routines: {\sc vdw\_generate},
{\sc metal\_generate}, {\sc metal\_table\_derivatives} and
{\sc tersoff\_generate}.

\subsection{Modifications for the Ewald Sum}

For systems with periodic boundary conditions \D employs the Ewald
Sum\index{Ewald!summation} to calculate the coulombic interactions
(see Section~\ref{SPME}).  It should be noted that \D uses only the
Smoothed Particle Mesh (SPME) form of the Ewald sum.

Calculation of the real space component in \D employs the
algorithm for the calculation of the non-bonded interactions
outlined above, since the real space interactions are now short
ranged (implemented in {\sc ewald\_real\_forces} routine).

The reciprocal space component is calculated using Fast Fourier
Transform (FFT) scheme of the SMPE method \cite{essmann-95a,bush-06a} as
discussed in Section~\ref{SPME}.  The parallelisation of this scheme is
entirely handled within the \D by the 3D FFT routine {\sc parallel\_fft},
(using {\sc gpfa\_module}) which is known as the Daresbury advanced
Fourier Transform, due to I.J. Bush \cite{bush-00a}.  This routine
distributes the SPME charge array over the processors in a manner
that is completely commensurate with the distribution of the
configuration data under the DD strategy.  As a consequence the
FFT handles all the necessary communication implicit in a
distributed SPME application.  The \D subroutine {\sc ewald\_spme\_forces}
perfoms the bulk of the FFT operations and charge array construction,
while {\sc spme\_forces} calculates the forces.

Other routines required to calculate the Ewald sum include {\sc ewald\_module}, \\
{\sc ewald\_excl\_forces}, \\ {\sc ewald\_frozen\_forces} and {\sc spme\_container}.

\subsection{Metal Potentials}

The simulation of metals (Section~\ref{metal}) by \D makes use of density
dependent potentials.  The dependence on the atomic density presents
no difficulty however, as this class of potentials can be resolved
into pair contributions.  This permits the use of the distributed
Verlet neighbour list\index{Verlet neighbour list} as
outlined above.  \D implements these potentials in various
subroutines with names beginning with {\sc metal\_}.

\subsection{Tersoff, Three-Body and Four-Body Potentials}

\D can calculate Tersoff, three-body and four-body interactions.
Although some of these interactions have similar terms to some
intramolecular ones (three-body to the bond angle and four-body to
inversion angle), these are not dealt with in the same way as the
normal bonded\index{potential!bonded} interactions.  They are
generally very short ranged and are most effectively calculated
using a link-cell scheme \cite{hockney-81a}.  No reference is made
to the Verlet neighbour list\index{Verlet neighbour list} nor the
excluded atoms list.  It follows that atoms involved these
interactions can interact via non-bonded (pair) forces and ionic
forces also.  Note that contributions from frozen pairs of atoms
to these potentials are excluded.  The calculation of the Tersoff
three-body and four-body terms is distributed over
processors on the basis of the domain of the central atom in them.
\D implements these potentials in the following routines
{\sc tersoff\_forces}, {\sc tersoff\_generate},
{\sc three\_body\_forces} and {\sc four\_body\_forces}.

\subsection{Globally Summed Properties}

The final stage in the DD strategy, is the global summation of
different (by terms of potentials) contributions to energy, virial
and stress, which must be obtained as a global sum of the
contributing terms calculated on all nodes.

The DD strategy does not require a global summation of the forces,
unlike the Replicated Data method used in \C, which
limits communication overheads and provides smooth parallelisation
to large processor counts.

\subsection{The Parallel (DD tailored) SHAKE and RATTLE Algorithms}

The essentials of the DD tailored SHAKE\index{algorithm!SHAKE} and
RATTLE\index{algorithm!RATTLE} algorithms (see Section~\ref{shake-rattle})
are as follows:
\begin{enumerate}
\item The bond constraints acting in the simulated system are allocated
between the processors, based on the location (i.e. domain) of
the atoms involved.
\item Each processor makes a list of the atoms bonded by
constraints\index{constraints!bond} it must process. Entries are zero
if the atom is not bonded.
\item Each processor passes a copy of the array to the neighbouring
processors which manage the domains in contact with its own.
The receiving processor compares the incoming list with its own and
keeps a record of the shared atoms and the processors which share
them.
\item In the first stage of the algorithms, the atoms
are updated through the usual Verlet\index{algorithm!Verlet}
algorithm, without regard to the bond
constraints\index{constraints!bond}.
\item In the second (iterative) stage of the algorithms, each
processor calculates the incremental correction vectors for the
bonded\index{potential!bond} atoms in its own list of bond
constraints.  It then sends specific correction vectors to all
neighbours that share the same atoms, using the information
compiled in step 3.
\item When all necessary correction vectors have been received and
added the positions of the constrained atoms are corrected.
\item Steps 5 and 6 are repeated until the bond constraints are
converged.
\item Finally, the change in the atom positions from the previous
time step is used to calculate the atomic velocities.
\end{enumerate}

The compilation of the list of constrained atoms on each
processor, and the circulation of the list (items 1 - 3 above) is
done at the start of the simulation, but thereafter it needs only
to be done every time a constraint bond atom is relocated from one
processor to another.  In this respect DD-SHAKE and DD-RATTLE
resemble every other intramolecular term.

Since the allocation of constraints is based purely on geometric
considerations, it is not practical to arrange for a strict load
balancing of the DD-SHAKE and DD-RATTLE algorithms.  For many
systems, however, this deficiency has little practical impact on
performance.

\subsection{The Parallel Rigid Body Implementation}

The essentials of the DD tailored RB algorithms (see Section~\ref{rigid})
are as follows:
\begin{enumerate}
\item Every processor works out a list of all local and halo atoms
that are qualified as free (zero entry) or as members of a RB
(unit entry.
\item The rigid body units in the simulated system are allocated
between the processors, based on the location (i.e. domain) of
the atoms involved.
\item Each processor makes a list of the RB and their constituting
atoms that are fully or partially owned by the processors domain.
\item Each processor passes a copy of the array to the neighbouring
processors which manage the domains in contact with its own.
The receiving processor compares the incoming list with its own and
keeps a record of the shared RBs and RBs' constituent atoms, and
the processors which share them. {\em Note that a RB can be shared
between up to {\bf eight} domains!}
\item The dynamics of each RB is calculated in full on each domain
but domains only update $\{\vek{r},\vek{v},\vek{f}\}$ of RB atoms which
they own.  {\em Note that a site/atom belongs to {\bf one and
only one} domain at a time (no sharing) !}
\item Strict bookkeeping is necessary to avoid multiple counting of
kinetic properties.  $\{\vek{r},\vek{v},\vek{v}\}$ updates are
necessary for halo parts (particles) of partially shared RBs.  For
all domains the kinetic contributions from each fully or partially
present RB are evaluated in full and then waited with the ratio -
number of RB's sites local to the domain to total RB's sites, and
then globally summed.
\end{enumerate}

The compilation of the lists in items 1 - 3 above and their
circulation of the list is done at the start of the simulation,
but thereafter these need updating on a local level every time
a RB site/atom is relocated from one processor to another.  In this
respect RBs topology transfer resembles every other intramolecular term.

Since the allocation of RBs is based purely on geometric
considerations, it is not practical to arrange for a strict load
balancing.  For many systems, however, this deficiency has little
practical impact on performance.
