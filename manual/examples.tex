\section{Example Simulations}

Because of the size of the data files for the \D
example simulations, they are not shipped in the standard
download of the \D source.  Instead users are requested
to download them from the CCP5 FTP server as follows:

\begin{lstlisting}
FTP site : ftp.dl.ac.uk
Username : anonymous
Password : your email address
Directory: ccp5/DL_POLY/DL_POLY_4.0/DATA
Files    : test_X.tar.xz
\end{lstlisting}

where `\_X' stands for the example simulation number.

Remember to use the BINARY data option when transferring these
files.

Unpack the files in the `data' subdirectory using `gunzip' and
`tar -xf' to create the `TEST\_X' directory.

These are provided to give examples of \D simulations
and demonstrate a limited set of relevant functionality over
a limited extent of molecular systems' complexity only.
{\bf Without modification, they are not necessarily appropriate
for serious simulation of the given systems.} In other words,
the examples are not warranted to have well-defined force
fields in terms of applicability, transferability and fullness, nor
are they likely to have a well-defined state point (i.e. initial
configurations may be away from equilibrium, if physical at all).

The README.txt file supplied both in the {\em data} directory and
in the directory on the CCP5 FTP server provides a list of all
example simulations used as test cases to check that \D is working
correctly, including those described in more detail below.  All the
jobs are of a size suitable to test the code in parallel execution.  They
may not be suitable for a single processor computer.  The files
are stored in compressed format.  The examples can be run by typing

{\sl select } n

\noindent from the {\em execute} directory, where n is the number of
the test case.  The {\sl select} macro will copy the appropriate
input files (at least CONTROL, CONFIG, and FIELD in all cases)
to the {\em execute} directory ready for execution.  The output file
OUTPUT may be compared with the file supplied in the {\em data}
directory.

\subsection{Example 1: Sodium Chloride}

This is a 27,000 ion system with unit electric charges on sodium and
chlorine.  Simulation at 500~K with a NVT Berendsen ensemble.  The SPME
method is used to calculate the Coulombic interactions.

\subsection{Example 2: DPMC in Water}

The system consists of 200 DMPC molecules in 9379 and water molecules.
Simulation at 300~K using NVE ensemble with SPME and RATTLE algorithm
for the constrained motion.  The total system size is 51,737 atoms.

\subsection{Example 3: KNaSi$_{2}$O$_{5}$ - Potassium/Sodium Disilicate Glass}

Potassium Sodium disilicate glass (NaKSi$_{2}$O$_{5}$) using two and
three-body potentials.  Some of the two-body potentials are read
from the TABLE file.  Simulation at 1000~K using NVT Nos\'e-Hoover
ensemble with SPME.  Cubic periodic boundaries are in use.  The
total system size is 69,120 ions.

\subsection{Example 4: Gramicidin A Molecules in Water}

The system consists of 8 gramicidin A molecules in aqueous solution
(32,096 water molecules) with total of 99,120 of atoms.  Simulation
at 300~K using NPT Berendsen ensemble with SPME and SHAKE/RATTLE
algorithm for the constrained motion.

\subsection{Example 5: SiC with Tersoff Potentials}

The system consists of 74,088 atoms.  Simulation at 300~K using NPT
Nos\'e-Hoover ensemble with Tersoff forces and no electrostatics.

\subsection{Example 6: Cu$_{3}$Au alloy with Sutton-Chen (metal) Potentials}

The systems consists of 32,000 atoms.  Simulation at 300~K using NVT
Nos\'e-Hoover ensemble with Sutton-Chen forces and no electrostatics.

\subsection{Example 7: Lipid Bilayer in Water}

The systems consists of 12,428 atoms.  Simulation at 300~K using NVT
Berendsen ensemble with SPME and SHAKE/RATTLE algorithm for the constrained motion.

\subsection{Examples 8 and 9: MgO with Adiabatic and with Relaxed Shell Models}

These system consist of 8,000 (4,000 shells) charged points.  Simulation
at 3000~K using NPT Berendsen ensemble with SPME.

\subsection{Example 10: Potential of Mean Force on K+ in Water}

The system consists of 13,500 (500 PMFs) atoms.  Simulation at 300~K
using NPT Berendsen ensemble with SPME and SHAKE/RATTLE algorithm for
the constrained motion.

\subsection{Example 11: Cu$_{3}$Au Alloy with Gupta (metal) Potentials}

The system consists of 32,000 atoms.  Simulation at 300~K using NVT
Nos\'e-Hoover ensemble with Gupta forces and no electrostatics.

\subsection{Example 12: Cu with EAM (metal) Potential}

The system consists of 32,000 atoms.  Simulation at 300~K using NPT
Berendsen ensemble with EAM tabulated forces and no electrostatics.

\subsection{Examples 13 and 14: Al with Analytic and with EAM Tabulated Sutton-Chen (metal) Potentials}

The system consists of 32,000 atoms.  Simulation at 300~K using NVT
Evans ensemble with Sutton-Chen forces and no electrostatics.

\subsection{Examples 15: NiAl Alloy with EAM (metal) Potentials}

The system consists of 27,648 atoms.  Simulation at 300~K using NVT
Evans ensemble with EAM tabulated forces and no electrostatics.

\subsection{Examples 16: Fe with Finnis-Sincair (metal) Potential}

The system consists of 31,250 atoms.  Simulation at 300~K using NPT
Berendsen ensemble with Finnis-Sinclair forces and no electrostatics.

\subsection{Examples 17: Ni with EAM (metal) Potential}

The system consists of 32,000 atoms.  Simulation at 300~K using NPT
Berendsen ensemble with EAM tabulated forces and no electrostatics.

\subsection{Examples 18 and 19: SPC IceVII Water with CBs and with RBs}

The system consists of 11,664 (34,992 atoms) water molecules. Simulation
at 25~K using NVE ensemble with CGM force minimisation and SPME electrostatics.

\subsection{Example 20: NaCl Molecules in SPC Water Represented as CBs+RBs}

The system consists of 64 NaCl ion pairs with 4,480 water molecules
represented by constraint bonds and 4,416 water molecules represented
by ridig bodies.  Totalling 26,816 atoms.  Simulation at 295~K using NPT
Berendsen ensemble with CGM energy minimisation and SPME electrostatics.

\subsection{Example 21: TIP4P Water: RBs with a Massless Charged Site}

The system consists of 7,263 TIP4P rigid body water molecules totaling
29,052 particles.  Simulation at 295~K using NPT Berendsen ensemble with
CGM energy minimisation and SPME electrostatics.

\subsection{Example 22: Ionic Liquid Dimethylimidazolium Chloride as RBs}

The system consists of 44,352 ions.  Simulation at 400~K using NPT
Berendsen ensemble, using both particle and rigid body dynamics with SPME electrostatics.

\subsection{Example 23: Calcite Nano-Particles in TIP3P Water}

In this case 600 molecules of calcium carbonate in the calcite
structure form 8 nano-particles which are suspended in 6,904 water
molecules, represented by a flexible 3-centre TIP3P model.  Simulation
with SPME electrostatics at 310~K and 1~atmosphere maintained in
a Hoover NPT ensemble.  The system consists of 23,712 ions.

\subsection{Example 24: Iron/Carbon Alloy with 2BEAM (metal) Potentials}

In this case a steel alloy of iron and carbon in ratio
35132 to 1651 is modelled using an EEAM potential forcefield.
Simulation at 1000~K and 0~atmosphere is maintained in a
Berendsen NPT ensemble.  The system consists of 36,803 particles.

\subsection{Example 25: Iron/Chromium Alloy with 2BEAM (metal) Potentials}

In this case a steel alloy of iron and chromium in ratio
27635 to 4365 is modelled using an 2BEAM potential forcefield.
Simulation at 300~K and 0~atmosphere is maintained in an Evans NVT
isokinetic ensemble.  The system consists of 32,000 particles.

\subsection{Examples 26 and 27: Hexane and Methanol Melts, with
Full Atomistic and Coarse-Grained Force-Fields}

These two examples contain a Hexane and a Methanol melt respectively,
(1000 molecules each) modelled by the OPLSAA force-field (FF).
Each system is also supplied in a CG-mapped representation as converted
by VOTCA, \href{http://www.votca.org/}{http://www.votca.org/}\index{WWW},
or DL\_CGMAP \href{http://www.ccp5.ac.uk/projects/ccp5\_cg.shtml}{http://www.ccp5.ac.uk/projects/ccp5\_cg.shtml}\index{WWW}.

These test cases are to exemplify the Coarse-Graining (CG) procedure
(see Chapter~\ref{coarse-graining}), including FA-to-CG mapping and
obtaining the PMF data by means of Boltzmann Inversion \cite{reith-03a}.
As a result, \D could be used for simulating a CG system with numerically
defined, tabulated FFs, see TABBND, TABANG, TABDIH and TABINV files for
intra-molecular potentials, and TABLE for inter-molecular (short-range, VDW)
potentials.

Both tests are also available as parts of the tutorial cases from the
VOTCA package \cite{ruhle-09a}.  Therefore, the CONFIG, CONTROL and FIELD
input files are fully consistent with the corresponding setup files found
in the VOTCA tutorial directories ``csg-tutorials/hexane'' and ``csg-tutorials/methanol'.

\subsection{Example 28: Butane in CCl$_{4}$ Solution with Umbrella Sampling via PLUMED}

Free Energy calculation for Buthane with respect to the dihedral angle as
collective variable.  We use umbrella sampling as implemented in PLUMED.

PLUMED enabling in CONTROL:
\begin{lstlisting}
plumed input umbrella.dat
\end{lstlisting}

Contents of umbrella.dat:
\begin{lstlisting}
phi: TORSION ATOMS=1,2,3,4
restraint-phi: RESTRAINT ARG=phi KAPPA=500 AT=1.20
PRINT STRIDE=10 ARG=phi,restraint-phi.bias FILE=COLVAR
\end{lstlisting}

Two extra output files are generated in this case: OUTPUT.PLUMED and COLVAR.

{\bf Note}, a \D version with PLUMED enabled is used for this.

\subsection{Example 29: Iron with tabulated EAM (metal) Potential, TTM and Cascade}

In this example 54,000 atoms of iron are modelled with a tabulated
embedded-atom potential optimised to produce correct energetics of point
defects and clusters (M07 in \cite{malerba-10a}). An energy impact of 10~keV
is applied to an atom and the resulting radiation damage is evolved using the
Two-Temperature Model (TTM) to represent energy transfers due to
electron-phonon coupling and electronic stopping between atoms
and a continuum electronic gas \cite{zarkadoula-14a}.

This test case produces additional output files: DUMP\_E, LATS\_E,
LATS\_I, PEAK\_E and PEAK\_I. It also requires an additional input file
(Ce.dat) to supply tabulated heat capacity data required for evolving the
electronic system.

\subsection{Example 30: Silicon with original Tersoff Potential, TTM and Swift heavy ion irradiation}

This system consists of 200,000 atoms of silicon modelled using an
original Tersoff (T3) potential. The Two-Temperature Model (TTM) is
in use and an energy deposition is applied to the electronic system
using a Gaussian spatial function, an exponentially decaying temporal
function and an electronic stopping power of 50,000~eV/nm.
This simulation represents Swift heavy ion irradiation in silicon, including
the resulting creation of ion tracks \cite{khara-16a}.

\subsection{Example 31: Tungsten with extended Finnis-Sinclair Potential, TTM and laser irradiation}

This system consists of 722,672 atoms of tungsten modelled using an
extended Finnis-Sinclair potential. The Two-Temperature Model (TTM)
is in use and an energy deposition is applied to the electronic system
using a spatial function that is homogeneous in x and y directions and
exponentially decaying in the z direction, as well as a Gaussian
temporal function. This energy deposition represents a laser applied to
the surface of a thin film of tungsten \cite{murphy-15a} with a surface
fluence of 36~mJ/cm$^2$ and penetration depth of 12.5~nm, causing
the film to expand outwards in the z direction.

Additional input files (Ce.dat and g.dat) are required to supply tabulated
heat capacity and electron-phonon coupling values.

\section{Benchmark Cases}

\D benchmark test cases are available to download them from the CCP5 FTP
server as follows:

\begin{lstlisting}
FTP site : ftp.dl.ac.uk
Username : anonymous
Password : your email address
Directory: ccp5/DL_POLY/DL_POLY_4.0/BENCH
\end{lstlisting}

The \D authors provide these on an "AS IS" terms.  For more information
refer to the README.txt file within.
