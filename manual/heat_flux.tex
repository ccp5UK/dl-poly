\section{Heat Flux}

\subsection{Introduction}
It is possible to use DL\_POLY\_4 to calculate the heat flux of a material with two-body interactions in an MD simulation.  The heat flux can subsequently be used as a means to calculate the thermal conductivity of a material via the Green-Kubo relation. Currently, the heat flux is only viable for two-body interactions, but valid for any two-body interactions.

To enable the calculation of heat flux add the {\bf heat\_flux} keyword into the CONTROL file.

\subsection{Theory}
The heat flux for two-body interactions is defined as:

\begin{equation}
\vek{J} = \frac{1}{V} \left[ \sum\limits^{N}_{i} e_{i} \vek{v}_{i} - \sum\limits^{N}_{i} \mat{S}_{i} \vek{v}_{i} \right]
\end{equation}
where $\mat{J}$ is the heat flux, $V$ is the volume of the cell, $N$ is the number of particles, $e$ is the energy, $\vek{v}$ is the velocity, $\mat{S}$ is the stress. All subscript $i$ refer to the particle.

The thermal conductivity can then be as an auto-correlation of the heat flux over a run:
\begin{equation}
  \kappa = \frac{V}{k_{B} T^{2}} \int\limits_{0}^{\infty} \langle \vek{J}(0)  \vek{J}(t) \rangle \, \mathrm{d}t
\end{equation}
where $\kappa$ is the thermal conductivity, $V$ is the volume of the cell, $k_{B}$ is the Boltzmann constant, $J$ is the heatflux.

\subsection{Implementation}
For the purposes of calculating per-particle SPME interactions, the long-range electrostatics forces and energies are calculated differently for the heat flux case. From the SPME equations, we can calculate a per particle contribution via:

\begin{equation*}
\begin{gathered}\Omega ^{\mathit{ABC}}=\sum _j\omega _j^{\mathit{ABC}}\\\omega _j^{\mathit{ABC}}=\frac 1{2\pi V}\sum
_{n_1n_2n_3=-\infty }^{\infty }\sum _{k_1}^{K_1-1}\sum _{k_2}^{K_2-1}\sum
_{k_3}^{K_3-1}Q_j^{\mathit{ABC}}(\mathbf k,\mathbf n)\sum _{\mathbf m\neq 0}\left[\prod _{\mu
}b_{\mu }(\mathbf m)e^{2\pi i\frac{m_{\mu }k_{\mu }}{K_{\mu }}}\right]f(\mathbf
m)\\Q_j^{\mathit{ABC}}(\mathbf k,\mathbf n)=q_j\left[\frac{K_{\alpha }}{2\pi i}\right]^A\frac{\partial
^A}{\partial u_{\alpha j}^A}M_n(u_{\alpha j}-k_{\alpha }-n_{\alpha }K_{\alpha })\times \\\left[\frac{K_{\beta }}{2\pi
i}\right]^B\frac{\partial ^B}{\partial u_{\beta j}^B}M_n(u_{\beta j}-k_{\beta }-n_{\beta }K_{\beta })\times
\\\left[\frac{K_{\gamma }}{2\pi i}\right]^C\frac{\partial ^C}{\partial u_{\gamma j}^C}M_n(u_{\gamma j}-k_{\gamma
}-n_{\gamma }K_{\gamma })\end{gathered}
\end{equation*}
where $\omega$ is the per-particle contribution for particle $j$, $q$ is the charge, other values are defined in the SPME section (see: \ref{SPME})
\subsection{File}
The heat flux method creates a file called HEATFLUX which contains the relevant data structured as:
STEP PRESSURE VOLUME HEAT-FLUX
