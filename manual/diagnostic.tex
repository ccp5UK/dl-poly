\section{Warning and Error Processing}

\subsection{The \D Internal Warning Facility}

\D contains a number of various in-built checks scattered
throughout the package which detect a range of possible
inconsistencies or errors.  In all cases, such a check fails the
subroutine {\sc warning} is called, resulting in an appropriate
message that identifies the inconsistency.  In some cases an
inconsistency is resolved by \D supplying a default value or \D
assuming a priority of one directive over the another (in clash of
mutually exclusive directives).  However, in other cases this
cannot be done and controlled termination of the program execution
is called by the subroutine {\sc error}.  In any case appropriate
diagnostic message is displayed notifying the user of the nature
of the problem.

\subsection{The \D Internal Error Facility}

\D contains a number of in-built error checks scattered throughout
the package which detect a wide range of possible errors.  In all
cases, when an error is detected the subroutine {\sc error} is
called, resulting in an appropriate message and termination of the
program execution (either immediately, or after some additional
processing).  In some case, if the cause for error is considered
to be mendable it is corrected and the subroutine {\sc warning}
results in an appropriate message.

Users intending to insert new error checks should ensure that all
error checks are performed {\em concurrently} on {\em all} nodes,
and that in circumstances where a different result may obtain on
different nodes, a call to the global status routine {\sc gcheck}
is made to set the appropriate global error flag on all nodes.
Only after this is done, a call to subroutine {\sc error} may be
made.  An example of such a procedure might be:

{\tt
\begin{tabbing}
XXXXXX\=\kill
\> Logical :: safe \\
\> safe = ({\em test\_condition}) \\
\> Call gcheck(safe) \\
\> If (.not.safe) Call error(message\_number)
\end{tabbing}
}

In this example it is assumed that the logical operation {\em
test\_condition} will result in the answer {\em .true.} if it is
safe for the program to proceed, and {\em .false.} otherwise.  The
call to {\sc error} requires the user to state the {\tt
message\_number} is an integer which used to identify the
appropriate message to be printed.

A full list of the \D error messages \index{error messages} and the
appropriate user action can be found in Appendix
\ref{error-messages} of this document.
