\section{Correlation Functions}

\subsection{Introduction}
\D includes on the fly computation of time correlations of observable quantities. This functionality allows for 
key correlation functions, and derived quantities, to be calculated
without saving trajectory data. 

The framework has been designed to support arbitrary correlation function. Currently, for the user, implemented observable quantities include: Atom velocity and the system's stress and 
heat flux (with two-body interactions, refer to Section~\ref{heat-flux}). Any pair of these observables can be composed into a correlation function. However a typical use case is to compute the
velocity (VAF), stress (SAF), and heat flux auto-correlations (HFAF). These allow the computation of the vibrational density of states (via a Fourier transform of the VAF) or shear-viscosity and thermal conductivity from Green-Kubo relations on the latter two. In fact the shear-viscosity and thermal conductivity are calculated whenever the required correlation functions are requested by the user in the (new style) CONTROL file (see Section~\ref{new-style}). For detailed usage instructions refer 
to the User Control in Section~\ref{cor-user}.

\subsection{Theory}

Taking as an example the shear-stress auto-correlation function (in discrete form),
\begin{equation}
    C^{\tau} = \frac{1}{T-\tau}\sum_{t'}^{T} \sigma_{xy}^{t'}\sigma_{xy}^{t'+\tau} \label{stress-cor}
\end{equation}
where $\tau$ indicates a discrete lag time, and $\sigma_{xy}^{t}$ indicates the $xy$ component of stress at discrete simulation time step $t$. Using a STATIS file (see Section~\ref{statis-file}) the values of $\sigma_{xy}^{t}$ can be read post simulation and calculated directly. This method however requires either saving stress data for all discrete time points $1,2, \ldots T$, or suffering a loss of accuracy by sub-sampling them. Additionally with correlation functions such as the velocity auto-correlation function, per-atom data is also required. Long timescale simulations and/or large system sizes
present a scaling issue for both memory and run-time.

\D utilises the Multiple-tau correlator \cite{Ramirez2010Efficient}, which is one particular on the fly correlation algorithm that addresses these issues. Briefly the method works by accumulating data in a series of hierarchical block averages. Three parameters control this {\bf correlation\_blocks}, {\bf correlation\_block\_points}, and {\bf correlation\_window} (written in terms of CONTROL directives). These control the number of hierarchical blocks, the number of distinct points within each block, and the length of an averaging window between blocks respectively. 

In more detail, given an empty correlator, as new data is submitted to it a sum is accumulated and the data points held in temporary storage. When the first block remains unfilled (less than {\bf correlation\_block\_points} data points have been submitted) the product of the new data point with each temporarily stored value is added to a correlation accumulator for this first block. At the first block all multiplications must be carried out, in subsequent blocks only points between ${\bf correlation\_block\_points}/{\bf correlation\_window}$ and {\bf correlation\_block\_points} need be updated. Once the first block contains {\bf correlation\_block\_points} data entries the sum divided by the {\bf correlation\_window} is passed to the next level, and the temporary data stored in the first level is cleared along with its accumulated sum (but not its accumulated correlation). In this way the complete correlation is
accumulated over a system's trajectory, with only ephemeral storage of ``raw'' data. As the simulation progresses, past values are retained at ever decreasing resolution replaced by current data. In terms of storage complexity the algorithm scales as $(p-1)w^b$ per correlation, where for brevity $p$ is {\bf correlation\_block\_points}, $w$ is {\bf correlation\_window} and $b$ is {\bf correlation\_blocks}. It should be noted that 
certain correlations, such as velocity, are computed on a per-particle basis. This requires $N$ correlators for a system of $N$ atoms. 

Particular correlation functions can be used to analyse the results of a simulation and compare to experimental systems. For example the SAF can be integrated to yield a Green-Kubo relation for sheer-viscosity. That is with analytic expressions for sheer-stress $\sigma_{xy}(t)$ in continuous time $t$, sheer-viscosity is

\begin{equation}
    \eta = \frac{V}{k_{b}T}\int_{0}^{\infty}dt' \langle \sigma_{xy}(0)\sigma_{xy}(t')\rangle.\label{viscosity-gk}
\end{equation}

Where $V$ and $T$ are the system volume and temperature respectively, with $k_{b}$ Boltzmann's constant. The integral can be performed numerically
over the discretised form of the correlation function in Equation~\ref{stress-cor} to estimate sheer-viscosity from simulation data. Similar relations exist for e.g. HFAF and thermal conductivity i.e.

\begin{equation}
    \lambda = \frac{V}{3k_{b}T^2}\int_{0}^{\infty} dt' \langle {\bf J}(0)\cdot {\bf J}(t') \rangle. \label{thermal-conductivity-gk}
\end{equation}

The prefactor includes a multiplication with volume due to the definition of heat flux, ${\bf J}(t)$, in \D already including a volume division, see Equation~\ref{heat-flux-definition}.

\subsection{User Control}\label{cor-user}

\subsubsection{Input}

In the (new style) CONTROL (see Section~\ref{new-style}) correlations are specified by an array of observable pairs in the format {\bf x-y} where {\bf x} and {\bf y} may take the string values in Table~\ref{tab-cor-control}. For example to compute the VAF and SAF one may write,

\begin{verbatim}
    correlation_observable [velocity-velocity s-s]
    correlation_block_points [600 5000]
    correlation_blocks [2 1]
    correlation_window [2 1]
\end{verbatim}

\D will then accumulate the VAF with 2 blocks each with 600 points per block, and a averaging length 2 and the SAF with a single block with
5000 points. By default {\bf correlation\_window} $=1$, {\bf correlation\_block\_points} $=100$ and {\bf correlation\_blocks} $=1$.

\begin{table}[]
    \centering
    \begin{tabular}{llll}
        \hline\hline
        String & Short-hand & Observable & Per-particle \\
        \hline
        {\bf velocity} & {\bf v} & particle velocity & yes \\
        \hline
        {\bf stress}   & {\bf s} & system stress & no \\
        \hline
        {\bf heat\_flux} & {\bf hf} & system heat flux & no \\
        \hline\hline
    \end{tabular}
    \caption{User control directives in the new style CONTROL file for on the fly correlations. Any combination of observables can 
    be correlated. Observables indicated as per-particle require storage of data scaling with system size $N$, as one correlator for each atom is created.}
\end{table}\label{tab-cor-control}

Through a combination of the three parameters short or long timescale correlations may be computed. For example taking {\bf correlation\_blocks} $= 1$, {\bf correlation\_block\_points} $= 100$, and {\bf correlation\_window} $= 1$ will result in a maximum
lag time correlated of $99 \Delta t$ for simulation time step $\Delta t$. Whereas taking {\bf correlation\_window} $= 2$ and {\bf correlation\_blocks} $= 2$ will give a higher maximum correlation lag time, $(198 \Delta t)$, but the averaging window of $2$ will result in a small accuracy reduction. 

\subsubsection{Output}

When correlation functions are specified by the user the resulting data is written as a YAML file, COR, containing the distinct correlations with their lag times, components, and any derived quantities. For example when computing the SAF the derived viscosity value for the simulation is automatically calculated, in this case the COR output file may look like the following

\begin{verbatim}
%YAML 1.2
---
title: argon fcc initial conditions
correlations:
    - name: [stress-stress                    , global]
      parameters:
            points_per_block: 5000
            number_of_blocks: 1
            window_size: 1
      derived:
            viscosity:
                  value:   0.57490361    
                  units: Katm ps 
      lags: [   0.0000000    ,  0.10000000E-03,  0.20000000E-03, ...]
      components: 
           stress_xx-stress_xx: [   1.0484384,   1.0484383,   1.0484382, ...]
           stress_xy-stress_xy: [                   ...                     ]
           ...
           stress_zz-stress_zz: [                   ...                     ]
\end{verbatim}