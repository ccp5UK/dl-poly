\section{Test Cases}

Because of the size of the data files for the \D
standard test cases, they are not shipped in the standard
download of the \D source.  Instead users are requested
to download them from the CCP5 FTP server as follows:

\begin{verbatim}
FTP site : ftp.dl.ac.uk
Username : anonymous
Password : your email address
Directory: ccp5/DL_POLY/DL_POLY_4.0/DATA
Files    : test_X.tar.gz
\end{verbatim}

where `\_X' stands for the test case number.

Remember to use the BINARY data option when transferring these
files.

Unpack the files in the `data' subdirectory using firstly
`gunzip' to uncompress them and then `tar -xf' to create the
`TEST\_X' directory.

These are provided so that you may check that your version of
\D is working correctly.  All the jobs are of a size suitable
to test the code in parallel execution.  They not not be
suitable for a single processor computer.  The files are stored
in compressed format.  The test cases can be run by typing

{\sl select } n

\noindent from the {\em execute} directory, where n is the number of
the test case.  The {\sl select} macro will copy the appropriate
CONTROL, CONFIG, and FIELD files to the {\em execute} directory
ready for execution.  The output file OUTPUT may be compared with
the file supplied in the {\em data} directory.

It should be noted that the potentials and the simulation
conditions used in the following test cases are chosen to
demonstrate a limited set of relevant functionality over
a limited extent of molecular systems' complexity only.
{\bf They are not necessarily appropriate for serious
simulation of the test systems.}  In other words, the
tests are not warranted to have well defined force field
in terms of applicability, transferability and fullness
as well as to have a well defined state point - thus
initial configurations may be away from equilibrium, if
physical at all.

\subsection{Test Case 1 and 2: Sodium Chloride}

These are a 27,000 and 216,000 ion systems respectively with unit
electric charges on sodium and chlorine.  Simulation at 500~K with a
NVT Berendsen ensemble.  The SPME method is used to calculate the
Coulombic interactions.

\subsection{Test Case 3 and 4: DPMC in Water}

These systems consist of 200 and 1,600 DMPC molecules in 9379 and
75032 water molecules respectively.  Simulation at 300~K using NVE
ensemble with SPME and RATTLE algorithm for the constrained motion.
Total system size is 51737 and 413896 atoms respectively.

\subsection{Test Case 5 and 6: KNaSi$_{2}$O$_{5}$}

Potassium Sodium disilicate glass (NaKSi$_{2}$O$_{5}$) using two and
three-body potentials.  Some of the two-body potentials are read
from the TABLE file.  Simulation at 1000~K using NVT Nos\'e-Hoover ensemble
with SPME.  Cubic periodic boundaries are in use. System size is
69120 and 552960 ions respectively.

\subsection{Test Case 7 and 8: Gramicidin A molecules in Water}

These systems consist of 8 and 16 gramicidin A molecules in aqueous
solution (32,096 and 256,768 water molecules) with total number of
atoms 99,120 and 792,960 respectively.  Simulation at 300~K using
NPT Berendsen ensemble with SPME and SHAKE/RATTLE algorithm for the
constrained motion.

\subsection{Test Case 9 and 10: SiC with Tersoff Potentials}

These systems consist of 74,088 and 343,000 atoms respectively.
Simulation at 300~K using NPT Nos\'e-Hoover ensemble with Tersoff forces
and no electrostatics.

\subsection{Test Case 11 and 12: Cu$_{3}$Au alloy with Sutton-Chen (metal) Potentials}

These systems consist of 32,000 and 256,000 atoms respectively.
Simulation at 300~K using NVT Nos\'e-Hoover ensemble with Sutton-Chen forces and
no electrostatics.

\subsection{Test Case 13 and 14: lipid bilayer in water}

These systems consist of 12,428 and 111,852 atoms respectively.
Simulation at 300~K using NVT Berendsen ensemble with SPME and
SHAKE/RATTLE algorithm for the constrained motion.

\subsection{Test Case 15 and 16: relaxed and adiabatic shell model MgO}

These systems consist of 8,000 (4,000 shells) and 64,000 (32,000
shells) atoms respectively.  Simulation at 3000~K using NPT
Berendsen ensemble with SPME.  FIELD and CONTROL files for each
shell model are provided separately.

\subsection{Test Case 17 and 18: Potential of mean force on K+ in water MgO}

These systems consist of 13,500 (500 PMFs) and 53,248 (2,048 PMFs)
atoms respectively.  Simulation at 300~K using NPT Berendsen
ensemble with SPME and SHAKE/RATTLE algorithm for the constrained
motion.

\subsection{Test Case 19 and 20: Cu$_{3}$Au alloy with Gupta (metal) Potentials}

These systems consist of 32,000 and 256,000 atoms respectively.
Simulation at 300~K using NVT Nos\'e-Hoover ensemble with Gupta forces and
no electrostatics.

\subsection{Test Case 21 and 22: Cu with EAM (metal) Potentials}

These systems consist of 32,000 and 256,000 atoms respectively.
Simulation at 300~K using NPT Berendsen ensemble with EAM tabulated
forces and no electrostatics.

\subsection{Test Case 23 and 24: Al with Sutton-Chen (metal) Potentials}

These systems consist of 32,000 and 256,000 atoms respectively.
Simulation at 300~K using NVT Evans ensemble with Sutton-Chen
forces and no electrostatics.

\subsection{Test Case 25 and 26: Al with EAM (metal) Potentials}

These systems consist of 32,000 and 256,000 atoms respectively.
Simulation at 300~K using NVT Evans ensemble with EAM tabulated
forces and no electrostatics.

\subsection{Test Case 27 and 28: NiAl alloy with EAM (metal) Potentials}

These systems consist of 27,648 and 221,184 atoms respectively.
Simulation at 300~K using NVT Evans ensemble with EAM tabulated
forces and no electrostatics.

\subsection{Test Case 29 and 30: Fe with Finnis-Sincair (metal) Potentials}

These systems consist of 31,250 and 250,000 atoms respectively.
Simulation at 300~K using NPT Berendsen ensemble with Finnis-Sinclair
forces and no electrostatics.

\subsection{Test Case 31 and 32: Ni with EAM (metal) Potentials}

These systems consist of 32,000 and 256,000 atoms respectively.
Simulation at 300~K using NPT Berendsen ensemble with EAM tabulated
forces and no electrostatics.

\subsection{Test Case 33 and 34: SPC IceVII water with constraints}

These systems consist of 11,664 (34,992 atoms) and 93,312 (279,936
atoms) water molecules respectively.  Simulation at 25~K using NVE
ensemble with CGM force minimisation and SPME electrostatics.  Both
constraint bond and rigid body dynamics cases are available.

\subsection{Test Case 35 and 36: NaCl molecules in SPC water represented as CBs+RBs}

These systems consist of 64 (512) NaCl ion pairs with 4,480 (35,840
) water molecules represented by constraint bonds and 4,416 (35,328)
water molecules represented by ridig bodies.  Totalling 26,816 (214,528)
atoms.  Simulation at 295~K using NPT Berendsen ensemble with CGM energy
minimisation and SPME electrostatics.

\subsection{Test Case 37 and 38: TIP4P water: RBs with a massless charged site}

These systems consist of 7,263 and 58,104 TIP4P rigid body water molecules
totaling 29,052 and 232,416 particles respectively.  Simulation at 295~K
using NPT Berendsen ensemble with CGM energy minimisation and SPME electrostatics.

\subsection{Test Case 39 and 40: Ionic liquid dimethylimidazolium chloride}

These systems consist of 44,352 and 354,816 ions respectively.  Simulation
at 400~K using NPT Berendsen ensemble, using both particle and rigid body
dynamics with SPME electrostatics.

\subsection{Test Case 41 and 42: Calcite nano-particles in TIP3P water}

In this case 600 and 4,800 molecules of calcium carbonate in the calcite
structure form 8 and 64 nano-particles which are suspended in 6,904 and
55,232 water molecules represented by a flexible 3-centre TIP3P model.
Simulation with SPME electrostatics at 310~K and 1~atmosphere maintained
in a Hoover NPT ensemble.  These systems consist of 23,712 and 189,696
ions respectively.

\subsection{Test Case 43 and 44: Iron/Carbon alloy with EEAM}

In this case a steel alloy of iron and carbon in ratio
35132 to 1651 is modelled using an EEAM potential forcefield.
Simulation at 1000~K and 0~atmosphere is maintained in a Berendsen NPT
ensemble.  These systems consist of 36,803 and 294,424
particles respectively.

\subsection{Test Case 45 and 46: Iron/Cromium alloy with 2BEAM}

In this case a steel alloy of iron and chromium in ratio
27635 to 4365 is modelled using an 2BEAM potential forcefield.
Simulation at 300~K and 0~atmosphere is maintained in an Evans NVT
isokinetic ensemble.  These systems consist of 32,000 and 256,000
particles respectively.

\section{Benchmark Cases}

\D benchmark test cases are avaliable to download them from the CCP5 FTP
server as follows:

\begin{verbatim}
FTP site : ftp.dl.ac.uk
Username : anonymous
Password : your email address
Directory: ccp5/DL_POLY/DL_POLY_4.0/BENCH
\end{verbatim}

The \D authors provide these on an "AS IS" terms.  For more information
refer to the README.txt file within.
