\section{The DL\_POLY Package}

DL\_POLY \cite{smith-96a} is a package of subroutines, programs and
data files, designed to facilitate molecular dynamics simulations of
macromolecules, polymers, ionic systems and solutions on a
distributed memory parallel computer. It is available in two forms:
\C (written by Bill Smith \& Tim Forester, \WEC\index{WWW}) and \D (written
by Ilian Todorov \& Bill Smith) \cite{todorov-04a,todorov-06a}.  Both versions
were originally written on behalf of CCP5\index{CCP5}, the UK's
Collaborative Computational Project on Molecular Simulation, which
has been in existence since 1980 (\cite{smith-87a}, \WEB\index{WWW}).

The two forms of DL\_POLY differ primarily in their method of
exploiting parallelism.  \C uses a Replicated Data (RD)
strategy \cite{smith-91a,smith-93a,smith-94a,smith-94b} which
works well simulations of up to 30,000 atoms on up to 100 processors.
\D is based on the Domain Decomposition (DD) strategy
\cite{todorov-04a,todorov-06a,pinches-91a,rapaport-91b,smith-91a,smith-93a},
and is best suited for large molecular simulations from $10^{3}$
to $10^{9}$ atoms on large processor counts.  The two packages
are reasonably compatible, so that it is possible to scale up from
a \C to a \D simulation with little effort.  It should
be apparent from these comments that \D is not intended as a
replacement for \C.

Users are reminded that we are interested in hearing what other
features could be usefully incorporated.  We obviously have ideas
of our own and CCP5\index{CCP5} strongly influences developments,
but other input would be welcome nevertheless.  We also request
that our users respect the integrity of \D source and not pass it
on to third parties.  We require that all users of the package
register with us, not least because we need to keep everyone
abreast of new developments and discovered bugs.  We have
developed various forms of licence\index{licence}, which we hope
will ward off litigation (from both sides), without denying access
to genuine scientific users.

Further information on the DL\_POLY packages may be obtained from
the DL\_POLY project website - \\ \noindent \WEB\index{WWW}.

\section{Functionality}

The following is a list of the features \D supports.

\subsection{Molecular Systems}

\D will simulate the following molecular species:

\begin{itemize}
\item Simple atomic systems and mixtures, e.g. Ne, Ar, Kr, etc.
\item Simple unpolarisable point ions, e.g. NaCl, KCl, etc.
\item Polarisable point ions and molecules, e.g. MgO, H$_{2}$O, etc.
\item Simple rigid molecules\index{rigid body} e.g. CCl$_{4}$, SF$_{6}$, Benzene, etc.
\item Rigid molecular\index{rigid body} ions with point charges e.g. KNO$_{3}$,
(NH$_{4}$)$_{2}$SO$_{4}$, etc.
\item Polymers with rigid bonds\index{constraints!bond}, e.g. C$_{n}$H$_{2n+2}$
\item Polymers with flexible and rigid bonds\index{constraints!bond} and point charges, e.g. proteins, macromolecules etc.
\item Silicate glasses and zeolites
\item Simple metals and metal alloys, e.g. Al, Ni, Cu, Cu$_{3}$Au, etc.
\item Covalent systems as hydro-carbons and transition elements, e.g. C, Si, Ge, SiC, SiGe, ets.
\end{itemize}

\subsection{Force Field\index{force field}}

The \D force field\index{force field!DL\_POLY} includes the
following features:
\begin{enumerate}
\item All common forms of non-bonded\index{potential!non-bonded} atom-atom (van der Waals) potentials
\item Atom-atom (and site-site) coulombic\index{potential!electrostatics} potentials
\item Metal-metal\index{potential!metal} (local density dependent) potentials \cite{baskes-84a,baskes-86a,finnis-84a,sutton-90a,sutton-91a,todd-93a}
\item Tersoff\index{potential!Tersoff} (local density dependent) potentials (for hydro-carbons) \cite{tersoff-89a}
\item Three-body\index{potential!three-body} valence angle\index{potential!valence angle} and hydrogen bond\index{potential!bond} potentials
\item Four-body\index{potential!four-body} inversion\index{potential!inversion} potentials
\item Ion core-shell polarasation\index{polarisation!shell model}
\item Tether\index{potential!tether} potentials
\item Chemical bond\index{potential!chemical bond} potentials
\item Valence angle\index{potential!valence angle} potentials
\item Dihedral angle\index{potential!dihedral} (and improper dihedral angle\index{potential!improper dihedral}) potentials
\item Inversion angle\index{potential!inversion} potentials
\item External field\index{potential!external field} potentials.
\end{enumerate}

\noindent The parameters describing these potentials may be
obtained, for example, from the GROMOS\index{force
field!GROMOS}\index{GROMOS} \cite{gunsteren-87a},
Dreiding\index{force field!Dreiding} \cite{mayo-90a} or
AMBER\index{force field!AMBER}\index{AMBER} \cite{weiner-86a}
forcefield, which share functional forms.  It is relatively easy
to adapt \D to user specific force fields\index{force field}.

\subsection{Boundary Conditions}
\index{boundary conditions}

\D will accommodate the following boundary conditions:
\begin{enumerate}
\item None, e.g. isolated molecules {\em in vacuo}
\item Cubic periodic boundaries
\item Orthorhombic periodic boundaries
\item Parallelepiped periodic boundaries
\item Slab (x,y periodic, z non-periodic).
\end{enumerate}

\noindent These are described in detail in Appendix
\ref{boundary-conditions}.  Note that periodic boundary conditions
(PBC) $1$ and $5$ above require careful consideration to enable
efficient load balancing on a parallel computer.

\subsection{Java Graphical User Interface}
\index{Java GUI}

The \D Graphical User Interface (GUI) is the same one that also comes with
\C, which is written in the Java\textregistered programming
language from Sun\textregistered Microsystems.  A major advantage
of this is the free availability of the Java programming environment
from Sun\textregistered, and also its portability across platforms.
The compiled GUI may be run without recompiling on any Java\textregistered
supported machine.  The GUI is an integral component of the DL\_POLY
suites and is available on the same terms (see the GUI manual \cite{smith-gui}).

\subsection{Algorithms\index{algorithm}}

\subsubsection{Parallel Algorithms\index{parallelisation}}

\D exclusively employs the {Domain Decomposition}
\index{parallelisation!Domain Decomposition} parallelisation
strategy \cite{pinches-91a,rapaport-91b,smith-91a,smith-93a} (see
Section~\ref{parallelisation}).

\subsubsection{Molecular Dynamics Algorithms}

\D offers a selection of MD integration algorithms\index{algorithm}
couched in both Velocity Verlet (VV) and Leapfrog Verlet (LFV)
manner \index{algorithm!Verlet} \index{algorithm} \cite{allen-89a}.
These generate NVE, NVE$_{kin}$, NVT, NPT and N\mat{\sigma}T
ensembles\index{ensemble} with a selection of
thermostats\index{thermostat} and barostats\index{barostat}.
Parallel versions of the RATTLE\index{algorithm!RATTLE}
\cite{andersen-83a} and SHAKE\index{algorithm!SHAKE}
\cite{smith-94b} algorithms are used for solving bond
constraints\index{constraints!bond} in the VV and LFV cast
integrations respectively.  The rotational motion of
rigid bodies\index{rigid body} (RBs) is handled with Fincham's
implicit quaternion\index{quaternions} algorithm\index{algorithm!FIQA}
(FIQA) \cite{fincham-92a} under the LFV scheme or with the
``NOSQUISH'' algorithm of Miller {\em et al} \index{algorithm!NOSQUISH}
\cite{miller-02a} under the VV integration.

The following MD algorithms\index{algorithm} are available:
\begin{enumerate}
\item Constant E algorithm\index{ensemble!NVE}
\item Evans\index{ensemble!Evans NVT} constant E$_{kin}$ algorithm \cite{evans-84a}
\item Langevin\index{ensemble!Langevin NVT} constant T algorithm \cite{adelman-76a}
\item Andersen\index{ensemble!Andersen NVT} constant T algorithm \cite{andersen-79a}
\item Berendsen\index{ensemble!Berendsen NVT} constant T algorithm \cite{berendsen-84a}
\item Nos\'{e}-Hoover\index{ensemble!Nos\'{e}-Hoover NVT} constant T algorithm \cite{hoover-85a}
\item Langevin\index{ensemble!Langevin NPT} constant T,P algorithm \cite{quigley-04a}
\item Berendsen\index{ensemble!Berendsen NPT} constant T,P algorithm \cite{berendsen-84a}
\item Nos\'{e}-Hoover\index{ensemble!Nos\'{e}-Hoover NPT} constant T,P algorithm \cite{hoover-85a}
\item Martyna, Tuckerman and Klein (MTK)\index{ensemble!Martyna-Tuckerman-Klein NPT} constant T,P algorithm \cite{martyna-96a}
\item Langevin\index{ensemble!Langevin N$\sigma$T} constant T,$\mat{\sigma}$ algorithm \cite{quigley-04a}
\item Berendsen\index{ensemble!Berendsen N$\sigma$T} constant T,$\mat{\sigma}$ algorithm \cite{berendsen-84a}
\item Nos\'{e}-Hoover\index{ensemble!Nos\'{e}-Hoover N$\sigma$T} constant T,$\mat{\sigma}$ algorithm \cite{hoover-85a}
\item Martyna, Tuckerman and Klein (MTK)\index{ensemble!Martyna-Tuckerman-Klein N$\sigma$T} constant T,$\mat{\sigma}$ algorithm \cite{martyna-96a}.
\end{enumerate}

\subsection{\C features incompatible or unavalable in \D\index{dlpoly2}}

\begin{itemize}
\item Force field
\begin{itemize}
\item Rigid bodies connected with constraint links are {\bf not available}
\item Shell models specification is {\bf solely} determined
by the presence of mass on the shells
\item Dihedral potentials with more than three {\em original}
parameters (see OPLS) have two artificially added parameters,
defining the 1-4 electrostatic and van der Waals scaling
factors, which {\bf must} be placed at fourth and fifth position
respectively, extending the original parameter list split by them
\end{itemize}
\item Boundary conditions
\begin{itemize}
\item Truncated octahedral periodic boundaries ({\tt imcon} = 4)
are {\bf not available}
\item Rhombic dodecahedral periodic boundaries ({\tt imcon} = 5)
are {\bf not available}
\item Hexagonal prism periodic boundaries ({\tt imcon} = 7)
are {\bf not available}
\end{itemize}
\item Electrostatics
\begin{itemize}
\item Standard Ewald Summation is {\bf not available}, but is
{\bf substituted} by Smoothed Particle Mesh Ewald (SPME) summation
\item Hautman-Klein Ewald Summation for 3D non-periodic
but 2D periodic systems is {\bf not available}
\end{itemize}
\item Non-standard functionality
\begin{itemize}
\item Temperature Accelerated Dynamics
\item Hyperdynamics
\item Solvation Energies
\end{itemize}
\end{itemize}

\section{Programming Style}

The programming style of \D is intended to be as uniform as
possible.  The following stylistic rules apply throughout.
Potential contributors of code are requested to note the stylistic
convention.

\subsection{Programming Language}

\D is written in free format FORTRAN90\index{FORTRAN90}.  In \D
we have adopted the convention of {\em explicit type declaration}
i.e. we have used
\begin{lstlisting}

             Implicit None

\end{lstlisting}
in all subroutines.  Thus all variables must be given an explicit
type: {\tt Integer, Real( Kind = wp)}, etc.

\subsection{Modularisation and Intent}

\D exploits the full potential of the modularisation concept in
FORTRAN90\index{FORTRAN90}.  Variables having in common description of certain
feature or method in \D are grouped in modules.  This simplifies
subroutines' calling sequences and decreases error-proneness in
programming as subroutines must define what they use and from
which module.  To decrease error-proneness further, arguments that
are passed in calling sequences of functions or subroutines have
defined intent, i.e. whether they are to be:
\begin{itemize}
\item passed in only ({\tt Intent (In)}) - the argument is not
allowed to be changed by the routine
\item passed out only ({\tt Intent (Out)}) - the ``coming in''
value of the argument is unimportant
\item passed in both directions in and out ({\tt Intent (InOut)})
- the ``coming in'' value of the argument is important and the
argument is allowed to be changed.
\end{itemize}

\subsection{Memory Management}

\D exploits the dynamic array allocation features of FORTRAN90
\index{FORTRAN90} to assign the necessary array dimensions.

\subsection{Target Platforms}

\D is intended for distributed memory parallel computers.

Compilation of \D in parallel mode requires {\bf only} a FORTRAN90
\index{FORTRAN90} compiler and Message Passing Interface (MPI) to handle
communications.  Compilation of \D in serial mode is also possible
and requires {\bf only} a FORTRAN90 \index{FORTRAN90} compiler.

\subsection{Internal Documentation}

All subroutines are supplied with a header block of FORTRAN90
\index{FORTRAN90} comment (!) records giving:
\begin{enumerate}
\item The name of the author and/or modifying author
\item The version number or date of production
\item A brief description of the function of the subroutine
\item A copyright statement.
\end{enumerate}

\noindent Elsewhere FORTRAN90\index{FORTRAN90} comment cards (!)
are used liberally.

\subsection{FORTRAN90 Parameters and Arithmetic Precision}
\label{precision}

All global parameters defined by the FORTRAN90\index{FORTRAN90}
parameter statements are specified in the
module file: {\sc setup\_module}, which is included at compilation
time in all subroutines requiring the parameters.  All parameters
specified in {\sc setup\_module} are described by one or more
comment cards.

One super-global parameter is defined at compilation time in the
{\sc kinds\_f90} module file specifying the working precision ({\tt
wp}) by kind for real and complex variables and parameters.  The
default is 64-bit (double) precision, i.e. {\tt Real(wp)}.  Users
wishing to compile the code with quadruple precision must ensure
that their architecture and FORTRAN90 compiler can allow that and
then change the default in {\sc kinds\_f90}.  Changing the precision
to anything else that is allowed by the FORTRAN90 compiler and the
machine architecture must also be compliant with the MPI working
precision {\tt mpi\_wp} as defined in {\sc comms\_module} (in such
cases users must correct for that in there).

\subsection{Units}\label{units}

Internally all \D subroutines and functions assume the use of the
following defined {\em molecular units}\index{units!DL\_POLY}:
\begin{itemize}
\item The unit of time ($t_{o}$) is $1 \times 10^{-12}$ seconds (i.e. picoseconds)
\item The unit of length ($\ell_{o}$) is $1 \times 10^{-10}$ metres (i.e. \AA ngstroms)
\item The unit of mass ($m_{o}$) is $1.6605402 \times 10^{-27}$ kilograms (i.e. Daltons - atomic mass units)
\item The unit of charge ($q_{o}$) is $1.60217733 \times 10^{-19}$ Coulombs (i.e. electrons - units of proton charge)
\item The unit of energy ($E_{o}=m_{o}(\ell_{o}/t_{o})^{2}$) is $1.6605402 \times 10^{-23}$~Joules (10~J~mol$^{-1}$)
\item The unit of pressure\index{units!pressure} (${\cal P}_{o}=E_{o}\ell_{o}^{-3}$) is $1.6605402 \times 10^{7}$~Pascals ($163.882576$~atmospheres)
\item Planck's constant ($\hbar$) which is $6.350780668 \times E_{o} t_{o}$~~.
\end{itemize}

\noindent In addition, the following conversion factors are used:

\begin{itemize}
\item The coulombic\index{potential!electrostatics} conversion
factor ($\gamma_{o}$) is:
\[ \gamma_{o} = \frac{1}{E_{o}} \left[ \frac{q_{o}^{2}}{4 \pi \epsilon_{o} \ell_{o}} \right] = 138935.4835~~, \]
such that:
\[U_{\tt MKS}=E_{o}\gamma_{o}U_{\tt Internal}~~,\]
where $U$ represents the configuration energy.
\item The Boltzmann factor ($k_{B}$) is $0.831451115~E_{o}$~K$^{-1}$,
such that:
\[T=E_{kin}/k_{B}\]
represents the conversion from kinetic energy (in internal units)
to temperature (in Kelvin).
\end{itemize}

\noindent {\bf Note:} In the \D OUTPUT file, the print out of
pressure\index{units!pressure} is in units of katms
(kilo-atmospheres) at all times.  The unit of energy is either
DL\_POLY units specified above, or in other units specified by the
user at run time (see Section~\ref{field-file}).  The default is the
DL\_POLY unit.

Externally, \D accepts information in its own specific formatting
as described in Section~\ref{input-files}.  Irrespective of formatting
rules, all values provided to define input entities are read in
DL\_POLY units (except otherwise specified as in the case of energy
units) or their composite mixture representing the corresponding
entity physically, i.e. velocities' components are in \AA ngsroms/picosecond.

\noindent {\bf Exception:} It should be noted that when \D is used
in a DPD mode (see Section~\ref{dpd} and Appendix~\ref{DPD-all})
then the meaning of the molecular units is somewhat lost and it is
only the interrelationship between units that is important (which
can be exploited by the modeller)!  The fundamental units for a DPD
simulation are related those of mass $[M]$, length $[L]$ and energy
$[E]$ - all irrespectively of the actually chosen energy units by
the {\bf UNITS} directive in the FIELD file.  Therefore, the DPD
unit of time is equivalent to $[L]\sqrt{[M]/[E]}$ while temperature
(in the form $k_{B}T$) is defined as two-thirds of the kinetic
energy of the system's particles.  Similarly, volume is in units
of $[L]^{3}$ and pressure in $[E]/[L]^{3}$.

\subsection{Error Messages}

All errors detected by \D during run time initiate a call to the
subroutine {\sc error}, which prints an error message in the
standard output file and terminates the program.  All terminations
of the program are global (i.e. every node of the parallel
computer will be informed of the termination condition and stop
executing).

In addition to terminal error messages, \D will sometimes print
warning messages.  These indicate that the code has detected
something that is unusual or inconsistent.  The detection is
non-fatal, but the user should make sure that the warning does
represent a harmless condition.

\section{Directory Structure}
\label{directory-structure}

The entire \D package is stored in a UNIX directory structure. The
topmost directory is named {\em dl\_poly\_4.nn}, where {\em nn} is
a generation number.  Beneath this directory are several
sub-directories named:
{\em manual\index{sub-directory!manual}},
{\em source\index{sub-directory!source}},
{\em build\index{sub-directory!build}},
{\em cmake\index{sub-directory!cmake}},
{\em utils\index{sub-directory!utils}},
{\em execute\index{sub-directory!execute}},
{\em data\index{sub-directory!data}},
{\em bench\index{sub-directory!bench}},
{\em java\index{sub-directory!java}}, and
{\em utility\index{sub-directory!utility}}.

Briefly, the content of each sub-directory is as follows:
\begin{tabbing}
XXXXXXXXXXXXXXX\= XXXXXXXX\= \kill
sub-directory\> ~ \> contents \\
\> ~ \> \\
{\em manual}  \> \D main user manual and \D Java GUI manual \\
{\em source}  \> primary subroutines for the \D package \\
{\em build}   \> makefiles to assemble and compile \D source \\
{\em cmake}   \> contains files needed for \D {\sl cmake} build system \\
{\em utils}   \> contains a series of scripts needed for testing \\
{\em execute} \> the \D run-time directory \\
{\em data}    \> example input and output files for \D \\
{\em bench}   \> large test cases suitable for benchmarking \\
{\em java}    \> directory of Java and FORTRAN routines for the Java GUI \\
{\em utility} \> directory of routines donated by \D users.
\end{tabbing}

\noindent A more detailed description of each sub-directory follows.

\subsection{The {\em source} Sub-directory}

In this sub-directory all the essential source code for \D,
excluding the utility software is stored.  In keeping with the
`package' concept of \D, it does not contain any complete
programs; these are assembled at compile time using an appropriate
makefile.  The subroutines in this sub-directory are documented in
Chapter~\ref{source}.

\subsection{The {\em build} Sub-directory}

This sub-directory contains legacy makefiles for the
creation (i.e. compilation and linking) of the \D simulation
program.  The makefiles supplied select the appropriate
subroutines from the {\em source} sub-directory and deposit the
executable program in the {\em execute} directory.  Building \D
by using these legacy makefiles is described in Section~\ref{make}.

\subsection{The {\em cmake} Sub-directory}

This sub-directory contains necessary scripts and information
needed for the \D CMake system.  Building \D with {\sl cmake}
is described in Section~\ref{cmake}.

\subsection{The {\em utils} Sub-directory}

This sub-directory contains a framework of scripts needed by \D
developers for testing purposes.  The general user is welcome to
look and learn from it.  The scripts are the documentation themselves.

\subsection{The {\em execute} Sub-directory}

In the supplied version of \D, this sub-directory contains only a
few macros for copying and storing data from and to the {\em data}
sub-directory and for submitting programs for execution (see
Appendix~\ref{macros}).  However, if the \D program is assembled
by using a legacy makefile, the executable will be placed in this
sub-directory and could be used from here.  Then output files from
a job run in here will also appear here, so users may find it
convenient to use this sub-directory as originally intended.
(The experienced user is not at all required to use \D this way however.)

\subsection{The {\em data} Sub-directory}

This sub-directory contains examples of input and output files for
testing the released version of \D.  The examples of input data
are copied into the {\em execute} sub-directory when a program is
being tested.  The test cases are documented in Chapter~\ref{data}.
Note that these are no longer within the distribution of any
DL\_POLY version but are made available on-line at the DL\_POLY FTP
- \FTP\index{FTP}.

\subsection{The {\em bench} Sub-directory}

This directory contains examples of input and output data for \D
that are suitable for benchmarking \D on large scale computers.
These are described in Chapter~\ref{data}. Note that these are no
longer within the distribution of any DL\_POLY version but
are made available on-line at the DL\_POLY FTP - \FTP\index{FTP}.

\subsection{The {\em java} Sub-directory}
\index{Java GUI}

The \D Java Graphical User Interface (GUI)\index{GUI} is based on
the Java language developed by Sun.  The Java source code for this
GUI is to be found in this sub-directory.  The source is complete
and sufficient to create a working GUI, provided the user has installed
the Java Development Kit, (1.7 or above) which is available free
from Sun at \href{http://java.sun.com/}{http://java.sun.com/}\index{WWW}.
The GUI, once compiled, may be executed on any machine where Java
is installed \cite{smith-gui}.

\subsection{The {\em utility} Sub-directory}

This sub-directory contains assorted routines donated by DL\_POLY
users.  Potential users should note that these routines are {\bf
unsupported} and come {\bf without any guarantee or liability
whatsoever}.  They should be regarded as potentially useful
resources to be hacked into shape as needed by the user.  Some of
the various routines in this sub-directory are documented in the \C
User Manual.  Users who devise their own utilities are advised
to store them in the {\em utility} sub-directory.

\section{Obtaining the Source Code}
\label{distribution} \index{user registration} \index{\D software licence}

To obtain a copy of \D it is necessary to have internet connection.
Log on to the DL\_POLY website - \WEB\index{WWW}, and follow the
links to the \D registration page, where you will firstly be shown the
\D academic software licence (see Appendix~\ref{licence}), which details
the terms and conditions under which the code will be supplied.  {\bf By
proceeding further with the registration and download process you
are signalling your acceptance of the terms of this licence.}  Click
the `Registration' button to find the registration page, where you
will be invited to enter your name, address and e-mail address.  The
code is supplied free of charge to {\bf academic} users, but
{\bf commercial} users will be required to purchase a software licence.

Once the online registration has been completed, information
on downloading the \D source code will be sent by e-mail, so {\bf
it is therefore essential to supply a correct e-mail address}.

The {\em data} and {\em bench} subdirectories of \D are not issued
in the standard package, but can be downloaded directly from the
FTP site (in the \mbox{ccp5/DL\_POLY/DL\_POLY\_4.0/} directory).

\noindent {\bf Note:} Daresbury Laboratory is the {\bf sole centre} for
the distribution of \D and copies obtained from elsewhere will be
regarded as illegal and will not be supported.

\section{OS and Hardware Specific Ports}
%
%\D is available as a Microsoft port, offered with Microsoft\textregistered
%(\href{http://www.microsoft.com/}{http://www.microsoft.com/})\index{WWW}
%self-installers (MSI) for 32- and 64-bit Windows OS's to build
%an OS native executable, which can utilise the parallelism of
%modern multi-core/multi-processor personal computers.
%
%\D is also available as a CUDA+OpenMP port, offered as extra
%source within the {\em source} directory (see the README.txt
%for further information).  The purpose of this development,
%a collaboration with the Irish Centre for High-End Computing
%(ICHEC - \href{http://www.ichec.ie/}{http://www.ichec.ie/}),
%is to harness the power offered by NVIDIA\textregistered
%(\href{http://www.nvidia.com/}{http://www.nvidia.com/}\index{WWW}) GPUs.

{\bf Note that no support is offered for these highly specific
developments!}

\section{Other Information}\label{other}

The DL\_POLY website - \WEB\index{WWW}, provides additional information in the form of
\begin{enumerate}
\item Access to all documentation (including licences)
\item Frequently asked questions
\item Bug reports
\item Access to the DL\_Software portal.
\end{enumerate}

\noindent
Daresbury Laboratory also maintains a \D associated electronic mailing
list, {\em dl\_poly\_4\_news}, to which all registered \D users
are automatically subscribed.  It is via this list that error reports
and announcements of new versions are made.  If you are a \D
user, but not on this list you may request to be added by sending
a mail message to {\href{mailto:majordomo@dl.ac.uk}{majordomo@dl.ac.uk}}
with the one-line message: $subscribe~dl\_poly\_4\_news$.

The DL\_Software {\bf Portal} is a web based centre for all DL\_POLY users
to exchange comments and queries.  You may access the forum through the
DL\_POLY website.  A registration (and vetting) process is required
before you can use the forum, but it is open, in principle, to everyone.
